%! TeX root = /Users/trustinnguyen/Downloads/Berkeley/Math/Math143/Homework/Math143Hw11/Math143Hw11.tex

\documentclass{article}
\usepackage{/Users/trustinnguyen/.mystyle/math/packages/mypackages}
\usepackage{/Users/trustinnguyen/.mystyle/math/commands/mycommands}
\usepackage{/Users/trustinnguyen/.mystyle/math/environments/article}
\graphicspath{{./figures/}}

\title{Math143Hw11}
\author{Trustin Nguyen}

\begin{document}

    \maketitle

\reversemarginpar

\textbf{Exercise 1}: Let $P_{1}, P_{2}, P_{3}$ (resp. $Q_{1}, Q_{2}, Q_{3}$) be three points in $\mathbb{P}^{2}$ not lying on a line. Show that there is a projective change of coordinates $T : \mathbb{P}^{2} \rightarrow \mathbb{P}^{2}$ such that $T(P_{i}) = Q_{i}$, for $i = 1, 2, 3$.
    \begin{proof}
        We want to solve for a transformation matrix $A = \begin{bmatrix}
            a_{1, 1} & a_{1, 2} & a_{1, 3} \\
            a_{2, 1} & a_{2, 2} & a_{2, 3} \\
            a_{3, 1} & a_{3, 2} & a_{3, 3}   
        \end{bmatrix}$ such that:
            \begin{equation*}
                \begin{bmatrix}
                    a_{1, 1} & a_{1, 2} & a_{1, 3} \\
                    a_{2, 1} & a_{2, 2} & a_{2, 3} \\
                    a_{3, 1} & a_{3, 2} & a_{3, 3}   
                \end{bmatrix} \begin{bmatrix}
                    \mid   &  \mid &  \mid \\
                     P_{1} & P_{2} & P_{3} \\
                    \mid   &  \mid &  \mid   
                \end{bmatrix} = \begin{bmatrix}
                    \mid  &  \mid &  \mid \\
                    Q_{1} & Q_{2} & Q_{3} \\
                    \mid  &  \mid &  \mid   
                \end{bmatrix}
            \end{equation*}
       Suppose that $P = \begin{bmatrix}
            \mid   &  \mid &  \mid \\
             P_{1} & P_{2} & P_{3} \\
            \mid   &  \mid &  \mid   
        \end{bmatrix}$ is not invertible. Then we have a linear dependence of the rows of $P$. Let $R_{1}, R_{2}, R_{3}$ be the rows of $P$. Then
            \begin{equation*}
                a_{1}R_{1} + a_{2}R_{2} + a_{3}R_{3} = \begin{bmatrix}
                    0 & 0 & 0   
                \end{bmatrix}
            \end{equation*}
        and $P_{1}, P_{2}, P_{3}$ all satisfy the solution to the equation $a_{1}x_{1} + a_{2}x_{2} + a_{3}x_{3} = 0$ which is the equation of a line. So they lie on the same line. Then since the points are not on the same line, $P$ is invertible. So $A = QP^{-1}$. It is a change of coordinates because $A$ is invertible also as $A^{-1} = PQ^{-1}$ since $Q$ is invertible by the same reasoning as above.
   \end{proof}

Extra Credit: Extend this to $n + 1$ points in $\mathbb{P}^{n}$, not lying on a hyperplane.
    \begin{proof}
        This can be extended to $n + 1$ points by solving for $A$ in the system:
            \begin{equation*}
                \begin{bmatrix}
                    a_{1, 1}      & a_{1, 2}     &  \cdots &  a_{1, n + 1}     \\
                    a_{2, 1}      & a_{2, 2}     &  \cdots &  a_{2, n + 1}     \\
                    \vdots        &  \vdots      &  \ddots &  \vdots           \\
                     a_{n + 1, 1} & a_{n + 1, 2} &  \cdots &  a_{n + 1, n + 1}   
                \end{bmatrix} \begin{bmatrix}
                    \mid   &  \mid &         &  \mid      \\
                     P_{1} & P_{2} &  \cdots &  P_{n + 1} \\
                    \mid   &  \mid &         &  \mid        
                \end{bmatrix} = \begin{bmatrix}
                    \mid   &  \mid &        &  \mid      \\
                     Q_{1} & Q_{2} & \cdots &  Q_{n + 1} \\
                    \mid   &  \mid &        &  \mid        
                \end{bmatrix}
            \end{equation*}
        We find that the rows are linearly independent if $P$ is not invertible, so
            \begin{equation*}
                a_{1}R_{1} + \cdots + a_{n + 1}R_{n + 1} = \begin{bmatrix}
                    0 & 0 & \cdots &  0   
                \end{bmatrix}
            \end{equation*}
        and this shows that $P_{1}, \ldots, P_{n + 1}$ satisfy the solution to the hyperplane $a_{1}x_{1} + \cdots + a_{n + 1}x_{n + 1} = 0$. Therefore, by contrapositive, $P, Q$ are invertible, $A = QP^{-1}$, $A^{-1} = PQ^{-1}$.
    \end{proof}

\textbf{Exercise 2}: Duals. Let $ \Lambda = \mathbb{V}(a_{1}x_{1} + \cdots + a_{n + 1}x_{n + 1}) \subseteq \mathbb{P}^{n}$ with $a_{1}, \ldots, a_{n + 1} \in k$ not all zero. Recall that we call $\Lambda$ a \textit{hyperplane}. Note that $(a_{1}, \ldots, a_{n + 1})$ is determined by $\Lambda$ up to rescaling. As discussed in class, assigning $[a_{1} : \cdots : a_{n + 1}] \in \mathbb{P}^{n}$ to $\Lambda$ sets up a one-to-one correspondence between $\{\text{hyperplanes in $\mathbb{P}^{n}$}\}$ and $\mathbb{P}^{n}$.
    \begin{itemize}
        \item [(a)] Given $P = [a_{1} : \cdots : a_{n + 1}] \in \mathbb{P}^{n}$, write $P^{*} = \mathbb{V}(a_{1}x_{1} + \cdots + a_{n + 1}x_{n + 1})$ for the corresponding hyperplane; if $\Lambda$ is a hyperplane, let $\Lambda^{*}$ denote the corresponding point. Prove that $(P^{*})^{*} = P$ and $(\Lambda^{*})^{*} = \Lambda$.
            \begin{proof}
                First, we have
                    \begin{align*}
                        (P^{*})^{*} &= (\mathbb{V}(a_{1}x_{1} + \cdots + a_{n + 1}x_{n + 1}))^{*} \\
                                    &= [a_{1} : \cdots : a_{n + 1}]                                 
                    \end{align*}
                and then 
                    \begin{align*}
                            (\Lambda^{*})^{*} &= ((\mathbb{V}(a_{1}x_{1} + \cdots + a_{n + 1}x_{n + 1}))^{*})^{*} \\
                                                                  &= ([a_{1} : \cdots : a_{n + 1}])^{*}                               \\
                                                                  &= \mathbb{V}(a_{1}x_{1} + \cdots + a_{n + 1}x_{n + 1})             \\
                                                                  &= \Lambda                                                            
                    \end{align*}
                which completes the proof.
            \end{proof}

        \item [(b)] Show that $P \in \Lambda$ if and only if $ \Lambda^{*} \in P^{*}$. 
            \begin{proof}
                ($\rightarrow$) Suppose that $P \in \Lambda$. Then $P \in \mathbb{V}(a_{1}x_{1} + \cdots + a_{n + 1}x_{n + 1})$. So we have $P = [p_{1} : \cdots : p_{n + 1}]$ such that 
                    \begin{equation*}
                        a_{1}p_{1} + \cdots + a_{n + 1}p_{n + 1} = 0
                    \end{equation*}
                We then have $\Lambda^{*} = [a_{1} : \cdots : a_{n + 1}]$ and $P^{*} = \mathbb{V}(p_{1}x_{1} + \cdots + p_{n + 1}x_{n + 1})$. Then we have that $\Lambda^{*} \in P^{*}$ because
                    \begin{equation*}
                        p_{1}a_{1} + \cdots + p_{n + 1}a_{n + 1} = a_{1}p_{1} + \cdots + a_{n + 1}p_{n + 1} = 0
                    \end{equation*}
                ($\leftarrow$) Suppose that $\Lambda^{*} \in P^{*}$. Then by the previous direction, we have that $(P^{*})^{*} \in ( \Lambda^{*})^{*}$. Also by part $(a)$, we have $P \in \Lambda$ which concludes the proof.
            \end{proof}
    \end{itemize}

\textbf{Exercise 3}: Assume $k$ infinite. Show that for any finite set of points $P_{1}, \ldots, P_{r} \in \mathbb{P}^{2}$, there exists a line $L \subseteq \mathbb{P}^{2}$ that does not pass through any of them.
    \begin{proof}
        We require that for all $i$, $P_{i} \notin \Lambda$, which would indicate that the line $\Lambda$ does not pass through $P_{i}$. By the previous problem, this is equivalent to saying that $\Lambda^{*} \notin P^{*}_{i}$ for all $i$. So we require 
            \begin{equation*}
                P_{1}^{*}  \cup P_{2}^{*} \cup \cdots \cup P_{r}^{*} \neq \mathbb{P}^{2}
            \end{equation*}
        Since $k$ is infinite, $\mathbb{A}^{2} \subseteq \mathbb{P}^{2}$ is infinite. Each $P^{*}_{i}$ contains only one point, because we have:
            \begin{equation*}
                P_{i}^{*} = \mathbb{V}(a_{1}x_{1} + a_{2}x_{2})
            \end{equation*}
        So we have:
            \begin{equation*}
                \{[x_{1} : x_{2}] : a_{1}x_{1} + a_{2}x_{2} = 0\}
            \end{equation*}
        Then 
            \begin{equation*}
                x_{2} = \begin{cases}
                    \dfrac{-a_{2}}{a_{1}} &\text{ if } a_{1} \neq 0 \\
                    0 &\text{ if } a_{1} = 0               
                \end{cases}
            \end{equation*}
        So 
            \begin{equation*}
                P_{i}^{*} = \{[1 : \dfrac{-a_{2}}{a_{1}}]\} \text{ or } \{[0 : 1]\}
            \end{equation*}
        Therefore, 
            \begin{equation*}
                P_{1}^{*}  \cup P_{2}^{*} \cup \cdots \cup P_{r}^{*} \neq \mathbb{P}^{2}
            \end{equation*}
        Therefore, we can find some $\Lambda^{*} \notin P_{i}^{*}$. So $P_{i} \notin \Lambda$ as desired.
    \end{proof}

\textbf{Exercise 4}: Veronese embeddings.
    \begin{itemize}
        \item [(a)] Prove that Veronese embedding $\nu_{1, 3} : \mathbb{P}^{1} \rightarrow \mathbb{P}^{3}$ given by $[s : t] \mapsto [s^{3} : s^{2}t : st^{2} : t^{3}]$ is an isomorphism onto its image.
            \begin{proof}
                Consider the mapping:
                    \begin{equation*}
                        [x_{1} : x_{2} : x_{3} : x_{4}] \mapsto \begin{cases}
                            [x_{1} : x_{2}] &\text{ if } x_{1} \neq 0 \text{ over } (U_{1} \cap \varphi (\mathbb{P}^{1}))\\
                            [x_{3} : x_{4}] &\text{ if } x_{4} \neq 0 \text{ over } (U_{4} \cap \varphi (\mathbb{P}^{1}))
                        \end{cases}
                    \end{equation*}
                We first need to check that our morphism uses all the points in $\varphi (\mathbb{P}^{1})$. Consider the complement of $(U_{1} \cap \varphi(\mathbb{P}^{1})) \cup (U_{4} \cap \varphi (\mathbb{P}^{1}))$. These are elements in the image where $x_{1} =  0$ and $x_{4} = 0$. Then that would mean that for $[s^{3} : s^{2}t : st^{2} : t^{3}] \in \varphi (\mathbb{P}^{1})$, we have $s = 0$ and $t = 0$. Then that is the empty set because $[0 : 0 : 0 : 0] \notin \mathbb{P}^{3}$. So this is a morphism of projective algebraic sets.

                Well defined: Suppose that we had a point $[s^{3} : s^{2}t : st^{2} : t^{3}]$ such that $s \neq 0$ and $t \neq 0$. Then
                    \begin{align*}
                        \varphi^{-1}([s^{3} : s^{2}t : st^{2} : t^{3}]) &= [s^{3} : s^{2}t] = [s : t]  \\
                                                                        &= [st^{2} : t^{3}] = [s : t]
                    \end{align*}

                ($\varphi (\varphi^{-1}) = id_{\mathbb{P}^{1}}$) We have
                    \begin{align*}
                        \varphi (\varphi^{-1}([s^{3} : s^{2}t : st^{2} : t^{3}])) &= \varphi ([s : t])                 \\
                                                                                  &= [s^{3} : s^{2}t : st^{2} : t^{3}]   
                    \end{align*}

                ($\varphi^{-1}(\varphi) = id_{\mathbb{P}^{3}}$) We have
                    \begin{align*}
                        \varphi^{-1}(\varphi ([s : t])) &= \varphi^{-1}([s^{3} : s^{2}t : st^{2}: t^{3}]) \\
                                                        &= [s : t]                                          
                    \end{align*}

                Which finishes the proof.
            \end{proof}

        \item [(b)] The Veronese embedding $\nu_{3, 2} : \mathbb{P}^{3} \rightarrow \mathbb{P}^{9}$ is given by
            \begin{equation*}
                [x_{1} : x_{2} : x_{3} : x_{4}] \mapsto [x_{1}^{2} : x_{1}x_{2} : x_{1}x_{3} : x_{1}x_{4} : x_{2}^{2} : x_{2}x_{3} : x_{2}x_{4} : x_{3}^{2}: x_{3}x_{4} : x_{4}^{2}]
            \end{equation*}
        Let $z_{1}, \ldots, z_{10}$ be the coordinates on $\mathbb{P}^{9}$. Find a matrix $M$ whose entries are polynomials in the $z_{i}$ so that the image of $\nu_{3, 2}$ is the set of points in $\mathbb{P}^{9}$ where the matrix $M$ has rank $\leq k$ for some integer $k$. Justify your answer. (This implies that the equations that define the image are the $(k + 1) \times (k + 1)$ minors of $M$.)
            \begin{proof}
                First is calculating the inverse morphism:
                    \begin{equation*}
                        [z_{1} : z_{2} : \cdots : z_{9} : z_{10}] \mapsto \begin{cases}
                            [z_{1} : z_{2} : z_{3} : z_{4}] &\text{ if } z_{1} \neq 0 \text{ over } U_{1} \cap \Im{\nu_{3, 2}} \\
                            [z_{2} : z_{5} : z_{6} : z_{7}] &\text{ if } z_{5} \neq 0 \text{ over } U_{5} \cap \Im{\nu_{3, 2}} \\
                            [z_{3} : z_{6} : z_{8} : z_{9}] &\text{ if } z_{8} \neq 0 \text{ over } U_{8} \cap \Im{\nu_{3, 2}} \\
                            [z_{4} : z_{7} : z_{9} : z_{10}] &\text{ if } z_{10} \neq 0 \text{ over } U_{10} \cap \Im{\nu_{3, 2}}   
                        \end{cases}
                    \end{equation*}
                Then the image is the set of points such that the mapping is well-defined:
                    \begin{equation*}
                        I = \left\{[z_{1} : z_{2} : \cdots : z_{9} : z_{10}] : \mathop{rank}\begin{bmatrix}
                            z_{1} & z_{2} & z_{3} & z_{4}  \\
                            z_{2} & z_{5} & z_{6} & z_{7}  \\
                            z_{3} & z_{6} & z_{8} & z_{9}  \\
                            z_{4} & z_{7} & z_{9} & z_{10}   
                        \end{bmatrix} = 1\right\} = \mathbb{V}(2 \times 2 \text{ minors})
                    \end{equation*}
                We see that the map well-defined, because if $[z_{1} : \cdots : z_{10}]$ satisfy all conditions, we have:
                    \begin{align*}
                        [z_{1} : \cdots : z_{10}] &\mapsto \begin{cases}
                            [x_{1}^{2} : x_{1}x_{2} : x_{1}x_{3} : x_{1}x_{4}] &\text{ if } z_{1} \neq 0 \\
                            [x_{1}x_{2} : x_{2}^{2} : x_{2}x_{3} : x_{2}x_{4}] &\text{ if } z_{2} \neq 0 \\
                            [x_{1}x_{3} : x_{2}x_{3} : x_{3}^{2} : x_{3}x_{4}] &\text{ if } z_{3} \neq 0 \\
                            [x_{1}x_{4} : x_{2}x_{4} : x_{3}x_{4} : x_{4}^{2}] &\text{ if } z_{4} \neq 0   
                        \end{cases}                   \\
                                                  &=       \begin{cases}
                                                      [x_{1} : x_{2} : x_{3} : x_{4}] &\text{ if } z_{1} \neq 0 \\
                                                      [x_{1} : x_{2} : x_{3} : x_{4}] &\text{ if } z_{2} \neq 0 \\
                                                      [x_{1} : x_{2} : x_{3} : x_{4}] &\text{ if } z_{3} \neq 0 \\
                                                      [x_{1} : x_{2} : x_{3} : x_{4}] &\text{ if } z_{4} \neq 0   
                                                  \end{cases}                   \\
                                                  &=       [x_{1} : x_{2} : x_{3} : x_{4}]   
                    \end{align*}
                This generalizes if it satisfies less than all $4$ conditions.

                We need to check composition of morphisms. Referring to the previous shown mapping of $[z_{1} : \cdots : z_{10}] \mapsto \text{stuff}$, we compose that with $v_{3, 2}$ to get:
                    \begin{align*}
                        \nu_{3, 2}([x_{1} : x_{2} : x_{3} : x_{4}]) &= \\
                        [x_{1} : x_{2} : x_{3} : x_{4}] &\mapsto [x_{1}^{2} : x_{1}x_{2} : x_{1}x_{3} : x_{1}x_{4} : x_{2}^{2} : x_{2}x_{3} : x_{2}x_{4} : x_{3}^{2}: x_{3}x_{4} : x_{4}^{2}] \\
                                                        &= [z_{1} : \cdots : z_{10}]
                    \end{align*}
                That is the composition $\nu_{3, 2}(\nu_{3, 2}^{-1}([z_{1} : \cdots : z_{10}]))$. For the other composition, we have
                    \begin{equation*}
                        [x_{1} : x_{2} : x_{3} : x_{4}] \mapsto [x_{1}^{2} : x_{1}x_{2} : x_{1}x_{3} : x_{1}x_{4} : x_{2}^{2} : x_{2}x_{3} : x_{2}x_{4} : x_{3}^{2}: x_{3}x_{4} : x_{4}^{2}]
                    \end{equation*}
                Then we see that it maps back to $[x_{1} : x_{2} : x_{3} : x_{4}]$ under $\nu_{3, 2}^{-1}$.

                The rank is less than or equal to $k$ because at least one of $z_{1}, z_{5}, z_{8}, z_{10}$ are non-zero. Then wlog, say that $z_{1} \neq 0$. For $z_{5}, z_{8}, z_{10}$, consider these cases similarly for each:
                    \begin{itemize}
                        \item $z_{i} = 0$. Then it follows that the entire row in the matrix is $0$ because $z_{i}$ is a factor of every element in that row. So we have a row linearly dependent to the first row.

                        \item $z_{i} \neq 0$. Then it follows that dividing the entire row by $z_{i}$ and multiplying by $z_{1}$ gives us the first row. So the rows are linearly dependent.
                    \end{itemize}
                So we can have at most $1$ linearly independent row, and we must have at least one linearly independent row. So $\mathop{rank} = 1$. This shows the inclusion of $\Im{\nu_{3, 2}^{-1}} \subseteq I$.
            \end{proof}

        \item [(c)] What is the preimage $\nu_{3, 2}^{-1}(\mathbb{V}(z_{1} + 4z_{3} - 2z_{7} + 5z_{9}))$? 
            \begin{answer}
                The preimage is obtained by replacing the $z_{1}, \ldots, z_{10}$ with what they correspond to in the image of $[x_{1}: x_{2} : x_{3} : x_{4}]$. So it is
                    \begin{equation*}
                        \mathbb{V}(x_{1}^{2} + 4x_{1}x_{3} - 2x_{2}x_{4} + 5x_{3}x_{4})
                    \end{equation*} 
            \end{answer}
    \end{itemize}

\textbf{Exercise 5}: Let $X, Y \subseteq \mathbb{A}^{2}$ be affine plane curves and let $\overline{X}$ and $\overline{Y}$ be their projective closures. For each of the following statements, prove or give a counter example.
    \begin{itemize}
        \item [(a)] If $\overline{X}$ and $\overline{Y}$ are projectively equivalent, then $X$ and $Y$ are isomorphic (as affine algebraic sets).
            \begin{answer}
                This is false. There is a projective equivalence between $\overline{X} = \mathbb{V}(x) \rightarrow \overline{Y} = \mathbb{V}(z)$ by 
                    \begin{equation*}
                        \begin{bmatrix}
                            0 & 1 & 0 \\
                            0 & 0 & 1 \\
                            1 & 0 & 0   
                        \end{bmatrix} \begin{bmatrix}
                            0 \\
                            y \\
                            z   
                        \end{bmatrix} = \begin{bmatrix}
                            y \\
                            z \\
                            0   
                        \end{bmatrix}
                    \end{equation*}
                But we have $\overline{V(x)} = \mathbb{V}(x)$ and $\overline{V(1)} = \mathbb{V}(z)$. But there is no isomorphism between $V(x) \rightarrow V(1) = \emptyset$.
            \end{answer}

        \item [(b)] If $X$ and $Y$ are isomorphic (as affine algebraic sets), then $\overline{X}$ and $\overline{Y}$ are projectively equivalent. 
            \begin{proof}
                Don't know
            \end{proof}
    \end{itemize}



















\end{document}
