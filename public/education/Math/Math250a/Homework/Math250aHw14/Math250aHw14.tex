%! TeX root = /Users/trustinnguyen/Downloads/Berkeley/Math/Math250a/Homework/Math250aHw14/Math250aHw14.tex

\documentclass{article}
\usepackage{/Users/trustinnguyen/.mystyle/math/packages/mypackages}
\usepackage{/Users/trustinnguyen/.mystyle/math/commands/mycommands}
\usepackage{/Users/trustinnguyen/.mystyle/math/environments/article}
\graphicspath{{./figures/}}

\title{Math250aHw14}
\author{Trustin Nguyen}

\begin{document}

    \maketitle

\reversemarginpar

\textbf{Exercise 1}: Let $a_{1}, \ldots, a_{ n}$ be square-free, relatively prime integers not equal to $1$ or $-1$. Show that $K = \mathbb{Q}(\sqrt{a_{1}}, \ldots, \sqrt{ a_{n}})/\mathbb{Q}$ is Galois, of Galois group $(\mathbb{Z}/2)^{n}$. Show that if $0 \leq k \leq  n$ then the number of subfield of $K$ of degree $2^{k}$ over $\mathbb{Q}$ is the same as the number of subfields of degree $2^{n  - k}$ and find this number.
    \begin{proof}
        We need to show that each extension is nontrivial in the tower:
            \begin{equation*}
                \mathbb{Q} \subseteq \mathbb{ Q}(\sqrt{a_{1}}) \subseteq \mathbb{ Q}(\sqrt{a_{1}}, \sqrt{a_{2}}) \subseteq \cdots \subseteq  \mathbb{ Q}(\sqrt{a_{1}}, \sqrt{a_{2}}, \ldots, \sqrt{a_{n}})
            \end{equation*}
        To show that each extension is nontrivial, we need to show that 
            \begin{equation*}
                \sqrt{p_{n}} \notin \mathbb{ Q}(\sqrt{p_{1}}, \sqrt{p_{2}}, \ldots, \sqrt{p_{n - 1}})
            \end{equation*}
        We get a basis of $\mathbb{Q}(\sqrt{p_{1}}, \ldots, \sqrt{p_{n - 1}})$ over $\mathbb{Q}$ by taking the product of any combination of the $\sqrt{p_{i}}$. We see that if $\sqrt{p_{n}} \in \mathbb{ Q}(\sqrt{p_{1}}, \ldots, \sqrt{p_{n - 1}})$, then it should be an element of the basis, which is not true, because the $p_{i}$ are relatively prime.

        Now we know that the intersection of each $\mathbb{Q}(\sqrt{a_{i}})$ pairwise is just $\mathbb{Q}$. Also $\mathbb{Q}(\sqrt{a_{i}})/\mathbb{Q}$ Galois. Then building $\mathbb{Q}(\sqrt{a_{1}}, \ldots, \sqrt{a_{n}})$ by taking the compositum inductively, we get that $\mathbb{Q}(\sqrt{a_{1}}, \ldots, a_{n})$ is Galois over $\mathbb{Q}$ and that the Galois group is just the product of the Galois groups of $\mathbb{Q}(\sqrt{a_{i}}) /\mathbb{Q}$ or $(\mathbb{Z}/2)^{2}$.

        The number of subfields of degree $2^{k}$ over $\mathbb{Q}$ is the same as the number of subfields of degree $2^{n - k}$ because they correspond to the same subgroup of the Galois group. The Galois group is abelian and the number of subfields is $\binom{n}{k}$.
    \end{proof}


\textbf{Exercise 2}: Let $E/\mathbb{Q}$ be the splitting field of $x^{5} - 2$. Show that :
    \begin{itemize}
        \item [(a)] $[E : \mathbb{Q}] = 20$.
            \begin{proof}
                We have that $ \mathbb{Q} \subseteq \mathbb{ Q}(\sqrt[5]{2})$ is a degree $5$ extension since $\sqrt[5]{2}$ is a root of the irreducible polynomial $x^{5} - 2$. We also know that $\zeta \sqrt[5]{2}, \zeta^{ 2} \sqrt[5]{2}, \zeta^{ 3}\sqrt[5]{2}, \zeta^{ 4}\sqrt[5]{2}$ are also roots that do not lie in $\mathbb{Q}(\sqrt[5]{2})$ because they are complex. So when we adjoin again $\zeta \sqrt[5]{2}$, we get $\mathbb{Q}(\sqrt[5]{2}, \zeta \sqrt[5]{2}) = \mathbb{Q}(\sqrt[5]{2}, \zeta)$. This gives all the roots and is the splitting field of $x^{5} - 2$ over $\mathbb{Q}$. The degree of $[\mathbb{Q}(\sqrt[5]{2}, \zeta), \mathbb{ Q}(\sqrt[5]{2})]$ is $4$ because we get all the roots of $\frac{x^{5} - 2}{x - \sqrt[5]{2}} \in \mathbb{ Q}(\sqrt[5]{2})$ irreducible. So the extension is of degree
                    \begin{equation*}
                        [E : \mathbb{Q}] = [\mathbb{Q}(\sqrt[5]{2}, \zeta) : \mathbb{Q}(\sqrt[5]{2})][\mathbb{Q}(\sqrt[5]{2}) : \mathbb{Q}] = 5 \cdot 4 = 20
                    \end{equation*}
            \end{proof}

        \item [(b)] there is precisely $1$ subfield $F$ of $E$ with $[E : F] = 5$, and that $F$ is normal.
            \begin{proof}
                $E/\mathbb{Q}$ is a Galois extension, with $[E : \mathbb{Q}]$ the order of the Galois group. Then we have for $F = E^{H}$:
                    \begin{align*}
                        \mathbb{Q} &\subseteq  E^{H} \subseteq E \\
                        G          &\supseteq H \supseteq \{ e\}   
                    \end{align*}
                So we are looking for the subgroups of order $5$. We have that the number of sylow-$5$ subgroups of $G$ is $1$ because there are either $1, 2, 4$ groups and the number is $\equiv 1\pmod{ 5}$. So we have one sylow $5$ group, that is normal in $G$. We conclude that $H$ is normal in $G$. So $F \subseteq \mathbb{Q}$ is normal. There are no other order $5$ subgroups of $G$, so there is only one intermediate field that it corresponds to with degree $5$ from $E/F$.
            \end{proof}

        \item [(c)] the element $2^{1/5} + \zeta$ is a primitive element of $E/\mathbb{Q}$, where $\zeta$ is the $5$th root of $1$. 
            \begin{proof}
                Don't know
            \end{proof}

        \item [(d)] $\Gal(E/\mathbb{Q})$ contains elements $\sigma, \tau$ of orders $5$ and $4$ respectively, with $\tau \sigma  \tau^{-1} = \sigma^{ 2}$ .
            \begin{proof}
                The element $\sigma$ has order $5$, sending $\sqrt[5]{2}$ to one of $\sqrt[5]{2}, \zeta \sqrt[5]{2}, \zeta^{ 2}\sqrt[5]{2}, \zeta^{ 3}\sqrt[5]{2}, \zeta^{ 4}\sqrt[5]{2}$. The element $\tau$ has order $4$ sending $\zeta$ to any root of unity except $1$. We know that the sylow $5$ group is normal so 
                    \begin{equation*}
                        \tau \sigma \tau^{-1} = \sigma^{ i}
                    \end{equation*}
                Couldn't finish.
            \end{proof}
    \end{itemize}

\textbf{Exercise 3}: Let $K = k(x)$ be the field of rational functions in $1$ variable over a field $k$. Show that for any $t = f/g \in K \backslash  k$, with $f, g$ relatively prime, the field extension $K/k(t)$ is finite of degree $\max(\deg f, \deg g)$. Show that any automorphism $\sigma$ of $K/k$ is determined by the image of $x$, and has the form 
    \begin{equation*}
        \sigma ( X) = \dfrac{ax + b}{cx + d}
    \end{equation*}
with $ad - bc \neq 0$. Show that the numbers $a, b, c, d$ are well-defined up to a common scalar multiple, and $\Gal(K/k) \cong PGL( 2, k)$ (invertible $2 \times 2$ matrices mod scalars).
    \begin{proof}
        We have that 
            \begin{equation*}
                tg - f = 0
            \end{equation*}
        is a polynomial in $k(t)$ that kills $x$. It is also irreducible because. Since $t \neq 0$, we know that it is invertible, and coefficients in $f$, $g$ are in $k$. Then we can reduce this to a monic polynomial:
            \begin{equation*}
                c(g - t^{-1}f)^{-1}(g - t^{-1}f) = 0
            \end{equation*}
        So this polynomial is of degree $\max(\deg f, \deg  g)$ that lies in $k(t)$ which kills $x$. So we take the quotient from $k(t)$ to get an extension to $k(x)$ with degree $\max(\deg f, \deg g)$. 

        Any automorphism fixes $k$, so the action of $\sigma$ on $x$ determines the action of $\sigma$ on all of $k(x)$ and therefore the automorphisms of $K/k$. Don't know why 
            \begin{equation*}
                 \sigma  ( X) = \dfrac{ax + b}{cx + d}
            \end{equation*}
        But if that is the case, then $\sigma ( x) \notin k$. Then since either $c \neq 0$ or $d \neq 0$, we have that 
            \begin{itemize}
                \item $c \neq 0$, then $\frac{a}{c}(cx + d) = ax + \frac{da}{c}$. So
                    \begin{equation*}
                        \dfrac{ax + b}{ax + \dfrac{da}{c}} \notin k
                    \end{equation*}
                and therefore, $\frac{da}{c} \neq b$ or $ad - bc \neq 0$.

                \item $d \neq 0$, then $\frac{b}{d}(cx + d) = \frac{bc}{d}x + b$. Then
                    \begin{equation*}
                        \dfrac{ax + b}{\dfrac{bc}{d}x + b} \notin k
                    \end{equation*}
                and therefore, $\frac{bc}{d} \neq a$ and $ad - bc \neq 0$.
            \end{itemize}
        Now for well-defined, suppose that
            \begin{equation*}
                \dfrac{ax + b}{cx + d} = \dfrac{a^{\prime}x + b^{\prime}}{c^{\prime}x + d^{\prime}}
            \end{equation*}
        Cross multiply and we get 
            \begin{equation*}
                ac^{\prime}x^{2} + (\ldots) x + bd^{\prime} = a^{\prime}cx + (\ldots) x + b^{\prime}d
            \end{equation*}
        Then
            \begin{align*}
                ac^{\prime}           &= a^{\prime}c           & bd^{\prime}           &= b^{\prime}d           \\
                \dfrac{a}{a^{\prime}} &= \dfrac{c}{c^{\prime}} & \dfrac{b}{b^{\prime}} &= \dfrac{d}{d^{\prime}}   
            \end{align*}
        If the denominator is $0$, we see that the numerator has to be $0$ also and continue with that information. Now we have these ratios, and are now considering when
            \begin{equation*}
                \dfrac{ax + b}{cx + d} = \dfrac{kax + k^{\prime}b}{kcx + k^{\prime}d}
            \end{equation*}
        Cross multiplying and considering the degree $1$ terms:
            \begin{equation*}
                kbcx + k^{\prime}adx = kadx + k^{\prime}bcx
            \end{equation*}
        Then
            \begin{equation*}
                k(bc - ad) = k^{\prime}(bc - ad)
            \end{equation*}
        and since $bc - ad \neq 0$, we have
            \begin{equation*}
                \dfrac{k}{k^{\prime}} = 1
            \end{equation*}
        which shows that $a, b, c, d$ are well defined up to scaling. Then the Galois group is in bijection with matrices
            \begin{equation*}
                \begin{bmatrix}
                    a & b \\
                    c & d   
                \end{bmatrix}
            \end{equation*}
        invertible because $ad - bc \neq 0$ in $PGL(2, k)$ because it is mod scaling.
    \end{proof} 











\end{document} 
