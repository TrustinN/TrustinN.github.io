%! TeX root = /Users/trustinnguyen/Downloads/Berkeley/Math/Math250a/Homework/Math250aHw3/Math250aHw3.tex

\documentclass{article}
\usepackage{/Users/trustinnguyen/.mystyle/math/packages/mypackages}
\usepackage{/Users/trustinnguyen/.mystyle/math/commands/mycommands}
\usepackage{/Users/trustinnguyen/.mystyle/math/environments/article}

\title{Math250aHw3}
\author{Trustin Nguyen}

\begin{document}

    \maketitle

\reversemarginpar

\textbf{Exercise 1}: Prove that the ring of integers, $\mathbb{Z}$, is a principal ideal domain. (Hint: Use the Euclidean algorithm, which uses ``division with remainder'' in $\mathbb{Z}$ to compute the single generator of an ideal in the ring of integers defined by two integers $(a, b) \subseteq \mathbb{Z}$.)
    \begin{proof}
        Consider the ideal generated by $d_{1}, s = (d_{1}, s)$ where $d_{1}, s \neq 0, 1$ and such that $d_{1} \neq s$. Also remove the case where $d_{1} \divides s$ or $s \divides d_{1}$. Then we have that wlog, $d_{1} < s$. So by the division algorithm, we have
            \begin{equation*}
                s = d_{1}q_{1} + d_{2}
            \end{equation*}
        for some $q_{1} \in \mathbb{Z}$ and $d_{2} < d_{1}$. Since $d_{2} < d_{1}$, we apply the same process:
            \begin{equation*}
                d_{1} = d_{2}q_{2} + d_{3}
            \end{equation*}
        We continue this process until it stops, which we know it will because the remainder becomes smaller every time. The process stops when $d_{n} \divides d_{n - 1}$:
            \begin{equation*}
                d_{n - 1} = d_{n}q_{n}
            \end{equation*}
        But observe now that
            \begin{equation*}
                d_{n - 2} = d_{n - 1}q_{n - 1} + d_{n}
            \end{equation*}
        or if we substitute $d_{n}q_{n} = d_{n - 1}$:
            \begin{equation*}
                d_{n - 2} = d_{n}q_{n}q_{n - 1} + d_{n}
            \end{equation*}
        so $d_{n} \divides d_{n - 2}$. By backwards strong induction, we continue this process and conclude that $d_{n} \divides s, d_{1}$. We also conclude that $d_{n} \in (d_{1}, s)$ because the Euclidean algorithm was carried out within our ideal $(d_{1}, s)$. Finally, we can conclude that since $d_{n} \divides s, d_{1}$, then $(s, d_{1}) \subseteq (d_{n})$. Therefore, we have a double inclusion and the ideals are equal, showing that all ideals are generated by a single element.
    \end{proof}

\textbf{Exercise 2}: Let $\mathbb{Q}$ be the field of rational numbers. Use the fact that $\mathbb{Q}[x]$ is a principal ideal domain to show that $\mathbb{Q}[x]/(x^{2} + 1)$ is a field. (You don't need to find the formula for division to do this.) Show that
    \begin{equation*}
        \mathbb{Q}[x]/(x^{2} + 1) \cong \mathbb{Q}[i] := \{a + bi : a, b \in \mathbb{Q}, i^{2} = -1\}
    \end{equation*}
by finding the ring homomorphism that carries a vector space basis of the first ring to a vector space basis of the second. Inside $\mathbb{Q}[i]$ is the ring of \textit{Gaussian integers}, defined as a subring $\mathbb{Z}[i] = \{a + bi : a, b \in \mathbb{Z}\}$.
    \begin{proof}
        Consider the evaluation map of $\mathbb{Q}[x]$ at $i$ which goes to $\mathbb{Q}[i]$
            \begin{align*}
                \varphi &: \mathbb{Q}[x] \rightarrow \mathbb{Q}[i] \\
                \varphi(p(x)) &:= p(i)
            \end{align*}
        This is a homomorphism. Notice that $(x^{2} + 1) \subseteq \ker{\varphi}$. Now suppose that our ideal $\ker{\varphi}$ was also generated by another element $f$:
            \begin{equation*}
                (x^{2} + 1, f) \subseteq \ker{\varphi}
            \end{equation*}
        Since $\mathbb{Q}[x]$ is a PID, we can say that it is generated by a single element: $(g)$, so for $h_{1}, h_{2} \in \mathbb{Q}[x]$,
            \begin{align*}
                gh_{1} &= x^{2} + 1 \\
                gh_{2} &= f           
            \end{align*}
        But we know that $x^{2} + 1$ is irreducible in $\mathbb{Q}[x]$. Therefore, $g$ is either a unit or $x^{2} + 1$. It cannot be a unit. So $g$ is $x^{2} + 1$ meaning that $(x^{2} + 1) \divides f$. So $(x^{2} + 1, f) = (x^{2} + 1)$. Therefore, $\ker{\varphi} = (x^{2} + 1)$. Notice that $\varphi$ is also surjective. Therefore, $\mathbb{Q}[x]/(x^{2} + 1) \cong \mathbb{Q}[i]$. Now to show that $\mathbb{Q}[i]$ has inverses for every element except $0$, suppose that $a + bi \in \mathbb{Q}[i]$ where $a, b$ are not both zero. Then
            \begin{equation*}
                \dfrac{1}{a + bi} = \dfrac{a - bi}{a^{2} + b^{2}} = \dfrac{a}{a^{2} + b^{2}} - \dfrac{b}{a^{2} + b^{2}} i \in \mathbb{Q}[i]
            \end{equation*}
        So inverses map to inverses and therefore, $\mathbb{Q}[x]/(x^{2} + 1)$ is a field.
    \end{proof}

\textbf{Exercise 3}: Define the \textit{Norm} of any complex number $a + bi$ to be $N(a + bi) := a^{2} + b^{2}$. In the complex plane, show that every complex number differs from some Gaussian integer by a complex number whose norm is $\leq 1 /2$.
    \begin{proof}
        Suppose that $a + bi \in \mathbb{C}$. Consider $x + yi \in \mathbb{Z}[i]$. We want to find an $x, y$ such that
            \begin{equation*}
                N(a + bi - (x + yi)) \leq \dfrac{1}{2}
            \end{equation*}
        Observe that the norm is 
            \begin{equation*}
                (a - x)^{2} + (b - y)^{2}
            \end{equation*}
        Consider the decimal part of $a$ given by $0 \leq a - \lfloor a \rfloor \leq 1$. If $a - \lfloor a \rfloor > \frac{1}{2}$, let $x = \lfloor a \rfloor + 1$, otherwise, $x = \lfloor a \rfloor$. Notice that now, $\lvert a - x \rvert \leq \frac{1}{2}$. Therefore, $(a - x)^{2} \leq \frac{1}{4}$. Repeat the same thing for $b - y$ and we get
            \begin{equation*}
                N(a + bi - (x + yi)) \leq \dfrac{1}{4} + \dfrac{1}{4} = \dfrac{1}{2}
            \end{equation*}
        which concludes the proof.
    \end{proof}

\textbf{Exercise 4}: Show that $N((a + bi)(c + di)) = N(a + bi)N(c + di)$.
    \begin{proof}
        Just expand:
            \begin{align*}
                N((a + bi)(c + di)) &= N(ac - bd + (ad + bc)i)                                   \\
                                    &= (ac - bd)^{2} + (ad + bc)^{2}                             \\
                                    &= (ac)^{2} - 2abcd + (bd)^{2} + (ad)^{2} + 2abcd + (bc)^{2} \\
                                    &= (ac)^{2} + (ad)^{2} + (bd)^{2} + (bc)^{2}                 \\
                                    &= (a^{2} + b^{2})(c^{2} + d^{2})                            \\
                                    &= N(a + bi)N(c + di)                                        
            \end{align*}
        so we are done.
    \end{proof}

\textbf{Exercise 5}: Show that $\mathbb{Z}[i]$ is a \textit{Euclidean ring} with norm $N$, in the sense that given Gaussian integers $a + bi$ and $c + di$, there is a Gaussian integer $e + fi$ such that
    \begin{equation*}
        a + bi = (c + di)(e + fi) + \varepsilon
    \end{equation*}
where $\varepsilon$ is a Gaussian integer and $N(\varepsilon) < N(c + di)$. (Hint: approximate the result of dividing in the field $\mathbb{Q}[i]$. You don't need to find a formula for division to do this.)
    \begin{proof}
        Consider division over the field of fractions $\mathbb{Q}[i]$. We want to find a $z \in \mathbb{Z}[i]$ such that
            \begin{equation*}
                \dfrac{a + bi}{c + di} - z = r
            \end{equation*}
        where our remainder $r$ has a norm less than $c + di$. Notice that $\frac{a + bi}{c + di} \in \mathbb{Q}[i]$, so by the previous problem, we have that there is a $z$ such that
            \begin{equation*}
                N\left(\dfrac{a + bi}{c + di} - z\right) = N(r) \leq \dfrac{1}{2}
            \end{equation*}
        Now we solve for $a + bi$:
            \begin{align*}
                \dfrac{a + bi}{c + di} - z &= r                     \\
                \dfrac{a + bi}{c + di}     &= z + r                 \\
                a + bi                     &= (c + di)z + r(c + di)   
            \end{align*}
        We see that $\varepsilon = r(c + di)$, so therefore, $N(\varepsilon) = N(r)N(c + di)$, but since $N(r) < 1$, we have $N(\varepsilon) < N(c + di)$.
    \end{proof}

\textbf{Exercise 6}: Imitate the Euclidean algorithm to prove that $\mathbb{Z}[i]$ is a principal ideal domain.
    \begin{proof}
        Consider the ideal generated by $d_{1}, s = (d_{1}, s)$ with $d_{1}, s \in \mathbb{Z}[i]$ and where $d_{1}, s \neq 0, 1$ such that $d_{1} \neq s$. Also remove the case where $d_{1} \divides s$ or $s \divides d_{1}$. Then we have that wlog, $N(d_{1}) < N(s)$. So by the division algorithm, we have
            \begin{equation*}
                s = d_{1}q_{1} + d_{2}
            \end{equation*}
        for some $q_{1} \in \mathbb{Z}$ and $N(d_{2}) < N(d_{1})$. Since $N(d_{2}) < N(d_{1})$, we apply the same process:
            \begin{equation*}
                d_{1} = d_{2}q_{2} + d_{3}
            \end{equation*}
        We continue this process until it stops, which we know it will because $N(d_{i}) < N(d_{i - 1})$. The process stops when $d_{n} \divides d_{n - 1}$:
            \begin{equation*}
                d_{n - 1} = d_{n}q_{n}
            \end{equation*}
        But observe now that
            \begin{equation*}
                d_{n - 2} = d_{n - 1}q_{n - 1} + d_{n}
            \end{equation*}
        or if we substitute $d_{n}q_{n} = d_{n - 1}$:
            \begin{equation*}
                d_{n - 2} = d_{n}q_{n}q_{n - 1} + d_{n}
            \end{equation*}
        so $d_{n} \divides d_{n - 2}$. By backwards strong induction, we continue this process and conclude that $d_{n} \divides s, d_{1}$. We also conclude that $d_{n} \in (d_{1}, s)$ because the Euclidean algorithm was carried out within our ideal $(d_{1}, s)$. Finally, we can conclude that since $d_{n} \divides s, d_{1}$, then $(s, d_{1}) \subseteq (d_{n})$. Therefore, we have a double inclusion and the ideals are equal, showing that all ideals are generated by a single element.
    \end{proof}





























\end{document}
