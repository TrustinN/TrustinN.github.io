%! TeX root = 	

\documentclass{article}
\usepackage{/Users/trustinnguyen/MyStyle/mystyle}

\title{Math113Hw2}
\author{Trustin Nguyen}


\begin{document}
\maketitle
\reversemarginpar

\begin{topic}
	\section*{Homework 2}
\end{topic}

\textbf{Exercise 1}: 
\begin{enumerate}
	\item Write the following permutations as products of disjoint cycles and hence calculate their orders and signs:
		\begin{enumerate}
			\item [(a)] $(1 \, 2)(1 \, 2 \, 3 \, 4)(1 \, 2)$:

				We see that $1 \mapsto 2 \mapsto 3$, $3 \mapsto 4$, $4 \mapsto 1 \mapsto 2$, $2 \mapsto 1\mapsto 2\mapsto 1$. So the cycle is $(1 \, 3 \, 4 \, 2)$.

			\item [(b)] $(1 \, 2 \, 3)(2 \, 3 \, 5)(3 \, 4 \, 5) (4 \, 5)$

				There is $1 \mapsto 2$, $2\mapsto 3 \mapsto 1$, $4 \mapsto 5\mapsto 3 \mapsto 5$, $5 \mapsto 4 \mapsto 5 \mapsto 2 \mapsto 3$. Therefore, we have the cycle: $(1 \, 2)(3 \, 4 \, 5)$. 
		\end{enumerate}
	\item What is the largest possible order of an element of $S_{5}$? What about $S_{9}$? Justify your answers.

		The largest possible order of $S_{5}$ is 6 since we have a cycle of lenth 2 and another of length 3. We want the closest pair of nonequal primes which add up to 5, which will yield the largest lcm. 

		As for $S_{9}$ we notice that the largest lcm cannot have 3 prime factors since $2, 3, 5$ the samallest ones add up to a number greater than $9$. Then by the previous rule we made, we chose 3, 5, which add up to 8. The largest order is 15.

	\item Show that any element of $S_{10}$ of order 14 is odd. 

		We observe that order 2 cycles and order 7 cycles (disjoint) must make up the element. since
		\begin{equation*}
			2x + 7y < 10
		\end{equation*}
		for positive $x, y$ is only satisfied when $x = y = 1$, the element has one $7$ order cycle and one $2$ order cycle. Noting that signs are a homomorphism, 
		\begin{equation*}
			\text{sgn}(c_{1}c_{2}) = \text{sgn}(c_{1})\text{sgn}(c_{2})
		\end{equation*}
		where $c_{1}$ can be broken down into a product of 2 - 1 transpositions and $c_{1}$ is a product of $7 - 1$ transpositions.
		\begin{equation*}
			\text{sgn}(c_{1}c_{2}) = (-1)^{1}(-1)^{6} = (-1)^{7} - 1
		\end{equation*}
		which says that it is odd.
\end{enumerate}

\textbf{Exercise 2}:
\begin{enumerate}
	\item Show that for any divisor $d$ of 24, there is a subgroup of $S_{4}$ of order $d$.

		\begin{proof}
			The divisors 1 and 24 are trivial, 12 is the order of $A_{4}$. For $2, 3, 4$ we take $\brac{(1 \, 2)}, \brac{(1 \, 2 \, 3)}, \brac{(1 \, 2 \, 3\, 4)}$. For $6, 8$, $D_{6}$ and $D_{8}$ are subgroups.
		\end{proof}

	\item Find two non-isomorphic subgroups of $S_{4}$ of order 4.

		\begin{proof}
			Let the groups be 
			\begin{equation*}
				G = \{e, (1 \, 2 \, 3 \, 4), (1 \, 3)(2 \, 4), (1 \, 4 \, 3 \, 2)\} \\
				H = \{e, (1 \, 2)(3 \, 4), (1 \, 3)(2 \, 4), (1 \, 4)(2 \, 3)\}
			\end{equation*}
			Notice that $G$ is cyclic and $H$ is isomorphic to $C_{2} \times C_{2}$. these are not isomorphic.
		\end{proof}

	\item Find a subgroup of $A_{4}$ of order 4. Is there a subgroup of order 6? Justify your answer.

		\begin{proof}
			The subgroup of $A$ of order 4 is the group $C_{2} \times C_{2}$ since all elements are written as a product of an even number of transpositions. Notice that $A_{4}$ contains all three cycles and the disjoint transpositions that make up the $C_{2} \times C_{2}$ groups. These are the minimal subgroups of $A_{4}$, the ones with the least order other than the trivial subgroup. But it we have a group of order $6$, it must be either dihedral or cyclic. It cannot be dihedral because we have no $C_{2}$ group. It is not cyclic because then it need to have 6 elements and our group only has 4.   
		\end{proof}

		Idea behind this proof: I first attempted to look at how the 6 group was generated which led to a lot of case work such as how the elements of order $3$ mulitplied with that of order $4$. This was messy and did not look at the group structure of the group. This proof is much better because it looks at what the $A_{4}$ group looks like: it contains 3-cycles and product of two transpositions. And this helps to see that there is no group of order 6 simply because these minimal subgroups cannot generate it and that if the group of order 6 was not generated by these groups, it would be too big.

\end{enumerate}

\textbf{Exercise 3}: Let $G$ be a group which contains elements of order $6$ and $10$. Show that $G$ contains at least $30$ elements.

\begin{proof}
	By Lagrange's Theorem, we have 
	\begin{equation*}
		\abs{K}\abs{G : K} = \abs{G}
	\end{equation*}
	So for a cyclic subgroup of $G$, $K$, the order of that element divides the order of $G$, therefore, $6 \divides \abs{G}$, $10 \divides \abs{G}$, $\lcm{(6,10)} \divides \abs{G}$. And we have the least common multiple as $30$, since $G$ is non-empty, there is an element of order 30 which is the group operation of the one of order 6 and the other of order 10.
\end{proof} 

\textbf{Exercise 4}: We say that a finite group $G$ is generated by a set $T$ if any element of $G$ can be written as a finite product (with repetitions) of powers of elements of $T$. Show that $S_{n}$ is generated by each of the following sets:

\begin{enumerate}
	\item The set $\{(j \, k) : 1 \leq j \leq k \leq n\}$;
		\begin{proof}
			We know that every permutation can be written as a product of cycles, each of which can be  written as a product of transpositions. This is the set of all transpositions in $S_{n}$, so we can generate every permutation with a product of the elements in the set.
		\end{proof}

	\item The set $\{(j \, j + 1) : 1 \leq j \leq n\}$;
		\begin{proof}
			This seems like a set that generates the previous set. Suppose $(j \, k)$ is a transposition where $j = k - a$ for some $a \geq 0$. For $a = 0$,
			\begin{equation*}
				(j \, j + 1)(j \, j + 1)
			\end{equation*}
			Gives us the identity map for $j$. For $a = 1$, this follows by definition. For $a > 1$, 
			\begin{gather*}
				(j \, j + 1)(j + 1 \, j + 2) \cdots (j + a \, j + a - 1) \cdots (j + 1 \, j + 2)(j \, j + 1) \\
				(j \, j + a)(j + 1) \\
				(j \, k)
			\end{gather*}
			So this set generates the previous set, and therefore generates $G$. 
		\end{proof}

	\item The set $\{(1 \, k) : 1 < k \leq n\}$;
		\begin{proof}
			This set generates the set in $1$, since for $1 < k < n, k < j \leq n$, 
			\begin{equation*}
				(1 \, k)(1 \, j)(1 \, k) = (1)(k \, j) = (k \, j)
			\end{equation*}
			which is what we wanted.
		\end{proof}

	\item The set $\{(1 \, 2), (1 \, 2 \, \ldots \, n)\}$;
		\begin{proof}
			We can create a construction of the identity:
			\begin{align*}
				\underbrace{(1 \, 2 \, \ldots \, n)\ldots(1 \, 2 \, \ldots \, n)}_{n \text{ times}} &= e \\
				\underbrace{(1 \, 2 \, \ldots \, n)\cdots(1 \, 2\, \ldots \, n)}_{n - 1 \text{ times}} &= (n \, \ldots 2 \, 1)
			\end{align*}
			So the invers of both elements exist. Now a special property:
			\begin{equation*}
				(1 \, 2 \, \ldots \, n)(j \, j + 1)(n \, \ldots \, 2 \, 1) = (j + 1 \, j + 2)
			\end{equation*}
			By induction using $(1 \, 2)$, we have $\{(j \, j + 1) : 1 \leq j \leq n\}$ is a subset of the group generated by $\{(1 \, 2), (1 \, 2 \, \ldots \, n)\}$ which completes the proof by number 2.
		\end{proof}
		Idea behind this proof: Trying to match up this set to the previous subsets that we know. The fact that there were inverses is cool because the idea is that by using the base $(1 \, 2)$, we can generate the next element by taking $c_{1}(1 \, 2)c_{1}^{-1}$ which is cool and this happens to match up with one of our previous generating sets.
\end{enumerate}

\textbf{Exercise 5}: Let $G$ be a finite group in which every element other than the identity has order 2. Show that $\abs{G}$ is a power of $2$.
\begin{proof}
	Since $G$ is a finite group with elements of order 2 only, if $G = \{e\}$, $\abs{G} = 2^{0}$. If $G$ contains exactly one transposition, we have $\abs{G} = 2^{1}$. If $G$ has more than 1 transposition, we prove that they are disjoint. Suppose they are not. Then for $b \neq c$,
	\begin{equation*}
		(a \, b)(a \, c) = (a \, c \, b)
	\end{equation*}
	which is in $G$ but has order $3$. Contradiction. Therefore, we take the set of all transposiions and for each transposition, a unique element in $G$ has property:
	\begin{equation*}
		g = \tau_{1}^{k_{1}}\tau_{2}^{k_{2}}\cdots\tau_{n}^{k_{n}}
	\end{equation*}
	Since all $k_{i} = 0$ or $1$, we have $\abs{G} = 2^{n}$.
\end{proof}
\end{document}
