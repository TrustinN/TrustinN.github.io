%! TeX root = /Users/trustinnguyen/Downloads/Berkeley/Math/Math104/Homework/Math104Hw2/Math104Hw2.tex

\documentclass{article}
\usepackage{/Users/trustinnguyen/.mystyle/math/packages/mypackages}
\usepackage{/Users/trustinnguyen/.mystyle/math/commands/mycommands}
\usepackage{/Users/trustinnguyen/.mystyle/math/environments/article}

\title{Math104Hw2}
\author{Trustin Nguyen}

\begin{document}

    \maketitle

\reversemarginpar

\textbf{Exercise 1}: State what it means mathematically that a sequence $(s_{n})$ converges to $s \in \mathbb{R}$.
    \begin{answer}
        For a sequence $(s_{n})$ to converge to $s$, we must have that $\forall \varepsilon > 0$, there exists an $N$ such that $\forall n > N$, we have that:
            \begin{equation*}
                \lvert s_{n} - s \rvert < \varepsilon
            \end{equation*}
    \end{answer}

\textbf{Exercise 2}: Assume that $\text{lim}(s_{n}) = 0$. Prove that $\text{lim}(\sqrt[3]{s_{n}}) = 0$.
    \begin{proof}
        Since $\text{lim}(s_{n}) = 0$, we have that for all $\varepsilon$, there is an $N$ such that $\forall n > N$, 
            \begin{equation*}
                \lvert s_{n} \rvert < \varepsilon
            \end{equation*}
        or
            \begin{equation*}
                -\varepsilon < s_{n} < \varepsilon
            \end{equation*}
        If we take the cube root of both sides, we have
            \begin{equation*}
                \sqrt[3]{-\varepsilon} < \sqrt[3]{s_{n}} <  \sqrt[3]{\varepsilon}
            \end{equation*}
        But now we are done as we can replace $\sqrt[3]{\varepsilon} = \delta$ and say that there is an $N$ such that for all $n > N$,
            \begin{equation*}
                \lvert \sqrt[3]{s_{n}} \rvert <  \delta
            \end{equation*}
        We would just choose $N$ based on the epsilon for our previous sequence, where $\varepsilon = \delta^{3}$.
    \end{proof}

\textbf{Exercise 3}: Use the definition of convergence to prove $\text{lim}(\frac{3n}{n + 1}) = 3$; no theorem in Ross 9 is allowed.
    \begin{proof}
        We need to show that $\forall \varepsilon >  0$, we have an $N$ such that $\forall n > N$,
            \begin{equation*}
                \left\lvert \dfrac{3n}{n + 1} - 3 \right\rvert < \varepsilon
            \end{equation*}
        So we can collect the terms in the absolute value:
            \begin{equation*}
                \left\lvert \dfrac{-3}{n + 1} \right\rvert < \varepsilon
            \end{equation*}
        Therefore, we require that:
            \begin{equation*}
                \dfrac{3}{n + 1} < \varepsilon \text{ or } n > \dfrac{3}{\varepsilon} - 1
            \end{equation*}
        Now let $N = \frac{3}{\varepsilon} - 1$. Then we plug in $n > \frac{3}{\varepsilon} - 1$ into
            \begin{equation*}
                \left\lvert \dfrac{3n}{n + 1} - 3 \right\rvert = \left\lvert \dfrac{-3}{n + 1} \right\rvert
            \end{equation*}
        We get that 
            \begin{align*}
                n &> \frac{3}{\varepsilon} - 1 \\
                n + 1 &> \frac{3}{\varepsilon} \\
                \varepsilon &> \frac{3}{n + 1}
            \end{align*}
        But $\lvert \frac{-3}{n + 1} \rvert = \frac{3}{n + 1}$. So we can conclude that for any $\varepsilon$, we have found an $N$ such that $\forall n > N$,
            \begin{equation*}
                \left\lvert \dfrac{3n}{n + 1} - 3 \right\rvert < \varepsilon
            \end{equation*}
        So we are done.
    \end{proof}

\textbf{Exercise 4}: Use the definition of convergence to prove $\text{lim}(\frac{n - 1}{n^{2} + 1}) = 0$; no theorem in Ross 9 is allowed.
    \begin{proof}
        We want that $\forall \varepsilon > 0$, there exists an $N$ such that $\forall n > N$, we have
            \begin{equation*}
                \left\lvert \dfrac{n - 1}{n^{2} + 1} \right\rvert < \varepsilon
            \end{equation*}
        We can choose an intermediate function:
            \begin{equation*}
                \left\lvert \dfrac{n - 1}{n^{2} + 1} \right\rvert < \left\lvert \dfrac{n - 1}{n^{2} - 1} \right\rvert < \varepsilon
            \end{equation*}
        So we require that $N \geq 1$. But now we see that 
            \begin{equation*}
                \dfrac{n - 1}{n^{2} - 1} = \dfrac{1}{n + 1} < \varepsilon
            \end{equation*}
        and so we take
            \begin{equation*}
                \dfrac{1}{\varepsilon} - 1 < n
            \end{equation*}
        Using the original constraints, let $N = \max(1, \frac{1}{\varepsilon} - 1)$. Now if $n > \max(1, \frac{1}{\varepsilon} - 1)$, we get
            \begin{align*}
                n - 1     &>  0                                                       \\
                \dfrac{1}{n + 1} &< \varepsilon
            \end{align*}
        Therefore,
            \begin{equation*}
                \left\lvert \dfrac{n - 1}{n^{2} - 1} \right\rvert = \left\lvert \dfrac{1}{n + 1} \right\rvert = \dfrac{1}{n + 1} < \varepsilon
            \end{equation*}
        For $n > 1$, we have 
            \begin{equation*}
                \left\lvert \dfrac{n - 1}{n^{2} + 1} \right\rvert =  \dfrac{n - 1}{n^{2} + 1} < \dfrac{n - 1}{n^{2} - 1} = \left\lvert \dfrac{n - 1}{n^{2} - 1} \right\rvert
            \end{equation*}
        So we have that $\forall \varepsilon$, there is an $N$ such that $\forall n > N$,
            \begin{equation*}
                \left\lvert \dfrac{n - 1}{n^{2} + 1} \right\rvert < \left\lvert \dfrac{n - 1}{n^{2} - 1} \right\rvert < \varepsilon
            \end{equation*}
        which finishes the proof.
    \end{proof}

\textbf{Exercise 5}: Use the definition of convergence to prove that the sequence $(s_{n})$ diverges where $s_{n} = \sqrt{n}$.
    \begin{proof}
        We wish to show that the sequence does not converge. So we want that $\exists \varepsilon > 0$ we have that for any $N$, there is an $n > N$ such that 
            \begin{equation*}
                \lvert \sqrt{n} - L \rvert \geq \varepsilon
            \end{equation*}
        where $L$ is an arbitrary number in $\mathbb{R}$. Suppose for contradiction, there was an $N$ such that for all $n > N$, 
            \begin{equation*}
                \lvert \sqrt{n} - L \rvert < \varepsilon
            \end{equation*}
        Take $\varepsilon = 1$. Take an arbitrary $n_{0} > N$ and square it. We have $N < n_{0} \leq n_{0}^{2}$. So plugging in $n = n_{0}^{2}$, we get:
            \begin{equation*}
                -1 + L < n_{0} < 1 + L
            \end{equation*}
        But now consider $N < n_{0}^{2} < (n_{0} + 5)^{2}$. So
            \begin{align*}
                -1 + L < n_{0} + 5 < 1 + L \\
                -1 - L < -n_{0} < 1 - L
            \end{align*}
        The sum of the inequalities gives us:
            \begin{equation*}
                -2 < 5 < 2
            \end{equation*}
        which is a contradiction. So for any $N$, there is an $n > N$ such that
            \begin{equation*}
                \lvert \sqrt{n} - L \rvert \geq \varepsilon
            \end{equation*}
        This shows non-convergence.
    \end{proof}

\textbf{Exercise 6}: Assume that $\text{lim}(s_{n}) = s$ and $s_{n} \geq 0$. Prove that $s \geq 0$.
    \begin{proof}
        Since the limit exists, we have that $\forall \varepsilon > 0$, there is an $N$ such that $\forall n > N$, we have
            \begin{equation*}
                \lvert s_{n} - s \rvert < \varepsilon
            \end{equation*}
        This also means
            \begin{equation*}
                -\varepsilon + s < s_{n} < \varepsilon + s
            \end{equation*}
        But since $s_{n} \geq 0$, we have
            \begin{equation*}
                0 \leq s_{n} < \varepsilon + s \implies 0 \leq \varepsilon + s
            \end{equation*}
        Suppose for contradiction that $s < 0$. We want to find an $\varepsilon$ small enough such that its sum with $s$ lies below $0$. Take $\varepsilon = \lvert \frac{s}{2} \rvert$. So
            \begin{equation*}
                \left\lvert \dfrac{s}{2} \right\rvert + s = s - \dfrac{s}{2} = \dfrac{s}{2} < 0
            \end{equation*}
        But that is impossible because we started with: $\forall \varepsilon > 0$, $0 \leq \varepsilon + s$. Therefore, $s \geq 0$.
    \end{proof}











\end{document}
