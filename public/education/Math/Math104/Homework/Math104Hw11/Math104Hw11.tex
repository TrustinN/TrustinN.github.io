%! TeX root = /Users/trustinnguyen/Downloads/Berkeley/Math/Math104/Homework/Math104Hw11/Math104Hw11.tex

\documentclass{article}
\usepackage{/Users/trustinnguyen/.mystyle/math/packages/mypackages}
\usepackage{/Users/trustinnguyen/.mystyle/math/commands/mycommands}
\usepackage{/Users/trustinnguyen/.mystyle/math/environments/article}
\graphicspath{{./figures/}}

\title{Math104Hw11}
\author{Trustin Nguyen}

\begin{document}

    \maketitle

\reversemarginpar

\textbf{Exercise 1}: Show that $f_{n} = \frac{x^{n}}{n}$ converges uniformly on $[0, 1]$.
    \begin{proof}
        We will show that $f_{n} \rightarrow 0$ uniformly by $\limsup\limits_{n \to \infty}  \{\lvert f_{n}(x) \rvert : x \in [0, 1]\} = 0$. Notice that for each $f_{n}$, the derivative $f^{\prime}_{n} = x^{n - 1}$, which is positive on $[0, 1]$, so $f_{n}$ achieves its max at $x = 1$. It achieves its minimum at $x = 0$. We have that
            \begin{equation*}
                0 = \lvert f_{n}(0) \rvert < \lvert f_{n}(1) \rvert = \dfrac{1}{n}
            \end{equation*}
        And indeed, $\lim\limits_{n \to \infty} \frac{1}{n} = 0$. So it converges uniformly on $[0, 1]$.
    \end{proof}

\textbf{Exercise 2}: Assume that $\sum \lvert a_{k} \rvert < \infty$, prove that $\sum a_{k}x^{k}$ converges uniformly on $[-1, 1]$.
    \begin{proof}
        By the Weierstrass M-Test, we know that $\sum a_{k}x^{k}$ converges uniformly on $S$ if $\lvert a_{k}x^{k} \rvert \leq \lvert a_{k} \rvert$ for $x \in S$. So we have:
             \begin{align*}
                 \lvert a_{k} \rvert \lvert x^{k} \rvert &\leq \lvert a_{k} \rvert \\
                                     \lvert x^{k} \rvert &\leq 1 \\
                                           -1 \leq x^{k} &\leq 1
             \end{align*}
        which is true exactly when $x \in [-1, 1]$, so we are done.
    \end{proof}

\textbf{Exercise 3}: Show that $\sum_{n = 1}^{\infty} nx^{n} = \frac{x}{( 1- x)^{2}}$ for $\lvert x \rvert < 1$. 
    \begin{proof}
        We have that $\sum_{n \geq 0} x^{n} = \frac{1}{1 - x}$. Then we use the ratio test:
            \begin{equation*}
                \beta = \lim\limits_{n \to \infty}\dfrac{1}{1} = 1
            \end{equation*}
        So $R = \frac{1}{\beta} = 1$. The series does not converge on $-1, 1$. So $\sum_{n \geq 0} x^{n}$ is differentiable on $ (-1, 1)$. Taking the derivative:
            \begin{align*}
                \dv{x}\sum_{n \geq 0}x^{n} &= \dv{x}\dfrac{1}{1 - x}  \\
                \sum_{n \geq 1}nx^{n - 1}  &= \dfrac{1}{(1 - x)^{2}}   
            \end{align*}
        Then multiply by $x$ on both sides:
            \begin{equation*}
                \sum_{n \geq 1}nx^{n} = \dfrac{x}{(1 - x)^{2}}
            \end{equation*}
    \end{proof}

\textbf{Exercise 4}: Evaluate $\sum_{n = 1}^{\infty}\frac{n}{2^{n}}$.
    \begin{proof}
        Recall that $\sum_{n \geq 0}y^{n} = \frac{1}{1 - y}$. Substituting $y = \frac{1}{2}x$, we get:
            \begin{equation*}
                \sum_{n \geq 0}\dfrac{1}{2^{n}}x^{n} = \dfrac{1}{1 - \dfrac{1}{2}x} = \dfrac{1}{\dfrac{2 - x}{2}} = \dfrac{2}{2 - x}
            \end{equation*}
        To make sure we can take the derivative, find the radius of convergence. Use the ratio test:
            \begin{equation*}
                \beta = \lim\limits_{n \to \infty} \left\lvert \dfrac{\dfrac{1}{2^{n + 1}}}{\dfrac{1}{2^{n}}} \right\rvert = \lim\limits_{n \to \infty} \left\lvert \dfrac{2^{n}}{2^{n + 1}} \right\rvert = \dfrac{1}{2}
            \end{equation*}
        Then $R = \frac{1}{\beta} = 2$. We can take the derivative in the interval $[-1, 1]$, so it is fine.

        Taking the derivative of both sides we get:
            \begin{equation*}
                \sum_{n \geq 1}\dfrac{n}{2^{n}}x^{n - 1} = \dfrac{2}{(2 - x)^{2}}
            \end{equation*}
        Substituting $x = 1$, we find:
            \begin{equation*}
                \sum_{n \geq 1}\dfrac{n}{2^{n}} = \dfrac{2}{(2 - 1)^{2}} = 2
            \end{equation*}

    \end{proof}

\textbf{Exercise 5}: Use $Q3$ to find the explicit formula for $\sum_{n = 1}^{\infty} n^{2}x^{n}$ when $ \lvert x \rvert < 1$.
    \begin{proof}
        We have $\sum_{n \geq 1} nx^{n} = \frac{x}{(1 - x)^{2}}$ for $\lvert x \rvert < 1$. We can take the derivative again. Notice that radius of convergence is preserved on derivatives. So:
            \begin{align*}
                \dv{x}\sum_{n\geq 1}nx^{n}    &= \dv{x}\dfrac{x}{(1 - x)^{2}}                      \\
                \sum_{n \geq 2}n^{2}x^{n - 1} &= \dfrac{(1 - x)^{2} + 2(1 - x)x}{(1 - x)^{4}}      \\
                \sum_{n \geq 2}n^{2}x^{n - 1} &= \dfrac{1 - 2x + x^{2} + 2x - 2x^{2}}{(1 - x)^{4}} \\
                \sum_{n \geq 2}n^{2}x^{n - 1} &= \dfrac{1 - x^{2}}{(1 - x)^{4}}                    \\
                \sum_{n \geq 2}n^{2}x^{n}     &= \dfrac{x(1 - x^{2})}{(1 - x)^{4}}                   
            \end{align*}
        So that is the formula.
    \end{proof}

\textbf{Exercise 6}: Let $f(x) = \lvert x \rvert$ on $\mathbb{R}$, prove that there is no $(a_{n})$ such that $\sum_{n = 0}^{\infty} a_{n}x^{n} = f(x)$ for any $x \in \mathbb{R}$. 
    \begin{proof}
        Suppose for contradiction $f(x) = \sum_{n \geq 0}a_{n}x^{n}$ for some sequence $(a_{n})$. Then the radius of convergence contains $0$, and we know that $\sum_{n = 0}^{\infty} a_{n}x^{n}$ is differentiable. Then $f^{\prime}(x) = \sum_{n = 1}^{\infty} na_{n}x^{n - 1}$ for $x = 0$. Now we show that $\lvert x \rvert$ is not differentiable at $0$. Consider 
            \begin{equation*}
                \lim\limits_{x \to 0+}\dfrac{f(x) - f(0)}{x - 0} = \lim\limits_{x \to 0^{+}}\dfrac{\lvert x \rvert}{ x} = 1
            \end{equation*}
        and
            \begin{equation*}
                \lim\limits_{x \to 0^{-}}\dfrac{f(x) - f(0)}{x - 0} = \lim\limits_{x \to 0^{-}}\dfrac{\lvert x \rvert}{ x} = -1
            \end{equation*}
        Since the limits are not equal, the limit does not exist for 
            \begin{equation*}
                \lim\limits_{x \to 0}\dfrac{f(x) - f(0)}{x - 0} = f^{\prime}(0)
            \end{equation*}
        which is a contradiction.
    \end{proof}






\end{document}
