%! TeX root = /Users/trustinnguyen/Downloads/Berkeley/Math/Math104/Homework/Math104Hw14/Math104Hw14.tex

\documentclass{article}
\usepackage{/Users/trustinnguyen/.mystyle/math/packages/mypackages}
\usepackage{/Users/trustinnguyen/.mystyle/math/commands/mycommands}
\usepackage{/Users/trustinnguyen/.mystyle/math/environments/article}
\graphicspath{{./figures/}}

\title{Math104Hw14}
\author{Trustin Nguyen}

\begin{document}

    \maketitle

\reversemarginpar

\textbf{Exercise 1}: Let $f(x) = x$ when $x$ is rational an $f(x) = 0$ when $x$ is irrational. Find $L(f)$ and $U(f)$ on $[0, 1]$. Show that $f$ is not integrable on $[0, 1]$.
    \begin{proof}
        The definition of each is 
            \begin{align*}
                L(f) &= \sup\{L(f, P) : \text{$P$ is a partition}\} \\
                U(f) &= \inf\{U(f, P) : \text{$P$ is a partition}\}   
            \end{align*}
        Notice that no matter what partition we pick, each interval $(t_{k - 1}, t_{k})$ will contain a rational because $\mathbb{Q}$ is dense in $\mathbb{R}$. So we have that since $1$ is rational, $U(f) > 0$ because we have that $1(t_{k} - t_{k - 1})$ is a summand and all summands are positive, as $f(x) \geq 0$. Also, we have that the infimum of $f$ within any partition is always $0$ because the irrationals are dense. So $U(f) \neq L(f)$ and $f$ is not integrable on $[0, 1]$.
    \end{proof}

\textbf{Exercise 2}: If $f$ is integrable on $[a, b]$, then $f$ is integrable on $[c, d] \subseteq [a, b]$.
    \begin{proof}
        Let $P$ be a partition of $[a, b]$. We want to show that for any $\varepsilon$, there $\exists P^{\prime}$ partition of $[c, d]$ such that 
            \begin{equation*}
                U(f, P^{\prime}) - L(f, P^{\prime}) < \varepsilon
            \end{equation*}
        Well, this is true for $P$:
            \begin{equation*}
                U(f, P) - L(f, P) < \varepsilon
            \end{equation*}
        Then let $P^{\prime} = P \cup \{c, d\}$ to which we see that 
            \begin{equation*}
                U(f, P^{\prime}) < U(f, P)
            \end{equation*}
        and 
            \begin{equation*}
                L(f, P^{\prime}) > L(f, P)
            \end{equation*}
        So 
            \begin{equation*}
                U(f, P^{\prime}) - L(f, P^{\prime}) < U(f, P) - L(f, P) < \varepsilon
            \end{equation*}
        Now we split $P^{\prime}$ into three partitions, $P_{1}, P_{2}, P_{3}$, where $P_1$ partitions, $a$ to $c$, $P_{2}$, $c$ to $d$, $P_{3}$, $d$ to $b$. Then
            \begin{equation*}
                (U(f, P_{1}) - L(f, P_{1})) + (U(f, P_{2}) - L(f, P_{2})) + (U(f, P_{3}) - L(f, P_{3})) < \varepsilon
            \end{equation*}
        Since each summand is positive, we see that 
            \begin{equation*}
                U(f, P_{2}) - L(f, P_{2}) < \varepsilon
            \end{equation*}
        So we have found a partition for the $\varepsilon$.
    \end{proof}

\textbf{Exercise 3}: Suppose that $f, g$ are continuous on $[0, 1]$ and $\int_{0}^{1} f(x) \, \dd{x} = \int_{0}^{1} g(x) \, \dd{x} $, show that $\exists x \in (0, 1)$ such that $f(x) = g(x)$.
    \begin{proof}
        Consider $\int_{0}^{1} f(x) - g(x) \, \dd{x} $. If $f(x) - g(x) \geq 0$, we immediately get that $f(x) - g(x) = 0$. If $f(x) - g(x) \leq 0$, we get the same result. So we are done. We can also use the IVT for integrals. So there is an $x_{0}$ such that 
            \begin{equation*}
                f(x_{0}) - g(x_{0}) = \dfrac{1}{b - a}\int_{a}^{b} f(x) - g(x) \, \dd{x} 
            \end{equation*}
        so
            \begin{equation*}
                f(x_{0}) - g(x_{0}) = \int_{0}^{1} f(x) - g(x) \, \dd{x}  = 0
            \end{equation*}
        and we get 
            \begin{equation*}
                f(x_{0}) = g(x_{0})
            \end{equation*}
        which is a cleaner proof.
    \end{proof}

\textbf{Exercise 4}: Show $\lvert \int_{-2\pi}^{2\pi} x^{2}\sin^{8}{x}{e^{x}} \, \dd{x}  \rvert \leq \frac{16\pi^{3}}{3}$.
    \begin{proof}
        We have 
            \begin{equation*}
                \left\lvert \int_{-2\pi}^{2\pi} x^{2}\sin^{8}{x}e^{x} \, \dd{x}  \right\rvert \leq \int_{-2\pi}^{2\pi} \left\lvert x^{2}\sin^{8}{x}e^{x} \right\rvert \, \dd{x} 
            \end{equation*}
        Since $\lvert \sin^{8}{x} \rvert \leq 1$, $x^{2} \geq 0$ we have 
            \begin{equation*}
                \int_{-2\pi}^{2\pi} \lvert x^{2}\sin^{8}{x}e^{2} \rvert \, \dd{x}  \leq \int_{-2\pi}^{2\pi} x^{2} \, \dd{x}  = \left(\dfrac{x^{3}}{3}\right)\eval_{-2\pi}^{2\pi}
            \end{equation*}
        So
            \begin{equation*}
                \dfrac{8\pi^{3}}{3} + \dfrac{8\pi^{3}}{3} = \dfrac{16\pi^{3}}{3}
            \end{equation*}
        and we are done.
    \end{proof}

\textbf{Exercise 5}: Find $\lim\limits_{x \to 0}\frac{\int_{0}^{x} e^{t^{2}} \, \dd{t} }{x}$.
    \begin{proof}
        We have if $F(x) = \int_{0}^{x} e^{t^{2}} \, \dd{t} $, taking the limit of the numerator and denominator gives $0$. Then $F^{\prime}(x) = e^{x^{2}}$ and $x^{\prime} = 1$. So 
            \begin{equation*}
                \lim\limits_{x \to 0}F^{\prime}(x) = 1
            \end{equation*}
        which is therefore the limit of the top thing by L'Hopital. On the other hand, you can also view the fraction as 
            \begin{equation*}
                \dfrac{F(x)}{x}
            \end{equation*}
        and see that it is the limit of $F^{\prime}(x)$ as $x \rightarrow 0$.
    \end{proof}

\textbf{Exercise 6}: Let $f$ be a continuous function on $\mathbb{R}$ and define $F(x) = \int_{x - 1}^{x + 1} f(t) \, \dd{t} $. Prove that $F(x)$ is differentiable and find $F^{\prime}(x)$.
    \begin{proof}
        $F(x)$ is differentiable because we have 
            \begin{equation*}
                \int_{x - 1}^{x + 1} f(t) \, \dd{t}  = \int_{c}^{x + 1} f(t) \, \dd{t}  - \int_{c}^{x - 1} f(t) \, \dd{t} 
            \end{equation*}
        Since $f$ continuous, by fundamental theorem, we get that $F(x)$ is differentiable. Also we let 
            \begin{equation*}
                F_{1} = \int_{c}^{x + 1} f(t) \, \dd{t} \text{ and } F_{2} = \int_{c}^{x - 1} f(t) \, \dd{t} 
            \end{equation*}
        So $F_{1}^{\prime}(x) = f(x + 1)$ and $F_{2}^{\prime}(x) = f(x - 1)$. Then 
            \begin{equation*}
                F^{\prime}(x) = F_{1}^{\prime}(x) - F_{2}^{\prime}(x) = f(x + 1) - f(x - 1)
            \end{equation*}
    \end{proof}



\end{document}

