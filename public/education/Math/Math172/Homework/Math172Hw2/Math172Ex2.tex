%! TeX root = /Users/trustinnguyen/Downloads/Berkeley/Math/Math172/Homework/Math172Hw2/Math172Ex2.tex

\documentclass{article}
\usepackage{/Users/trustinnguyen/.mystyle/math/packages/mypackages}
\usepackage{/Users/trustinnguyen/.mystyle/math/commands/mycommands}
\usepackage{/Users/trustinnguyen/.mystyle/math/environments/article}

\title{Math172Ex2}
\author{Trustin Nguyen}

\begin{document}

    \maketitle

\reversemarginpar

\textbf{Exercise 8}: What digits are exactly to the left and to the right of the decimal point of $(\sqrt{10} + \sqrt{11})^{1000}$. (Explain your answer without using calculator to compute powers of $(\sqrt{10} + \sqrt{11})$. You can use calculators to estimate $\sqrt{10}$ and $\sqrt{11}$).
    \begin{proof}
        We have by the binomial theorem that:
            \begin{equation*}
                (\sqrt{10} + \sqrt{11})^{1000} = \sum_{i = 0}^{1000} \dbinom{1000}{i}(\sqrt{10})^{i}(\sqrt{11})^{1000 - i}
            \end{equation*}
    \end{proof}



\end{document}

