%! TeX root = /Users/trustinnguyen/Downloads/Berkeley/Math/Math172/Homework/Math172Hw8/Math172Hw8.tex

\documentclass{article}
\usepackage{/Users/trustinnguyen/.mystyle/math/packages/mypackages}
\usepackage{/Users/trustinnguyen/.mystyle/math/commands/mycommands}
\usepackage{/Users/trustinnguyen/.mystyle/math/environments/article}

\title{Math172Hw8}
\author{Trustin Nguyen}

\begin{document}

    \maketitle

\reversemarginpar

\textbf{Exercise 1}: Compute all the numbers $p(1), p(2), \ldots, p(10)$ using the pentagonal numbers.
    \begin{answer}
        Using $p(n) = \sum_{i \neq 0} (-1)^{i - 1}p(n - i(3i - 1)/2)$ and $p(0) = p(1) = 1$,  we have
            \begin{align*}
                p(1)  &= 1                              \\
                p(2)  &= p(1) + p(0) = 2                \\
                p(3)  &= p(2) + p(1) = 3                \\
                p(4)  &= p(3) + p(2) = 5                \\
                p(5)  &= p(4) + p(3) - p(0) = 7         \\
                p(6)  &= p(5) + p(4) - p(1) = 11        \\
                p(7)  &= p(6) + p(5) - p(2) - p(0) = 15 \\
                p(8)  &= p(7) + p(6) - p(3) - p(1) = 22 \\
                p(9)  &= p(8) + p(7) - p(4) - p(2) = 30 \\
                p(10) &= p(9) + p(8) - p(5) - p(3) = 42   
            \end{align*}
        is the answer
    \end{answer}

\textbf{Exercise 2}: Recall that the Bernoulli numbers $B_{n}$ are defined by $\frac{x}{1 - \exp(-x)} = \sum_{n \geq0}B_{n} \frac{x^{n}}{n!}$ (during the lecture there was a sign error in this definition, here is the correct version).  These numbers arise in the formula
    \begin{equation*}
        1^{k} + 2^{k} + 3^{k} + \cdots + n^{k} = \sum_{i = 0}^{k}B_{k - i}\dbinom{k}{i}\dfrac{n^{i + 1}}{i + 1}
    \end{equation*}
    \begin{itemize}
        \item Compute the numbers $B_{0}, B_{1}, B_{2}$.
            \begin{answer}
                For $B_{0}$ set $k = 0, n = 1$. Then we get:
                    \begin{equation*}
                        1^{0} = \sum_{i = 0}^{0} B_{0 - i}\dbinom{0}{i}\dfrac{1}{i + 1} = B_{0}
                    \end{equation*}
                Then for $B_{1}$, set $k = 1, n = 1$. So we get:
                    \begin{equation*}
                        1^{1} = \sum_{i = 0}^{1}B_{1 - i}\dbinom{1}{i}\dfrac{1}{i + 1} = B_{1} + \dfrac{1}{2}B_{0}
                    \end{equation*}
                Then for $B_{2}$, set $k = 2, n = 1$. We then get:
                    \begin{equation*}
                        1^{2} = \sum_{i = 0}^{2}B_{2 - i}\dbinom{2}{i}\dfrac{1}{i + 1} = B_{2} + B_{1} + \dfrac{1}{3}B_{0}
                    \end{equation*}
                So overall, $B_{0} = 1$, 
                    \begin{equation*}
                        B_{1} + \dfrac{1}{2}B_{0} = 1
                    \end{equation*}
                which means
                    \begin{equation*}
                        B_{1} + \dfrac{1}{2} = 1, B_{1}  = \dfrac{1}{2}
                    \end{equation*}
                Then finally,
                    \begin{align*}
                        B_{2} + B_{1} + \dfrac{1}{3}B_{0}   &= 1            \\
                        B_{1} + \dfrac{1}{2} + \dfrac{1}{3} &= 1            \\
                        B_{1} + \dfrac{5}{6}                &= 1            \\
                        B_{1}                               &= \dfrac{1}{6}   
                    \end{align*}
                Overall,
                    \begin{center}
                        \begin{tabular}{ c c c }
                            \hline $B_{0}$ & $B_{1}$       & $B_{2}$       \\
                            \hline $1$     & $\frac{1}{2}$ & $\frac{1}{6}$   
                        \end{tabular}
                    \end{center}
            \end{answer}

        \item Show that all the numbers $B_{2i + 1}$ are equal to $0$ with the exception of $B_{1}$. 
            \begin{proof}
                I couldn't solve it but here is some work I did: 

                There is the decomposition:
                    \begin{equation*}
                        1 = \sum_{i = 0}^{k}B_{k - i}\dbinom{k}{i}\dfrac{1}{i + 1} = \sum_{i = 0}^{k - 1}B_{k - i}\dbinom{k}{i}\dfrac{1}{i + 1} + \sum_{i = 0}^{k - 1}B_{k - i - 1}\dbinom{k}{i}\dfrac{1}{ i + 2}
                    \end{equation*}
                We can complete the left summand on the RHS:
                    \begin{equation*}
                        1 = - \dfrac{B_{0}}{k + 1} + \sum_{i = 0}^{k}B_{k - i}\dbinom{k}{i}\dfrac{1}{i + 1} + \sum_{i = 0}^{k - 1}B_{k - i - 1}\dbinom{k}{i}\dfrac{1}{ i + 2}
                    \end{equation*}
                So
                    \begin{equation*}
                        \dfrac{1}{k + 1} = \sum_{i = 0}^{k - 1}B_{k - i- 1} \dbinom{k}{i}\dfrac{1}{i + 2}
                    \end{equation*}
                Add $B_{k}$ to both sides:
                    \begin{equation*}
                        B_{k} + \dfrac{1}{k + 1} = \sum_{i = 0}^{k - 1}B_{k - i}\dbinom{k}{i + 1}\dfrac{1}{i + 1}
                    \end{equation*}
            \end{proof}
    \end{itemize}

\textbf{Exercise 3}: This is a continuation of Problem $3$ in Problem sets $6, 7$. Define the following formal power series:
    \begin{equation*}
        \ln(1 + x) = \sum_{k \geq1}(-1)^{k - 1}\dfrac{x^{k}}{k}
    \end{equation*}
    \begin{itemize}
        \item Show that $\dv{x}\exp(x) = \exp(x)$ and $\dv{x}\ln(1 + x) = \frac{1}{1 + x}$.
            \begin{proof}
                We have $\exp(x) = \sum_{i \geq0} \frac{x^{i}}{i!}$. Then taking the derivative:
                    \begin{equation*}
                        \dv{x}\sum_{i \geq0}\dfrac{x^{i}}{i!} = \sum_{i \geq1} \dfrac{x^{i - 1}}{(i - 1)!} = \sum_{i \geq 0}\dfrac{x^{i}}{i!} = \exp(x)
                    \end{equation*}
                And now for $\ln(1 + x)$, use the definition and take the derivative:
                    \begin{align*}
                        \ln(1 + x) &=           \sum_{k \geq 1}(-1)^{k - 1}\dfrac{x^{k}}{k}         \\
                                   &\rightarrow  \dv{x}\sum_{k  \geq 1}(-1)^{k - 1}\dfrac{x^{k}}{k} \\
                                   &=           \sum_{k \geq 1}(-1)^{k - 1}x^{k - 1}                \\
                                   &=           \sum_{k \geq 0}(-1)^{k}x^{k}                        \\
                                   &=           \sum_{k \geq 0}(-x)^{k}                             \\
                                   &=           \dfrac{1}{1 - (-x)}                                 \\
                                   &=           \dfrac{1}{1 + x}                                      
                    \end{align*}
                which concludes the proof.
            \end{proof}

        \item Show that $\ln(\exp(x)) = x$, where the left hand side is the result of plugging $\exp(x) - 1$ instead of the variable $t$ into $\ln(1 + t)$. Note that in class we have mentioned, without proof, that the identity from part $(2)$ implies that $\exp(\ln(1 + x)) - 1 = x$, which is nontrivial to show by direct computation.
            \begin{proof}
                Consider the derivative of this term. Last time, it was proved that $\dv{x}F(G(x)) = G^{\prime}(x)F(G(x))$ for two formal power series $F(x), G(x)$. Then
                    \begin{align*}
                        \dv{x}\ln(\exp(x)) &= \dv{x}\ln\left(1 + \sum_{k \geq 1}\dfrac{x^{k}}{k!}\right) \\ 
                                           &= \exp(x)\dfrac{1}{1 + \sum_{k \geq 1}\frac{x^{k}}{k!}} \\
                                           &= \exp(x)\dfrac{1}{\sum_{k \geq 0}\frac{x^{k}}{k!}} \\
                                           &= \exp(x)\dfrac{1}{\exp(x)} \\
                                           &= 1
                    \end{align*}
                Taking the anti-derivative with respect to $x$, we get $x + c$. But we see that
                    \begin{equation*}
                        \ln(\exp(x)) = \sum_{j \geq 1}(-1)^{j - 1}\dfrac{\left(\sum_{k \geq 1}\frac{x^{k}}{k!}\right)^{j}}{j}
                    \end{equation*}
                which $x$ divides because for $\sum_{k \geq 1}\frac{x^{k}}{k!}$, $x$ divides this. Therefore, $x \divides x + c$, $c = 0$. So $\ln(\exp(x)) = x$.
            \end{proof}
    \end{itemize}

\textbf{Exercise 4}: Find the exponential generating function for the number of derangements of $[n]$ (we have found this number earlier using inclusion-exclusion).
    \begin{proof}
        The number of derangements of $[n]$ was given by 
            \begin{equation*}
                n!\sum_{i = 0}^{n}(-1)^{i}\dfrac{1}{i!}
            \end{equation*}
        Then we get for the exponential generating function:
            \begin{equation*}
                \sum_{n \geq 0} \left(\sum_{i = 0}^{n} (-1)^{i}\dfrac{1}{i!}\right)x^{n}
            \end{equation*}
        But the coefficient of each power of $x$ is given by the columns of:
            \begin{align*}
                1 - x + \dfrac{x^{2}}{2!} &-          \dfrac{x^{3}}{3!} + \dfrac{x^{4}}{4!} - \dfrac{x^{5}}{5!} + \cdots \\
                 x - x^{2}                &+          \dfrac{x^{3}}{2!} - \dfrac{x^{4}}{3!} + \dfrac{x^{5}}{4!} - \cdots \\
                +x^{2}                    &-          x^{3} + \dfrac{x^{4}}{2!} - \dfrac{x^{5}}{3!} + \cdots             \\
                                          &+          x^{3} - \dfrac{x^{4}}{1!} + \dfrac{x^{5}}{2!} - \cdots             \\
                                          &     \hspace{20pt} \ddots                                                              
            \end{align*}
        Then we can rewrite the generating function as:
            \begin{equation*}
                \sum_{n \geq 0} x^{n}\exp(-1)
            \end{equation*}
        since each row in the sum represented $x^{i}\exp(-x)$. So we substituted $x = 1$. So that is the exponential generating function for the number of derangements.
    \end{proof}










\end{document}
