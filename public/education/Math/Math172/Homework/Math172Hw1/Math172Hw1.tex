%! TeX root = /Users/trustinnguyen/Downloads/Berkeley/Math/Math172/Homework/Math172Hw1/Math172Hw1.tex

\documentclass{article}
\usepackage{/Users/trustinnguyen/.mystyle/math/packages/mypackages}
\usepackage{/Users/trustinnguyen/.mystyle/math/commands/mycommands}
\usepackage{/Users/trustinnguyen/.mystyle/math/environments/article}

\title{Math172Hw1}
\author{Trustin Nguyen}

\begin{document}


\begin{titlepage}
    \maketitle
\end{titlepage}

\reversemarginpar

\textbf{Exercise 1}: Prove the following version of the Pigeon-hole principle: ``If the sum of $k$ real numbers is 1, then at least one of these real numbers must be greater or equal than $\frac{1}{k}$''. Use this version to find all triples of positive integers $a < b < c$ such that 
    \begin{equation*}
        \dfrac{1}{a} + \dfrac{1}{b} + \dfrac{1}{c} = 1
    \end{equation*}
    \begin{proof}
        Suppose for contradiction that the sum of $k$ real numbers $r_{1}, \ldots, r_{k}$ is 1 but that each number is less than $\frac{1}{k}$. Then we have the inequalities:
            \begin{align*}
                r_{1} &<      \dfrac{1}{k} \\
                      &\vdots              \\
                r_{k} &<      \dfrac{1}{k}   
            \end{align*}
        But this gives us:
            \begin{equation*}
                \sum_{i = 1}^{k} r_{k} <  \sum_{i = 1}^{k} \dfrac{1}{k} = 1
            \end{equation*}
        so we have a contradiction.

        By the problem restrictions, we have $\frac{1}{a}, \frac{1}{b}, \frac{1}{c} < 1$. By the previous proof, we have the bound on one number:
            \begin{equation*}
                \dfrac{1}{3} \leq \frac{1}{a} < 1
            \end{equation*}
        so
            \begin{equation*}
                1 < a \leq 3
            \end{equation*}
        If $a = 3$:
            \begin{equation*}
                \frac{1}{b} + \frac{1}{c} = \frac{2}{3}
            \end{equation*}
        If we use the same proof idea, then one of the components must be greater than or equal to 
            \begin{equation*}
                \dfrac{\frac{2}{3}}{2} = \frac{1}{3}
            \end{equation*}
        So we must have $b = 2$. But now 
            \begin{equation*}
                \frac{1}{c} = \dfrac{2}{3} - \dfrac{1}{2} = \dfrac{4}{6} - \dfrac{3}{6} = \dfrac{1}{6}
            \end{equation*}
        For the second case, we have $a = 2$. But wlog, that still points to the same solution, since we have $b = 2$ in the first case. So the only solution with $a < b < c$ is $2 < 3 < 6$
    \end{proof}

\textbf{Exercise 2}: Prove that among any $52$ integers there are always two integers such that either their sum or difference is divisible by 100.
    \begin{proof}
        Make $51$ pairs of numbers:
            \begin{equation*}
                \{(a, b): a \in \{1, \ldots, 50\} \land b = 100 - a\}
            \end{equation*}
        Notice that if we choose 52 numbers, then by the pigeonhole principle, then at least two numbers, mod 100 will lie in the same pair. This means that if they are equal mod 100, then $a \equiv b \pmod{100}$ or if they are not equal mod 100, then $a \mod{100} + b \mod{100} = 100$. The first case implies that $a - b \equiv 0 \pmod{100}$, as desired. The second case shows that $a + b \mod{100} = a \mod{100} + b \mod{100} = 100 \mod{100} = 0$. We are done.
    \end{proof}

\textbf{Exercise 3}: Set $a_{0} = 3$ and let $a_{n + 1} = \sqrt{a_{n} + 7}$. Prove that $3 < a_{n} < 4$ for all $n > 0$.
    \begin{proof}
        We will proceed by induction. 
            \begin{itemize}
                \item Base Case: Observe that $a_{1} = \sqrt{3 + 7}$ and $\sqrt{9} <  \sqrt{10} < \sqrt{16}$ so we have as desired.

                \item Inductive Case: Suppose that this is true for $a_{1}, \ldots, a_{n}$. We will show this for $a_{n + 1}$. We have that:
                    \begin{equation*}
                        a_{n + 1} = \sqrt{a_{n} + 7}
                    \end{equation*}
                But by inductive hypothesis, we have 
                    \begin{equation*}
                        \sqrt{3 + 7} < a_{n + 1} < \sqrt{4 + 7}
                    \end{equation*}
                This means that:
                    \begin{equation*}
                        3 <  \sqrt{3 + 7} < a_{n + 1} <  \sqrt{4 + 7} <  4
                    \end{equation*}
                as desired.
            \end{itemize}
        * The proof uses the fact that $\sqrt{n} <  \sqrt{n + 1}$ for $n \geq 0$. This can be proved by claiming that $n^{2} < (n + 1)^{2}$ from the fact that $(n + 1)^{2} - n^{2} = 2n + 1 >  0$ for $n \geq  0$
    \end{proof}

\textbf{Exercise 4}: Show that for any positive integer $n$
    \begin{equation*}
        1 + 3 + 5 + \cdots + (2n - 1) = n^{2}
    \end{equation*}
    \begin{proof}
        We will show this by induction.
            \begin{itemize}
                \item Base Case: This is true as for $n = 1$, $2n - 1 = 1 = 1^{2}$.

                \item Inductive Case: Suppose that this is true for $1, \ldots, n$. We will show that this holds for $n + 1$. 
                    Observe that:
                        \begin{equation*}
                            1 + 3 + \ldots + (2n - 1) + (2n + 1) = n^{2} + 2n + 1 = (n + 1)^{2}
                        \end{equation*}
                    So indeed:
                        \begin{equation*}
                            1 + 3 + \ldots + (2(n + 1) - 1) = (n + 1)^{2}
                        \end{equation*} 
                    This concludes the proof.
            \end{itemize}
    \end{proof}

\textbf{Exercise 5}: Let $H$ be a ten-element set of two-digit positive integers. Prove that there are two disjoint (i.e non-intersecting) subsets $A$ and $B$ such that the sum of elements in $A$ is equal to the sum of elements in $B$.
    \begin{proof}
        Notice that $H$ has $2^{10}$ possible subsets or $1028$ possible sums. But the largest sum we can make is 
            \begin{equation*}
                \underbrace{99 + 99 + \ldots + 99}_{10}
            \end{equation*}
        or $990$. But we have $1028$ sums, so two of the sums must be equal by the pigeonhole principle. So two subsets must contain the same sum call these $A, B$. They might not be disjoint so our solution instead will be 
            \begin{align*}
                A^{\prime} = A - (A \cap B) \\
                B^{\prime} = B - (A \cap B)
            \end{align*}
        These will have the same sum. This concludes the proof.
    \end{proof}

\textbf{Exercise 6}: How much time have you spent on this problem set? On a scale between 0 (trivial) and $10$ (impossible), how hard this problem set was?

    \hspace{30pt} I spent roughly $1\frac{1}{2}$ hours on the set, including some time typing up the solutions. I would say that the difficulty is $3$.





\end{document}
