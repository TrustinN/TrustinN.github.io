%! TeX root = 	

\documentclass{article}
\usepackage{/Users/trustinnguyen/MyStyle/mystyle}

\begin{document}
\begin{topic}
    \section*{Final Examples}
\end{topic}

\textbf{Exercise 1}: Let $R$ be an integral domain with field of fractions $F$. Suppose that $\varphi : R \rightarrow K$ is an injective homomorphism from $R$ to a field $K$. Show that $\varphi$ extends to an injective homomorphism $\varPhi : F \rightarrow K$. When is $\varphi$ not injective?

\begin{proof}
	Suppose that $\varphi$ is injective. Then define the homomorphism $\theta: F \rightarrow K$ as defined as 
	\begin{equation*}
		\theta (a, b) \mapsto \dfrac{\varphi(a)}{\varphi(b)} 
	\end{equation*}
	We get that $\theta$ is injective because $\varphi$ is injective. It follows that for $\frac{\varphi(a)}{\varphi(b)}$ to equal 0, we have that $\varphi(a) = 0$ and therefore, $a = 0$. Since $(0, b)$ is the zero element in $F$, we get that the $\ker{\theta} = \left\{0\right\}$.
\end{proof}

\textbf{Exercise 2}: Let $R$ be a ring. Show that $R[x]$ is a PID iff $R$ is a field.

\begin{proof}
	$(\rightarrow)$ Supose that $R[x]$ is a PID. Then consider the ideal generated by $x$. We will show that this is a maximal ideal and therefore, $R[x]/(x) \cong R$ is a field. Suppose that $(x, f)$ is an ideal. Since $R[x]$ is a PID, it must be generated by a single element. Furthermore, constants live in our ideal $(x, f)$ so that element must divide constants and therefore be a constant. So 
	\begin{align*}
		(c) = (x, f) \\
		x = cf'\\
		\deg{x} = \deg{c} + \deg{f'} \\
		\deg{f'} = 1 \\
		x = c(ax + b) \\
		x = cax + cb \\
		x = cax 
	\end{align*}
so $c$ is a unit and therefore, $(x)$ is maximal.  
\end{proof}

\textbf{Exercise 3}: If $S$ is a set of primes, let $\mathbb{Z}_{S}$ be the set of all rational numbers \textit{m/n} (in lowest terms) such that all prime factors of $n$ are in $S$. If $R$ is a subring of 
$\mathbb{Q}$ show that there is a set of primes $S$ such that $R$ is of the form $\mathbb{Z}_{S}$. What are the maximal subrings of $\mathbb{Q}$?

\begin{proof}
	
\end{proof}

\textbf{Exercise 4}: 
\begin{enumerate}
	\item Consider $f(x,y) = x^{3}y + x^{2}y^{2} + y^{3} - y^{2} - x - y + 1$ in $\mathbb{C}[x,y]$. Show that $f$ is prime.
		\begin{proof}
			We can rewrite the polynomial as an element in $\mathbb{C}[x][y]$:
			\begin{equation*}
				f(x,y) = y^{3} + (x^{2} - 1)y^{2} + (x^{3} - 1)y - (x - 1)
			\end{equation*}
			by eisenstein, this polynomial is irreducible because $x - 1$ divides all coefficients besides the first one, the polynomial is primitve, and $(x - 1)^{2}$ does not divide the last coefficient. We just need to check that $x - 1$ is irreducible:
			\begin{align*}
				x - 1 = fg \\
				\deg(x - 1) = \deg(f) + \deg(g) \\
				1 = 0 + 1 \\
				x - 1 = c(ax + b) \\
				x - 1 = cax + cb \\
				ca = 1
			\end{align*}
			therefore, $f = c$ is a unit and $x - 1$ is irreducible.
		\end{proof}

	\item Let $F$ be any field. Show that $f(x,y) = x^{2} + y^{2} - 1$ is irreducible in $\mathbb{F}[x,y]$ unless $\mathbb{F}$ has characteristic $2$. What happens in that case?
		\begin{proof}
			We can rewrite this equation as an element in $\mathbb{F}[x][y]$:
			\begin{equation*}
				f(y) = y^{2} + x^{2} -1 
			\end{equation*}
			and by eisenstein again, the polynomial $x - 1$ is irreducible such that the polynomial $f(x, y)$ is irreducible. Now if $\mathbb{F}$ has characteristic $2$, then we have 
			\begin{equation*}
				x^{2} + y^{2} - 1 = x^{2} + y^{2} + 1
			\end{equation*}
			but 
			\begin{equation*}
				(x + y + 1)^{2} = x^{2} + y^{2} + 1 + 2(x + y + xy) = x^{2} + y^{2} + 1
			\end{equation*}
			so this polynomial is not irreducible when $\mathbb{F}$ has characteristic $2$.
		\end{proof}
\end{enumerate}
\textbf{Exercise 5}: 
\begin{enumerate}
	\item Show that if $R$ is a PID, the gcd of $a, b \in R$ can be written as $ra + sb$ for some $r, s\in R$. Give an example of a UFD where this fails.
		\begin{proof}
			If $R$ is a PID, then we have that if $a, b \in R$, then considering the ideal generated by $a, b$:
			\begin{equation*}
				(a, b) = (s)
			\end{equation*}
			since the gcd of $a, b$ divides both elements, we can let $s$ be the gcd. This tells us that 
			\begin{equation*}
				s = r_{1}a + r_{2}b
			\end{equation*}
			which completes the proof. Now for an example of a UFD which does not satisfy the property, we can take $\mathbb{Z}[x]$ which is a UFD but not a PID since $(x, 2)$ is not a principal ideal. We can find the gcd of two elements in $\mathbb{Z}[x]$ such as $x^{2} + x + 1$ and $x^{2} + 1$. Such elements are irreducible because these polynomials have all their roots in $\mathbb{C}$ so if we could factor any of them to get polynomials of lower degree, that would imply that the complex numbers are in $\mathbb{Z}$ which is impossible. Notice that the gcd of the polynomials is $1$ but there is no way to write them as such:
			\begin{align*}
				r(x^{2} + x + 1) + s(x^{2} + 1) = 1 \\
				(r + s)x^{2} + rx + r + s = 1
			\end{align*}
			which does not work because the coefficient for $x$ in the LHS is non-zero.
		\end{proof}
	\item Find the gcd of $11 + 7i$ and $18 - i$ in $\mathbb{Z}[i]$.
		\begin{proof}

			We first take the norm of both numbers to find the prime factorization:
			\begin{align*}
				N(11 + 7i) = 121 + 49 = 170 = 17 * 5 * 2 \\
				N(18 - i) = 324 + 1 = 325 = 5^{2} * 13 
			\end{align*}
			Now we use the fact that integers $p \equiv 3 \pmod{4}$ are prime and that the ones $p \equiv 1 \pmod{4}$ and equal to $2$ can be written as a sum of squares and therefore factorizable in $\mathbb{Z}[i]$:
			\begin{align*}
				N(11 + 7i) = (1 \pm 4i)(1 \pm 2i)(1 \pm i)
				N(13 - i) = (1 \pm 2i)^{2}(2 \pm 3i)
			\end{align*}
			so our only choice is that either $1 + 2i$, $1 - 2i$, $2 + i$, or $2 - i$ divides both elements or otherwise, their gcd is 1. We can check:
			\begin{align*}
				\dfrac{11 + 7i}{1 + 2i} = \dfrac{11 + 14 - 22i + 7i}{5} = \dfrac{25 - 15i}{5} = 5 - 3i\\
				\dfrac{13 - i}{1 + 2i} = \dfrac{13 - 2 + 26i - i}{5} = \dfrac{11 + 25i}{5}
			\end{align*}
			But this shows that $1 + 2i$ divides $11 + 7i$ but not $18 - i$, so that one does not work. This tells us that $2 - i$ does not work also because $1 + 2i$ divides this. So we are done, the gcd is 1.
		\end{proof}
\end{enumerate}


\end{document}
