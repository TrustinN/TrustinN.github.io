%! TeX root =   

\documentclass{article}

\usepackage{/Users/trustinnguyen/MyStyle/mystyle}

\title{Math113Hw6}
\author{Trustin Nguyen}


\begin{document}
\maketitle
\reversemarginpar

\begin{topic}
    \section*{Homework 6}
\end{topic}

\textbf{Exercise 1}: Let $\omega = \frac{1}{2}(-1 + \sqrt{-3}) \in \mathbb{C}$. Recall that we wrote $\mathbb{Z}[\omega] = \{a + b\omega : a, b \in \mathbb{Z}\}$ and similarly $\mathbb{Q}[\omega] = \{a + b\omega : a, b \in \mathbb{Q}\}$. Show that $\mathbb{Z}[\omega]$ is a subring of $\mathbb{C}$ while $\mathbb{Q}[\omega]$ is  subfield. What are the units in $\mathbb{Z}[\omega]$?
    \begin{proof}
        Clearly, $\mathbb{Z}[\omega], \mathbb{Q}[\omega]$ are nonempty as they both have $0$. Suppose that
        \begin{equation*}
            a + \dfrac{b}{2}(-1 + \sqrt{-3}), \, c + \dfrac{d}{2}(-1 + \sqrt{-3}) \in \mathbb{Z}[\omega]
        \end{equation*}
        Observe that
        \begin{equation*}
            a + \dfrac{b}{2}(-1 + \sqrt{-3}) - c - \dfrac{d}{2}(-1 + \sqrt{-3}) = (a - c) - \dfrac{b - d}{2}(-1 - \sqrt{-3}) \in \mathbb{Z}[\omega]
        \end{equation*}
        Now for multiplicative closure:
        \begin{gather*}
            (a + \dfrac{b}{2}(-1 + \sqrt{-3}))(c + \dfrac{d}{2}(-1 + \sqrt{-3})) = \\
            ac + \dfrac{bc + ad}{2}(-1 + \sqrt{-3}) + \dfrac{bd}{4}(-2 - 2\sqrt{-3}) = \\
            ac - bd + \dfrac{ad - bd + ad}{2}(-1 + \sqrt{-3}) \in \mathbb{Z}[\omega]
        \end{gather*}
        The same argument works for $\mathbb{Q}[\omega]$. Also, $1 \in \mathbb{Z}[\omega], \mathbb{Q}[\omega]$ given by $1 + 0\omega$, so it has the identity elements $0, 1$. Suppose that $a + \dfrac{b}{2}(-1 + \sqrt{-3}) \in \mathbb{Z}[\omega]$. Then to find its inverse, observe that it would be 
        \begin{align*}
            \dfrac{1}{a + \dfrac{b}{2}(-1 + \sqrt{-3})} &= \dfrac{1}{a - \dfrac{b}{2} + \dfrac{b}{2}\sqrt{-3}} \\
                                                        &= \dfrac{a - \dfrac{b}{2} - \dfrac{b}{2}\sqrt{-3}}{\left( a - \dfrac{b}{2} \right)^{2} - \left( \dfrac{b}{2}\sqrt{-3} \right)^{2}} \\
                                                        &= \dfrac{a - \dfrac{b}{2}\left( 1 + \sqrt{-3} \right)}{\left( a - \dfrac{b}{2} \right)^{2} - \left( \dfrac{b}{2}\sqrt{-3} \right)^{2}} \\
                                                        &= \dfrac{a - b - \dfrac{b}{2}(-1 + \sqrt{-3})}{\left( a - \dfrac{b}{2} \right)^{2} - \left( \dfrac{b}{2}\sqrt{-3} \right)^{2}} \in \mathbb{Q}[\omega]
        \end{align*}
        so all elements except $0$ is a unit. So $\mathbb{Q}[\omega]$ is a subfield.
    \end{proof}

\textbf{Exercise 2}: Let $R$ be a non-zero ring. An element $r \in R$ is called nilpotent if $r^{n} = 0$ for some positive integer $n$. 
    \begin{enumerate}
        \item What are the nilpotent elements in $\mathbb{Z}/6\mathbb{Z}$?

            \begin{proof}
                The elements of $\mathbb{Z}/6\mathbb{Z}$are 
                \begin{equation*}
                    \{[0], [1], [2], [3], [4], [5]\}
                \end{equation*}
                Observe that $[0]$ is naturally nilpotent. For $[2]$, we have
                \begin{align*}
                    [2]^{2} &= 4 \\
                    [2]^{3} = [8] &= [2]
                \end{align*}
                so no powers of $[2]$ will be 0. For $[3]$
                \begin{equation*}
                    [3]^{2} = [9] = [3]
                \end{equation*}
                which means $[3]$ is not nilpotent. For $[4]$, 
                \begin{equation*}
                    [4]^{2} = [16]  = [4]
                \end{equation*}
                So $[4]$ is not nilpotent. For $[5]$, we have 
                \begin{align*}
                    [5]^{2} = [25] &= [1] \\
                    [5]^{3} &= [5]
                \end{align*}
                so $[5]$ is not nilpotent. $[0]$ is the only nilpotent element of the set.
            \end{proof}

        \item Show that if $r$ is nilpotent, the it's not a unit but $1 + r$ and $1 - 4$ are units.
            \begin{proof}
                If $r$ is nilpotent, suppost that it is a unit, for contradiction. Then there is some $r^{-1}$ such that $r^{-1}r = 1$. But notice that
                \begin{align*}
                    r^{n} &= 0 \\
                    r^{n}r^{-1} &= 0 \\
                    r^{n - 1}r^{-1} &= 0 \\
                                    &\vdots \\
                    rr^{-1} = 1 &= 0
                \end{align*}
                contradiction. So $r$ is not a unit. Now to show that $1 - r$ and $1 + r$ are units, let $r^{n} = 0$ and observe that
                \begin{equation*}
                    (1 - r)(1 + r)(1 + r^{2})(1 + r^{4}) \cdots (1 + r^{2n}) = 1 - r^{2n} = 1
                \end{equation*}
                So $1 - r$ has an inverse which is 
                \begin{equation*}
                    (1 + r)(1 + r^{2})(1 + r^{4}) \cdots (1 + r^{2n})
                \end{equation*}
                and for $1 + r$, it is
                \begin{equation*}
                    (1 - r)(1 + r^{2})(1 + r^{4}) \cdots (1 + r^{2n})
                \end{equation*}
            \end{proof}

        \item Let $N$ be the set of nilpotent elements. Show that it is an ideal in $R$. Describe the nilpotent elements in the quotient $R/N$.

            \begin{proof}
                To show that $N$ is an ideal, we show that it is a group under addition, is non-empty, and is closed under multiplication by elements from $R$.
                \begin{enumerate}
                    \item [(a)] $0 \in R$, $0^{1} = 0$, therefore, $0 \in N$ and $N$ is non-empty.

                    \item [(b)] Suppose $a, b \in N$. Then all terms of 
                        \begin{equation*}
                            (a - b)^{2n}
                        \end{equation*}
                        must have an $a^{k}b^{j}$ such that either $k \geq n$ or $j \geq n$. So
                        \begin{equation*}
                            (a - b)^{2n} = 0
                        \end{equation*}
                        and $N$ contains additive inverses closed under addition.

                    \item [(c)] Suppose $r \in R$, $n \in N$ with $n^{k} = 0$. Then we show that $rn \in N$.
                        \begin{equation*}
                            (rn)^{k} = r^{k}n^{k} = 0
                        \end{equation*}
                        since rings in this class are assumed commutative. So $rn \in N$.
                \end{enumerate} 
                therefore, $N$ is an ideal. Since we quotient out all nilpotent elements for $R/N$, the nilpotent elements in the quotient is the 0 elements or $0 + N$.
            \end{proof}
    \end{enumerate}

\textbf{Exercise 3}: Show that if $I$ and $J$ are ideals in $R$, then so is $I \cap J$ and $R(I \cap J)$ is isomorphic to a subring $R/I \times R/J$. Moreover, if there are $x \in I$ and $y \in J$ with $x + y = 1$, then $R/(I \cap J) \cong R/I \times R/J$.
    \begin{proof}
        (Part I) Consider the homomorphism $\varphi : R \rightarrow R/I \times R/J$
        \begin{equation*}
            \varphi(r) = (r + I, r + J)
        \end{equation*}
        Observe that the kernel is $I \cap J$, since if 
        \begin{equation*}
            \varphi(r) = (r + I, r + J) = (I, J)
        \end{equation*}
        then $r \in I \cap J$ and vice versa. So since the kernel is an ideal, $I \cap J$ is an ideal also. So by isomorphism theorem, $R/(I \cap J) \cong R/I \times R/J$.

        (Part II) If there is an $x \in I, y \in J$ such that $x + y = 1$, we will prove that the mapping given by $\varphi$ is surjective. Suppose that $(a + I, b + j) \in R/I \times R/J$. The consider
        \begin{align*}
            \varphi(ay + bx) &= (ay + I, bx + J) \\
                             &= ((a + I)(y + I), (b + J)(x + J)) \\
                             &= ((a + I)(R), (b + J)(R)) \\
                             &= ((a + I)(1 + I), (j + J)(1 + J)) \\
                             &= (a + I, b + J)
        \end{align*}
        so the map is urjective. By the isomorphism theorem, the domain is isomorphic to the image of the maps, so it is isomorphic to the whole codomain. We are done. 
    \end{proof}

\textbf{Exercise 4}: Let $R$ be a ring. We say that $r \in R$ is idempotent if $r^{2} = r$.
    \begin{enumerate}
        \item Describe the idempotents in $\mathbb{Z}/6\mathbb{Z}$.
            \begin{proof}
                The elements of $\mathbb{Z}/6\mathbb{Z}$ are 
                \begin{equation*}
                    \{[0], [1], [2], [3], [4], [5]\}
                \end{equation*}
                We check each one.
                \begin{gather*}
                    [0]^{2} = [0] \\
                    [1]^{2}  = [1] \\
                    [2]^{2} = [4] \neq [2] \\
                    [3]^{2} = [9] \equiv [3] \\
                    [4]^{2} = [16] \equiv [4] \\
                    [5]^{2} = [25] \equiv [1] \neq [5]
                \end{gather*}
                So the idempotent elements of $\mathbb{Z}/6\mathbb{Z}$ are 
                \begin{equation*}
                    \{[0], [1], [3], [4]\}
                \end{equation*}
            \end{proof}

        \item Show that if $r$ is idempotent, then so is $r^{\prime} = 1 - r$ and $rr^{\prime} = 0$. Furthermore, prove that the ideal $(r)$ is naturally a ring and that $R \cong (r) \times (r^{\prime})$ as rings.
            \begin{proof}
                (Part I) If $r$ is idempotent, then we have
                \begin{equation*}
                    rr^{\prime} = (1 - r)r = r - r^{2} = 0 
                \end{equation*}
                since $r = r^{2}$. So $rr^{\prime}$ is idempotent also. For the second part, Observe that if we take any element of $(r), \lambda r^{k}$, $r$ is the identity of that ideal:
                \begin{equation*}
                    \lambda r^{k} \cdot r = \lambda r^{k + 1} = \lambda r^{k}
                \end{equation*}
                so $(r)$ is a ring.

                (Part II) Consider the mapping $\varphi : R \rightarrow (r) \times (r^{\prime})$
                \begin{equation*}
                    \varphi(r) = (r + (r), r + (r^{\prime}))
                \end{equation*}
                Let $(a + (r), b + (r^{\prime})) \in (r) \times (r^{\prime})$ be arbitrary. We will show that $\varphi$ is surjective. Then consider $ar^{\prime} + br$. We have
                \begin{align*}
                    \varphi(ar^{\prime} + br) &= (ar^{\prime} + br + (r), ar^{\prime} + br + (r^{\prime})) \\
                                              &= (ar^{\prime} + (r), br + (r^{\prime})) \\
                                              &= (a - ar + (r), b(r^{\prime} + 1) + (r^{\prime})) \\
                                              &= (a - ar + (r), br^{\prime} + b + (r^{\prime})) \\
                                              &= (a + (r), b + (r^{\prime}))
                \end{align*} 
                as desired.
            \end{proof}
    \end{enumerate}
\end{document}




























































