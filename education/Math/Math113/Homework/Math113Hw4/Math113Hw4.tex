%! TeX root = 	

\documentclass{article}
\usepackage{/Users/trustinnguyen/MyStyle/mystyle}

\title{Math113Hw4}
\author{Trustin Nguyen}


\begin{document}
\maketitle
\reversemarginpar

\begin{topic}
	\section*{Homework 4}
\end{topic}

\textbf{Exercise 1}: Let $H$ be a normal subgroup of index $m$ in $G$. Show that for any $g \in G, g^{m} \in H$. 
\begin{proof}
	Since $H$ is normal, we know that $G/H$ is a group also which has order $m$. So for a $gH \in G/H$, its order divides that of $m$, the order of the group. Then we have
	\begin{align*}
		(gH)^{m} &= \underbrace{gH \cdots gH}_{m} = H \\
			 &=g^{m}H = H
	\end{align*}
	so $g^{m}$ is in $H$.
\end{proof}

\textbf{Exercise 2}: Let $G$ be the group of symmetries of a quadilateral in plane. Show that $\abs{G} \leq 8$. Find all $n$ for which there exists a quadrilateral with a group of symmetries $G$ such that $\abs{G} = n$.

\begin{proof}
	By the orbit-stablizer theorem, 
	\begin{equation*}
		\abs{G} = \abs{\text{Orb}(x)}\abs{\text{Stab}(x)}
	\end{equation*}
	We know that the orbit of an element is at most $4$ and that the stabilizer is at most 2, since we have $e$ and the one line of symmetry that goes through $x$. For the second part, observe that we can have the full symmetry of a square which is symmetric under the permutations of $D_{8}$. Then, there are symmetries which are the subgroups of the $D_{8}$. We have a subgroup generated by disjoint transpositions (two lines of symmetries) which has order 4. Then one line of symmetry which gives 2. Finally, there is the group $\{e\}$.
\end{proof}

\textbf{Exercise 3}: Let $G$ be a group. The center of $G$, denoted $Z(G)$ is the set
\begin{equation*}
	\{z \in G \divides \forall g \in G, zg = gz\}
\end{equation*}
\begin{enumerate}
	\item Show that $Z(G)$ is a normal subgroup.
		\begin{proof}
			By definition, we observe that for every  $g \in G$, $gzg^{-1}$ lies in the center.
		\end{proof}

	\item Let $G$ be a ginite group of order $p^{a}$ where $p$ is a prime number and $a > 0$. Show that the center of $G$ is non-trivial.
		\begin{proof}
			Suppose for contradiction that the normal group is trivial. Then consider the group action of $G$ on itself by conjugation:
			\begin{equation*}
				x \mapsto gxg^{-1}
			\end{equation*}
			Also consider the orbits of the group action and the fact that they all must divide $p^{a}$ by orbit stabilizer:
			\begin{equation*}
				\abs{G} = \sum_{}^{}\abs{\text{Orb}(x)}
			\end{equation*}
			consider the orbit of $e$, which is $geg^{-1} = e$. The order is 1 and by the orbit stabilizer theorem, its stabilizer has order $p^{a}$, and therefore, $e$ commutes with every element in $G$. So elements in orbits with order 1 belong in $Z(G)$ and vice versa. Since the order of orbits divide $p^{a}$,  we have now that
			\begin{equation*}
				\abs{\text{Orb}(e)} + \abs{\text{Orb}(x_{1})} + \cdots + \abs{\text{Orb}(x_{n})} = p^{a}
			\end{equation*}
			Since $Z(G)$ is trivial, we have 
			\begin{equation*}
				1 + p(b) = p^{a}
			\end{equation*}
			for some $b \in \mathbb{Z}$. But that is impossible since $1 \not\equiv 0\pmod{p}$.
		\end{proof}

	\item Show that any group of order $p^{2}$ is abelian and that, up to isomorphism, there are only two such groups for each prime $p$.
		\begin{proof}
			The group has order $p^{2}$, so it can have an element of order $p^{2}$, which means that lal subgroups of the group is normal to $G$. Second case is if it has two cyclic subgroups generated by elements of order $p$. But since $p$ is the smallest prime dividing the order of the gorup, the subgroup of index $p$ must be normal. This means that the two cyclic subgroups are normal in $G$ and terefor, all of $G$ is abelian. We note that since $\gcd{(p,p)} = p \neq 1$, we know that $C_{p} \times C_{p} \not\cong C_{p^{2}}$. So there are two distance groups for every $p$.
		\end{proof}
\end{enumerate}
\textbf{Exercise 4}: Let $\mathbb{R}^{3}$ be equipped with the standard Euclidean metric
\begin{equation*}
	d(x,y) = \sqrt{(x_{1} - y_{1})^{2} + (x_{2} - y_{2})^{2} + (x_{3} - y_{3})^{2}}.
\end{equation*}
Recall that an isometry of $\mathbb{R}^{3}$ is a bijection $f: \mathbb{R}^{3} \rightarrow \mathbb{R}^{3}$ which preserves metric i.e., 
\begin{equation*}
	d(f(x), f(y)) = d(x, y)
\end{equation*}
Let $I(\mathbb{R}^{3})$ be the gorup of isometries of $\mathbb{R}^{3}$. for a subset $\Pi \subseteq \mathbb{R}^{3}$, let $G(\Pi)$ be the subgroup of $I(\mathbb{R}^{3})$ which preserves $\Pi$. It's called the group of symmetries of $\Pi$. Let $\Pi$ be a regular tetrahedron in $\mathbb{R}^{3}$. Let $G(\Pi)$ be the group of symmetries of $\Pi$.
\begin{enumerate}
	\item Let $R$ be the subgroup of $G(\Pi)$ of all rotations. Find $\abs{R}$.
		\begin{proof}
			By the orbit stabilizer theorem, the action is transitive, since we can reach any other point by rotation. We also know that there are three actions that map a point to itself such as the identity, a single rotation about the axis going through the point, and a double rotation about the axis through the point. By the orbit stabilizer, $\abs{R} = 12$.
		\end{proof}

	\item What standard group is $R$ isomorphic to? Justify your answer.
		\begin{proof}
			$R$ is isomorphic to the alternating group $A_{4}$. This is because each rotation is a fixing of one of the four corners of the square, and a rotation of the three other points. We also know that the $A_{4}$ group is generated byy the $3$-cycle groups, since the first 9 elements are coverd by the $3$-cycles., and the 3 other elements of order 2 are covered by 
			\begin{align*}
				(1 \, 2 \, 3)(2 \, 3 \, 4) &= (1 \, 2)(3 \, 4) \\
				(1 \, 3 \, 4)(3 \, 4 \, 2) &= (1 \, 3)(2 \, 4) \\
				(1 \, 4 \, 3)(4 \, 3 \, 2) &= (1 \, 4)(2 \, 3) 
			\end{align*}
		\end{proof}

	\item Find $\abs{G(\Pi)}$. 
		\begin{proof}
			The size of the orbit is 4. The size of the stabilizer is 6, since we can permute the other three points and that gives us the size of the stabilizer. So $\abs{G(\Pi)} = 24$.
		\end{proof}
	
	\item What standard group is $G(\Pi)$ isomorphic to? Justify your answer.
		\begin{proof}
			 It is isomorphic to the group $S_{4}$, all possible permutations of a square. We see this because any permutation of the points preserves distance.
		\end{proof}
\end{enumerate}
\textbf{Exercise 5}: A necklace has 6 beads, each red or green. How many different necklaces are there?

$\hat{\lambda\lambda\lambda\lambda\lambda\lambda\lambda\lambda\lambda\lambda} 
$

\end{document}
