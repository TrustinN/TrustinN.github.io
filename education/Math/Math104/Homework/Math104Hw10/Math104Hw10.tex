%! TeX root = /Users/trustinnguyen/Downloads/Berkeley/Math/Math104/Homework/Math104Hw10/Math104Hw10.tex

\documentclass{article}
\usepackage{/Users/trustinnguyen/.mystyle/math/packages/mypackages}
\usepackage{/Users/trustinnguyen/.mystyle/math/commands/mycommands}
\usepackage{/Users/trustinnguyen/.mystyle/math/environments/article}
\graphicspath{{./figures/}}

\title{Math104Hw10}
\author{Trustin Nguyen}

\begin{document}

    \maketitle

\reversemarginpar

\textbf{Exercise 1}: Find the exact interval of convergence for the power series $\sum n^{2}x^{n}$.
    \begin{proof}
        We have that 
            \begin{equation*}
                \limsup\limits_{n \to \infty} \lvert n^{2} \rvert^{\frac{1}{n}} = 1 = \beta
            \end{equation*}
        Therefore,  $R = \frac{1}{\beta} = 1$. So it converges when $ \lvert x \rvert < 1$. Now to check for the boundary points, we have for $x = 1$:
            \begin{equation*}
                \sum n^{2}
            \end{equation*}
        and for $x = -1$:
            \begin{equation*}
                \sum (-1)^{n}n^{2}
            \end{equation*}
        Both of these diverge. So the interval of convergence is $(-1, 1)$.
    \end{proof}

\textbf{Exercise 2}: Find the exact interval of convergence for the power series $\sum\frac{n}{2^{n}}x^{n}$.
    \begin{proof}
        We can try the ratio test:
            \begin{equation*}
                \lim\limits_{n \to \infty}\dfrac{\frac{n + 1}{2^{n + 1}}}{\frac{n}{2^{n}}} = \dfrac{n + 1}{2n} = \dfrac{1}{2} = \beta
            \end{equation*}
        Then $R = \frac{1}{\beta} = 2$. So now to test the endpoints:
            \begin{equation*}
                \sum (-1)^{n}n \text{ and } \sum n
            \end{equation*}
        Both of these diverge, so the radius of convergence is $(-2, 2)$.
    \end{proof}

\textbf{Exercise 3}: Let $f_{n} = \frac{1 + \cos{nx}}{n}, x \in \mathbb{R}$. Find $f(x)$ so that $f_{n} \rightarrow f$ pointwise on $\mathbb{R}$, then check whether $f_{n} \rightarrow f$ uniformly or not on $\mathbb{R}$.
    \begin{proof}
        We have that $0 \leq \cos{nx} \leq 1$. So
            \begin{equation*}
                \dfrac{1}{n} \leq\dfrac{1 + \cos{nx}}{n} \leq\dfrac{2}{n}
            \end{equation*}
        Since $\lim\limits_{n \to \infty}\frac{2}{n} = 0$, by comparison test, we have that $f_{n} \rightarrow 0$. 

        Now we need to check uniform convergence or that $\forall \varepsilon > 0$, $\exists N > 0$ such that if $n> N$, we have:
            \begin{equation*}
                \lvert f_{n}(x) - f(x) \rvert = \left\lvert \dfrac{1 + \cos{nx}}{n} \right\rvert< \varepsilon
            \end{equation*}
        We have that
            \begin{gather*}
                \left\lvert \dfrac{1 + \cos{nx}}{n} \right\rvert \leq \left\lvert \dfrac{2}{n} \right\rvert < \varepsilon \\
                \frac{2}{n} < \varepsilon \\
                \frac{2}{\varepsilon} < n
            \end{gather*}
        So we require $N = \frac{2}{\varepsilon}$. Since $N$ does not depend on the value of $x$, it converges uniformly.
    \end{proof}

\textbf{Exercise 4}: Let $f_{n} = \frac{nx}{1 + n^{2} x^{2}}, x \in \mathbb{R}$. Prove that $f_{n} \rightarrow 0$ pointwise on $\mathbb{R}$, then check whether $f_{n} \rightarrow 0$ uniformly or not on $[0, 1]$.
    \begin{proof}
        We find the convergence of $f_{n}$ like so:
            \begin{align*}
                \lim\limits_{n \to  \infty}\dfrac{nx}{1 + n^{2}x^{2}} &= \lim\limits_{n \to \infty}\dfrac{\frac{1}{nx}}{\frac{1}{n^{2}x^{2}} + 1} \\
                                                                      &= \dfrac{\lim \frac{1}{nx}}{\lim \frac{1}{n^{2}x^{2}}\lim 1}               \\
                                                                      &= \dfrac{0}{1}                                                             \\
                                                                      &= 0                                                                          
            \end{align*}
        So $f_{n}\rightarrow 0$. 

        Now to check for uniform convergence, we have to show that $\forall \varepsilon > 0$, $\exists N > 0$ such that $\forall n > N$, we have:
            \begin{equation*}
                \lvert f_{n}(x) - f(x) \rvert = \left\lvert \dfrac{nx}{1 + n^{2}x^{2}} \right\rvert < \varepsilon
            \end{equation*}
        for $x \in [0, 1]$.

        Instead, it is equivalent to show that $\limsup\limits_{n \to \infty} \{\lvert f_{n}(x) \rvert : x \in [0, 1]\} = 0$. First, the derivative:
            \begin{equation*}
                \dv{x}\left(\dfrac{nx}{1 + n^{2}x^{2}}\right) = \dfrac{(1 + n^{2}x^{2})n - nx(2n^{2}x)}{(1 + n^{2}x^{2})^{2}} = \dfrac{n - n^{3}x^{2}}{(1 + n^{2}x^{2})^{2}}
            \end{equation*}
        So we check:
            \begin{equation*}
                n - n^{3}x^{2} = 0 \implies x^{2} = \dfrac{1}{n^{2}} \implies x = \pm \dfrac{1}{n}
            \end{equation*}
        We note that the derivative is positive on $[0, \frac{1}{n})$ and negative after $\frac{1}{n}$. So we obtain the supremum as 
            \begin{equation*}
                \dfrac{n\left(\dfrac{1}{n}\right)}{1 + n^{2}\left(\dfrac{1}{n}\right)^{2}} = \dfrac{1}{2}
            \end{equation*}
        and the infimum as 
            \begin{equation*}
                0
            \end{equation*}
        So the $ \limsup\limits_{n \to \infty} \lvert \frac{nx}{1 + n^{2}x^{2}} \rvert \neq 0$ which means that it does not uniformly converge.
    \end{proof}

\textbf{Exercise 5}: Same $f_{n}$ in Q4, check whether $f_{n} \rightarrow 0$ uniformly or not on $[1, \infty)$.
    \begin{proof}
        Recall from the previous problem that the function $\frac{nx}{1 + n^{2}x^{2}}$ is increasing on the interval $(0, \frac{1}{n})$ and decreasing on the interval $(\frac{1}{n}, \infty)$. Then the sequence:
            \begin{equation*}
                f_{n}(\dfrac{1}{n})
            \end{equation*}
        is increasing but the function $f_{n}(x) > f_{n}(y)$ for $x < y$. Therefore, the function assumes a maximum value in the interval $[1, \infty)$ for $x = 1$. So now we calculate the supremum, which is plugging in $1$:
            \begin{equation*}
                \dfrac{n}{1 + n^{2}}
            \end{equation*}
        and the infimum calculating the limit $x \to \infty$
            \begin{equation*}
                \lim\limits_{x \to \infty}\dfrac{nx}{1 + n^{2}x^{2}} = 0
            \end{equation*}
        Then we have:
            \begin{align*}
                \limsup\limits_{n \to \infty} \{\lvert f_{n}(x) : x \in [1, \infty) \rvert\} &= \limsup\limits_{n \to \infty} \dfrac{n}{1 + n^{2}} \\
                                                                                             &= 0                                                    
            \end{align*}
        Since the limit is $0$, we know that it uniformly converges on the interval.
    \end{proof}

\textbf{Exercise 6}: Use the definition to show: if $f_{n} \rightarrow f$ uniformly on $S$ and $g_{n} \rightarrow g$ uniformly on $S$, then $f_{n} + g_{n} \rightarrow f + g$  uniformly on $S$. 
    \begin{proof}
        If $f_{n} \rightarrow f$ uniformly, on $S$, then we know that $\forall \varepsilon > 0$, $\exists N_{1} > 0$ such that $\forall n> N_{1}$, $x \in S$, we have that:
            \begin{equation*}
                \lvert f_{n}(x) - f(x) \rvert < \varepsilon/2
            \end{equation*}
        Similarly, we know that there is an $N_{2} > 0$ such that $\forall n > N_{2}$, $x \in S$, we have
            \begin{equation*}
                \lvert g_{n}(x) - g(x) \rvert < \varepsilon/2
            \end{equation*}
        Then for all $\varepsilon> 0$, we take the maximum $\max(N_{1}, N_{2})$ that exists for the first two equations. Then for all $x \in S$, we have:
            \begin{align*}
                \lvert f_{n}(x) + g_{n}(x) - f(x) - g(x) \rvert & \leq  \lvert f_{n}(x) - f(x) \rvert + \lvert g_{n}(x) - g(x) \rvert \\
                                                                &<      \varepsilon                                                     
            \end{align*}
        This means that $f_{n}(x) + g_{n}(x)$ uniformly converges to $f(x) + g(x)$ on $S$.
    \end{proof}














\end{document}
