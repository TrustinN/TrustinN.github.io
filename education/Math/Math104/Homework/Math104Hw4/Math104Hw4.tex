%! TeX root = /Users/trustinnguyen/Downloads/Berkeley/Math/Math104/Homework/Math104Hw4/Math104Hw4.tex

\documentclass{article}
\usepackage{/Users/trustinnguyen/.mystyle/math/packages/mypackages}
\usepackage{/Users/trustinnguyen/.mystyle/math/commands/mycommands}
\usepackage{/Users/trustinnguyen/.mystyle/math/environments/article}

\title{Math104Hw4}
\author{Trustin Nguyen}

\begin{document}

    \maketitle

\reversemarginpar

\textbf{Exercise 1}: Let $s_{n} = (-1)^{n} + \frac{1}{n}$ for all $n \in \mathbb{N}$. Prove that $(s_{n})$ is bounded and find a monotone convergent subsequence.
    \begin{proof}
        We will show that the bound of a sum of sequences is the sum of their bounds. Suppose that $s_{n}$ and $(t_{n})$ are bounded. Then:
            \begin{align*}
                l_{1} \leq s_{n} &\leq u_{1} \\
                l_{2} \leq t_{n} &\leq u_{2}
            \end{align*}
        which is true for all $n$. Then we have:
            \begin{equation*}
                l_{1} + l_{2} \leq s_{n} + t_{n} \leq u_{1} + u_{2}
            \end{equation*}
        so therefore, we know that $(-1)^{n}$ is bounded below by $-1$ and above by $1$. We know that $\frac{1}{n}$ is bounded below by $0$ and above by $1$. So therefore, $(-1)^{n} + \frac{1}{n}$ is bounded below by $-1$ and above by $2$.
    \end{proof}

\textbf{Exercise 2}: Let $S$ be the set of all subsequential limits of a real number sequence $(s_{n})$. Prove that $S \cap \mathbb{R}$ is closed in $\mathbb{R}$. (Hint use Thm 11.9 and Prop 13.9)
    \begin{proof}
        We know that if we take a sequence $t_{n}$ on $S \cap \mathbb{R}$, then $\lim(t_{n}) = t \in S \cap \mathbb{R}$ by Thm 11.9. Then by Proposition 13.9, we know that this implies that $S \cap \mathbb{R}$ is bounded.
    \end{proof}

\textbf{Exercise 3}: Prove that $(s_{n})$ is bounded $\iff \mathop{limsup}(\lvert s_{n} \rvert) < \infty$. (Hint: see the proof of Thm 9.1)
    \begin{proof}
        ($\rightarrow $) We know that if $(s_{n})$ is bounded, then:
            \begin{equation*}
                \lvert s_{n} \rvert \leq M
            \end{equation*}
        for some $M = \max(\lvert u \rvert, \lvert l \rvert)$ for $u, l$ upper and lower bounds. So we know that:
            \begin{equation*}
                \mathop{sup}\{s_{n} : n > N\}
            \end{equation*}
        for every $n$ is $\leq M$ because the supremum is the least lower bound. So we can say that for some $L \leq M$, we have for every $\varepsilon > 0$, we can find an $N$ for all $n> N$ such that:
            \begin{equation*}
                \lvert \mathop{sup}\{s_{n} : n> N\} - L \rvert < \varepsilon
            \end{equation*}
        so $\mathop{limsup}(\lvert s_{n} \rvert) = L \leq M < \infty$. 

        ($\leftarrow $) Suppose that $\mathop{limsup}(\lvert s_{n} \rvert) < \infty$. Then we know that for every $\varepsilon > 0$, there is an $N$ such that $\forall n> N$:
            \begin{equation*}
                \lvert \mathop{sup}\{s_{n} : n> N\} - L \rvert < \varepsilon
            \end{equation*}
        for $L < \infty$. Now let $\varepsilon = 1$. So we can find an $N_{0}$ such that for all $n > N_{0}$, 
            \begin{equation*}
                \lvert \mathop{sup}\{s_{n} : n> N\} \rvert <  \varepsilon + L
            \end{equation*}
        and 
            \begin{equation*}
                -\varepsilon - L < \mathop{sup}\{s_{n} : n > N\} < \varepsilon + L
            \end{equation*}
        Now we see that for $\varepsilon = 1$, $\max(L + 1, s_{1}, s_{2}, \ldots , s_{N_{0}})$ is the upper bound and $\min((-1 - L, s_{1}, s_{2}, \ldots , s_{N_{0}}))$ is the lower bound. So we are done.
    \end{proof}

\textbf{Exercise 4}: Prove that the subsequence of a subsequence of a given sequence is a subsequence of the given sequence. Equivalently, let $(s_{n_{k}})$ be a subsequence of $(s_{n})$, and $(s_{k_{l}})$ be a subsequence of $(s_{n_{k}})$, show that $(s_{n_{k_{l}}})$ is a subsequence of $(s_{n})$.
    \begin{proof}
        Suppose that $s_{n_{k}}$ is a subsequence of $s_{n}$. Then we know that
            \begin{equation*}
                n_{1} < n_{2} < n_{3} < \cdots 
            \end{equation*}
        where $n_{k} \in \{1, 2, 3, \ldots \}$. Now suppose that $s_{n_{k_{l}}}$ is a subsequence of $s_{n_{k}}$. Then
            \begin{equation*}
                n_{k_{1}} < n_{k_{2}} < n_{k_{3}} < \cdots 
            \end{equation*}
        where $n_{k_{l}} \in \{n_{1}, n_{2}, n_{3}, \ldots\}$. By the above inequality chain and because $n_{k_{l}} \in \{n_{1}, n_{2}, n_{3}, \ldots , \} \subseteq \{1, 2, 3, \ldots \}$, then we have that $s_{n_{k_{l}}}$ is a subsequence of $s_{n}$.
    \end{proof}

\textbf{Exercise 5}: Define $s_{1} = 1$ and $s_{n + 1} = \frac{s_{n}}{n + 1}$ for any $n \in \mathbb{N}$ (actually $s_{n} = \frac{1}{n!}$). Prove that $\lim(s_{n})^{\frac{1}{n}} = 0$. (hint: cor 12.3)
    \begin{proof}
        By corollary 12.3, if the limit of $\frac{s_{n + 1}}{s_{n}}$ exists, then is equal to the limit of $s_{n}^{\frac{1}{n}}$. In this case, we have that 
            \begin{equation*}
                \dfrac{s_{n + 1}}{s_{n}} = \dfrac{\dfrac{s_{n}}{n + 1}}{s_{n}} = \dfrac{1}{n + 1}
            \end{equation*}
        which we know converges to $0$. Therefore, $\lim(s_{n})^{\frac{1}{n}} = 0$.
    \end{proof}

\textbf{Exercise 6}: Assume that $(s_{n})$ and $(t_{n})$ are bounded sequences. Prove that:
    \begin{equation*}
        \mathop{limsup}(s_{n} + t_{n}) \leq \mathop{limsup}(s_{n}) + \mathop{limsup}(t_{n})
    \end{equation*}
        \begin{proof}
            We see that:
                \begin{equation*}
                    \mathop{sup}\{s_{n} + t_{n}: n > N\} \leq \mathop{sup}\{s_{n} : n > N\} + \mathop{sup}\{t_{n} : n > N\}
                \end{equation*}
            because we have that 
                \begin{equation*}
                    \{s_{n} + t_{n}: n > N\} \subseteq \{s_{n} : n > N\} + \{t_{n} : n > N\} \ni \mathop{sup}\{s_{n} : n > N\} + \mathop{sup}\{t_{n} : n > N\}
                \end{equation*}
            If we interpret $s_{n}^{\prime} = \mathop{sup}\{s_{n} + t_{n}: n > N\}$ as a sequence and  $t^{\prime}_{n} = \mathop{sup}\{s_{n} : n > N\} + \mathop{sup}\{t_{n} : n > N\}$ as a sequence, because the limit exists and $s_{n} \leq t_{n}$ for all $n > N$, by the previous homework, we have that:
                \begin{equation*}
                    \lim(s^{\prime}_{n})\leq \lim(t^{\prime}_{n})
                \end{equation*}
            as desired.
        \end{proof}





\end{document} 
