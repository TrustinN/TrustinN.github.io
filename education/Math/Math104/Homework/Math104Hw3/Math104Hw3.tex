%! TeX root = /Users/trustinnguyen/Downloads/Berkeley/Math/Math104/Homework/Math104Hw3/Math104Hw3.tex

\documentclass{article}
\usepackage{/Users/trustinnguyen/.mystyle/math/packages/mypackages}
\usepackage{/Users/trustinnguyen/.mystyle/math/commands/mycommands}
\usepackage{/Users/trustinnguyen/.mystyle/math/environments/article}

\title{Math104Hw3}
\author{Trustin Nguyen}

\begin{document}

    \maketitle

\reversemarginpar

\textbf{Exercise 1}: Use the limit Theorem $9.2 - 9.7$ to prove that $\lim(\frac{n^{2}}{n^{2} + 1}) = 1$. Justify all steps.
    \begin{proof}
        By the theorems:
            \begin{equation*}
                \lim\limits_{n \to \infty} \dfrac{n^{2}}{n^{2} + 1} = \lim\limits_{n \to \infty} \dfrac{1}{1 + \dfrac{1}{n^{2}}}
            \end{equation*}
        we have $\lim(\frac{1}{n^{2}}) = 0$. The limit of the sum of the series $t_{n} = 1$ and $s_{n} = \frac{1}{n^{2}}$ is just the sum of the limits:
            \begin{equation*}
                \lim\limits_{n \to \infty} \dfrac{1}{1}
            \end{equation*}
        Now since the numerator as a sequence and denominator as a sequence don't converge to $0$, the limit of the division is the division of their limits which is just $1$.
    \end{proof}


\textbf{Exercise 2}: Assume that $\lim(s_{n})$ and $\lim(t_{n})$ exist, and $t_{n} \geq s_{n}$ for all $n$. Prove that $\lim(t_{n}) \geq \lim(s_{n})$.
    \begin{proof}
        We have that $t_{n} - s_{n} \geq 0$ for all $n$. Therefore, our sequence $(t_{n} - s_{n})$ is bounded by $0$ on the lower end. The limit of this sequence is therefore greater than or equal to $0$. But the limit of the sequence $t_{n} - s_{n}$ is the sum of the limits. So we have $\lim(t_{n} - s_{n}) = \lim(t_{n}) - \lim(s_{n}) \geq 0$. So we get
            \begin{equation*}
                \lim(t_{n}) \geq \lim(s_{n})
            \end{equation*}
        as desired.
    \end{proof}

\textbf{Exercise 3}: Find the limit of the following sequences if they exist, otherwise write DNE. No proof is required.
    \begin{itemize}
        \item $n^{n}$;
            \begin{answer}
                The limit of the sequence is $\infty$. We can show that it diverges to $\infty$ by showing that $\forall M > 0$, $\exists N$ such that $\forall n> N$, we have
                    \begin{equation*}
                        n^{n} > M
                    \end{equation*}
                Now pick $N = \max(M, 1)$. Then we have $n \geq M$ and $n \geq 1$. Therefore,
                    \begin{equation*}
                        n^{n} \geq n
                    \end{equation*}
                but $n > N \geq M$ so $n > M$. Therefore, we have found an $N$ such that for all $n > N$, 
                    \begin{equation*}
                        n^{n} > M
                    \end{equation*}
            \end{answer}

        \item $(-n)^{n}$;
            \begin{answer}
                The limit does not exist. If we take the subsequence of this sequence where $n$ is even and the subsequence where $n$ is odd, we see that there are two limits. But if $S$ is the set of subsequential limits, then there is a limit iff $\lvert S \rvert = 1$. In this case, we have the limits $\infty, -\infty$.
            \end{answer}

        \item $(1.1)^{n}$.
            \begin{answer}
                The limit of the sequence is $\infty$. We can rewrite this as
                    \begin{equation*}
                        \dfrac{11^{n}}{10^{n}}
                    \end{equation*}
                Divide the numerator and denominator by $11^{n}$:
                    \begin{equation*}
                        \dfrac{1}{\dfrac{10^{n}}{11^{n}}}
                    \end{equation*}
                The denominator converges to $0$, since $\frac{10}{11} <  1$. Therefore, we have that the number goes to $\infty$.
            \end{answer}
    \end{itemize}

\textbf{Exercise 4}: Let $s_{n} = \cos{\frac{n\pi}{3}}$. Use the definition to find $\mathop{limsup}(s_{n})$ and $\mathop{liminf}(s_{n})$, then explain why $s_{n}$ has no limit. 
    \begin{proof}
        We have by definition:
            \begin{align*}
                \mathop{liminf}(s_{n}) &= \lim\limits_{N \to \infty} \mathop{inf}\{s_{n} : n > N\} \\
                \mathop{limsup}(s_{n}) &= \lim\limits_{N \to \infty} \mathop{sup}\{s_{n} : n > N\}
            \end{align*}
        We know that $-1 \leq \cos{x} \leq 1$. We will show that $-1, 1 \in \{s_{n} : n> N\}$ for any $N$. Clearly, $3N > N$. Now let $n = 3N$. Then we have $\cos{N\pi} \in \{s_{n} : n> N\}$. If $N$ is even, then we therefore have $1 \in \{s_{n} : n > N\}$. The we can also let $n = 3(N + 1)$, which gives us $\cos{(N + 1)\pi} = -1 \in \{s_{n} : n > N\}$ also. For the case when $N$ is odd, we also have that $1, -1 \in \{s_{n} : n > N\}$.

        Now we prove that $1$ is the supremum and $-1$ is the infimum of $\{s_{n} : n > N\}$ for any $N$. This is immediate because any upper bound less than $1$ is not an upper bound because $1 \in \{s_{n} : n > N\}$. Same for the infimum argument.

        So now we just have:
            \begin{align*}
                \mathop{liminf}(s_{n}) &= \lim\limits_{N \to \infty} -1 = -1 \\
                \mathop{limsup}(s_{n}) &= \lim\limits_{N \to \infty} 1  = 1   
            \end{align*}
        so since $\mathop{liminf}(s_{n}) \neq \mathop{limsup}(s_{n})$, the limit does not exist.
    \end{proof}

\textbf{Exercise 5}: Define $s_{1} = 1$ and $s_{n + 1} = \frac{s_{n} + 1}{4}$ for all $n \in \mathbb{N}$. Prove that:
    \begin{itemize}
        \item for all $n$ we have $1 \geq s_{n} \geq 1/3$
            \begin{proof}
                We will show this by induction:
                    \begin{itemize}
                        \item Base Case: For $s_{1}$, we have $1 \geq s_{1} \geq 1/3$.

                        \item Inductive Case: Suppose that $1 \geq s_{n} \geq 1/3$ Now we have 
                            \begin{equation*}
                                \dfrac{4}{3} \leq s_{n} + 1 \leq 2
                            \end{equation*}
                        and therefore, 
                            \begin{equation*}
                                \dfrac{1}{3} \leq \dfrac{s_{n} + 1}{4} = s_{n + 1} \leq 1/2 \leq 1
                            \end{equation*}
                        so we have as desired
                    \end{itemize}
            \end{proof}

        \item $(s_{n})$ is decreasing;
            \begin{proof}
                We will check this by taking the difference:
                    \begin{equation*}
                        s_{n} - s_{n + 1} = s_{n} - \dfrac{s_{n} + 1}{4} = \dfrac{3s_{n} + 1}{4}
                    \end{equation*}
                But since $s_{n} > 0$, we have
                    \begin{equation*}
                        s_{n} - s_{n + 1} > 0
                    \end{equation*}
                Therefore, $(s_{n})$ is decreasing.
            \end{proof}

        \item $\lim(s_{n})$ exists and find $\lim(s_{n})$. 
            \begin{proof}
                (Part I) The limit exists because lower bounded decreasing sequences are convergent. We can pick the infimum of the set 
                    \begin{equation*}
                        \{s : s \in (s_{n})\}
                    \end{equation*}
                and say that $\mathop{inf}(s_{n}) + \varepsilon$ is not a lower bound, meaning we can find an $S_{N}$ such that
                    \begin{equation*}
                        S_{N} < \mathop{inf}(s_{n}) + \varepsilon
                    \end{equation*}
                but since $(s_{n})$ is decreasing, we have that for all $n > N$,
                    \begin{equation*}
                        -\varepsilon < s_{n} - \mathop{inf}(s_{n}) < \varepsilon
                    \end{equation*}
                so it has a limit.

                Now we know that there is a limit, so any subsequence converges to this same limit. So we have $\lim(s_{n + 1}) = \lim(s_{n})$. We can use the limit theorems to get the following simplifications:
                    \begin{align*}
                        \lim(s_{n + 1})         &= \lim(\dfrac{s_{n} + 1}{4})                  \\
                        \lim(s_{n})             &= \lim(\dfrac{s_{n}}{4}) + \lim(\dfrac{1}{4}) \\
                        \lim(\dfrac{3s_{n}}{4}) &= \dfrac{1}{4}                                \\
                        \dfrac{3}{4}\lim(s_{n}) &= \dfrac{1}{4}                                \\
                        \lim(s_{n})             &= \dfrac{1}{3}                                  
                    \end{align*}
                which is our limit.
            \end{proof}
    \end{itemize}

\textbf{Exercise 6}: Directly use the definition of the Cauchy sequence to show that:
    \begin{itemize}
        \item $a_{n} = 1/n$ is a Cauchy sequence;
            \begin{proof}
                We will show that $\forall \varepsilon > 0$, there is an $N$ such that $\forall n, m > N$ we have
                    \begin{equation*}
                        \lvert a_{n} - a_{m} \rvert < \varepsilon
                    \end{equation*}
                By triangle inequality:
                    \begin{equation*}
                        \lvert a_{n} - a_{m} \rvert \leq \lvert \lvert a_{n} \rvert + \lvert a_{m} \rvert \rvert
                    \end{equation*}
                it suffices to show that we can find an $N$ such that $\forall n, m > N$:
                    \begin{equation*}
                        a_{n} + a_{m} = \dfrac{1}{n} + \dfrac{1}{m} < \varepsilon
                    \end{equation*}
                So we want:
                    \begin{equation*}
                        \dfrac{1}{n} < \dfrac{\varepsilon}{2}
                    \end{equation*}
                or in other words,
                    \begin{equation*}
                        n > \dfrac{2}{\varepsilon} \implies N = \dfrac{2}{\varepsilon}
                    \end{equation*}
                So we can check:
                    \begin{align*}
                        n                           &>  \dfrac{2}{\varepsilon} & m            &>  \dfrac{2}{\varepsilon} \\
                        \dfrac{1}{n}                &<  \dfrac{\varepsilon}{2} & \dfrac{1}{m} &<  \dfrac{\varepsilon}{2} \\
                        \dfrac{1}{n} + \dfrac{1}{m} &<  \varepsilon            &              &                            
                    \end{align*}
                so we have as desired.
            \end{proof}

        \item $b_{n} = (-1)^{n}$ is not a Cauchy sequence.
            \begin{proof}
                So we want to show that there is an $\varepsilon > 0$ such that $\forall N$, we have there are are $n, m > N$ such that 
                    \begin{equation*}
                        \lvert b_{n} - b_{m} \rvert \geq \varepsilon
                    \end{equation*}
                Take $\varepsilon = 1$ and let $N$ be arbitrary. Clearly, $2N > N$ and we have:
                    \begin{equation*}
                        b_{2N} = (-1)^{2N} = 1
                    \end{equation*}
                and we also have 
                    \begin{equation*}
                        b_{2N + 1} = (-1)^{2N + 1} = -1
                    \end{equation*}
                Now take $n, m = 2N, 2N + 1$. Then
                    \begin{equation*}
                        \lvert b_{n} - b_{m} \rvert = \lvert 1 - (-1) \rvert = 2 \geq 1 = \varepsilon
                    \end{equation*}
                so this is not a Cauchy sequence.
            \end{proof}
    \end{itemize}


















\end{document}
