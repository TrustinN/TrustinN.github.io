%! TeX root = /Users/trustinnguyen/Downloads/Berkeley/Math/Math104/Homework/Math104Hw5/Math104Hw5.tex

\documentclass{article}
\usepackage{/Users/trustinnguyen/.mystyle/math/packages/mypackages}
\usepackage{/Users/trustinnguyen/.mystyle/math/commands/mycommands}
\usepackage{/Users/trustinnguyen/.mystyle/math/environments/article}

\title{Math104Hw5}
\author{Trustin Nguyen}

\begin{document}

    \maketitle

\reversemarginpar

\textbf{Exercise 1}: For any $x, y \in \mathbb{R}^{2}$, define $d_{1}(x, y) = \max(\{\lvert x_{1} - y_{1} \rvert, \lvert x_{2} - y_{2} \rvert\})$. Prove that $d_{1}$ is a metrics on $\mathbb{R}^{2}$.
    \begin{proof}
        We check three properties:
            \begin{itemize}
                \item We require that $d_{1}(x, x) = 0$. So we have $x = (x_{1}, y_{1})$ and
                    \begin{equation*}
                        d(x, x) = \max(\lvert x_{1} - x_{1} \rvert, \lvert y_{1} - y_{1} \rvert) = \max(0, 0) = 0
                    \end{equation*}
                    We need to show that if $a = (x_{1}, y_{1}) \neq (x_{2}, y_{2}) = b$, then $d(a, b) \neq 0$. So we have:
                        \begin{equation*}
                            d(a, b) = \max(\lvert x_{1} - x_{2} \rvert, \lvert y_{1} - y_{2} \rvert)
                        \end{equation*}
                    But we know that either $x_{1} \neq x_{2}$ or $y_{1} \neq y_{2}$. Wlog, if $x_{1} \neq x_{2}$, then $x_{1} - x_{2} \neq 0$ and $\lvert x_{1} - x_{2} \rvert \geq 0$. So $\lvert x_{1} - x_{2} \rvert > 0$. So the maximum must be greater than $0$ and therefore, $d(a, b) \neq 0$.

                \item We require that $d(a, b) = d(b, a)$. Indeed:
                    \begin{equation*}
                        d(a, b) = \max(\lvert x_{1} - x_{2} \rvert, \lvert y_{1} - y_{2} \rvert) = \max(\lvert x_{2} - x_{1} \rvert, \lvert y_{2} - y_{1} \rvert) = d(b, a)
                    \end{equation*}
                so second condition is verified.

                \item Lastly, we must prove that $d(a, c) \leq d(a, b) + d(b, c)$. So let $c = (x_{3}, y_{3})$ in this case. We have:
                    \begin{equation*}
                        d(a, b) + d(b, c) = \max(\lvert x_{1} - x_{2} \rvert, \lvert y_{1} - y_{2} \rvert) + \max(\lvert x_{2} - x_{3} \rvert, \lvert y_{2} - y_{3} \rvert)
                    \end{equation*} 
                But notice that 
                    \begin{equation*}
                        \max(\lvert a \rvert + \lvert c \rvert, \lvert b \rvert + \lvert d \rvert) \leq \max(\lvert a \rvert, \lvert b \rvert) + \max(\lvert c \rvert, \lvert d \rvert)
                    \end{equation*}
                which will be proved at the end if you're interested. That aside, we get as a result:
                    \begin{equation*}
                        \max(\lvert x_{1} - x_{2} \rvert + \lvert x_{2} - x_{3} \rvert, \lvert y_{1} - y_{2} \rvert + \lvert y_{2} - y_{3} \rvert) \leq \max(\lvert x_{1} - x_{2} \rvert, \lvert y_{1} - y_{2} \rvert) + \max(\lvert x_{2} - x_{3} \rvert, \lvert y_{2} - y_{3} \rvert)
                    \end{equation*}
                and by the triangle inequality:
                    \begin{equation*}
                        \max(\lvert x_{1} - x_{3} \rvert, \lvert y_{2} - y_{3} \rvert) \leq \max(\lvert x_{1} - x_{2} \rvert + \lvert x_{2} - x_{3} \rvert, \lvert y_{1} - y_{2} \rvert + \lvert y_{2} - y_{3} \rvert)
                    \end{equation*}
                so therefore, we have a string of inequalities giving us:
                    \begin{equation*}
                        \max(\lvert x_{1} - x_{3} \rvert, \lvert y_{2} - y_{3} \rvert) \leq \max(\lvert x_{1} - x_{2} \rvert, \lvert y_{1} - y_{2} \rvert) + \max(\lvert x_{2} - x_{3} \rvert, \lvert y_{2} - y_{3} \rvert)
                    \end{equation*}
                which shows that the distance formula obeys triangle inequality.

                (Proof of Claim) We want to show:
                    \begin{equation*}
                        \max(\lvert a \rvert + \lvert c \rvert, \lvert b \rvert + \lvert d \rvert) \leq \max(\lvert a \rvert, \lvert b \rvert) + \max(\lvert c \rvert, \lvert d \rvert)
                    \end{equation*}
                We can see this by cases. If $\lvert a \rvert \geq \lvert b \rvert$ and $\lvert c \rvert \geq \lvert d \rvert$, then we get $\lvert a \rvert + \lvert c \rvert$ on the RHS and the same on the LHS. If we get $\lvert c \rvert \leq \lvert d \rvert$, then we get $\lvert a \rvert + \lvert d \rvert$ on the RHS and either $\lvert a \rvert + \lvert c \rvert$ or $\lvert b \rvert + \lvert d \rvert$ on the LHS. Either way, both are less than or equal to the RHS.
            \end{itemize}
    \end{proof}

\textbf{Exercise 2}: For any $x, y \in \mathbb{R}^{2}$, define $d_{2}(x, y) = \lvert x_{1} - y_{1} \rvert + \lvert x_{2} - y_{2} \rvert$. Prove that $d_{2}$ is a metrics on $\mathbb{R}^{2}$.
    \begin{proof}
        We need to prove the three conditions:
            \begin{itemize}
                \item If we have $x = (x_{1}, y_{1})$, then 
                    \begin{equation*}
                        d(x, x) = \lvert x_{1} - x_{1} \rvert + \lvert y_{1} - y_{1} \rvert = 0
                    \end{equation*}
                as desired. If we have $y = (x_{2}, y_{2}) \neq x$, then we have:
                    \begin{equation*}
                        d(x, y) = \lvert x_{1} - x_{2} \rvert + \lvert y_{1} - y_{2} \rvert 
                    \end{equation*}
                and since either $x_{1} - x_{2}$ or $y_{1} - y_{2}$ is non-zero, we  know that one of $\lvert x_{1} - x_{2} \rvert, \lvert y_{1} - y_{2} \rvert$ is non-zero also. Since both absolute values are $ \geq 0$, we know that:
                    \begin{equation*}
                        d(x, y) > 0
                    \end{equation*}
                which proves the first part.

                \item Now we need to prove symmetry:
                    \begin{equation*}
                        d(x, y) = \lvert x_{1} - x_{2} \rvert + \lvert y_{1} - y_{2} \rvert = \lvert x_{2} - x_{1} \rvert + \lvert y_{2} - y_{1} \rvert = d(y, x)
                    \end{equation*}
                so that is the second condition done.

                \item Now we need to show that:
                    \begin{equation*}
                        d(x, z) \leq d(x, y) + d(y, z)
                    \end{equation*}
                We have:
                    \begin{align*}
                        d(x, y) + d(y, z) &= \lvert x_{1} - x_{2} \rvert + \lvert y_{1} - y_{2} \rvert + \lvert x_{2} - x_{3} \rvert + \lvert y_{2} - y_{3} \rvert \\
                                          &= \lvert x_{1} - x_{2} \rvert + \lvert x_{2} - x_{3} \rvert + \lvert y_{1} - y_{2} \rvert + \lvert y_{2} - y_{3} \rvert \\
                                          &\geq \lvert x_{1} - x_{3} \rvert + \lvert y_{1} - y_{3} \rvert \text{ by triangle inequality} \\
                                          &= d(x, z)
                    \end{align*}
                so we have proven that this distance formula obeys triangle inequality.
            \end{itemize}
    \end{proof}

\textbf{Exercise 3}: In $\mathbb{R}$, find the interior, closure and boundary of the set $\{1/n^{2} : n \in \mathbb{N}\}$.
    \begin{proof}
        (Interior) There are no points in the interior. We note that if $p \in \{1/n^{2} : n \in \mathbb{N}\}$, then we cannot find an $\varepsilon$ such that:
            \begin{equation*}
                E = \{r \in \mathbb{R} : \lvert p - r \rvert <  \varepsilon\} \subseteq \{1/n^{2} : n \in \mathbb{N}\} = M
            \end{equation*}
        To prove this, we see that $s_{n} = \frac{1}{n^{2}}$ is a decreasing sequence. We must have $\lvert E \rvert > 1$. So we can pick one other point in $E$ that is also in $M$. Call this $1/n^{2}$. If $p = 1/m^{2} >  1/n^{2}$, then we can build a chain:
            \begin{equation*}
                1/m^{2} > 1/(m + 1)^{2} > \cdots \geq  1/n^{2} 
            \end{equation*}
        But because the sequence is decreasing, we know that 
            \begin{equation*}
                \dfrac{1}{2}\left(\dfrac{1}{m^{2}} + \dfrac{1}{(m + 1)^{2}}\right)
            \end{equation*}
        is in $E$ but not $M$. If $p < 1/n^{2}$, then we build the chain backwards and say that 
            \begin{equation*}
                \dfrac{1}{2}\left(\dfrac{1}{n^{2}} + \dfrac{1}{(n - 1)^{2}}\right)
            \end{equation*}
        is in $E$ but not $M$. So $E$ is never a subset of $M$. So the interior is $\emptyset$.

        (Closure) We note that the complement  of $E \cup \{0\}$ is an open set. If our point $p$ was greater than all points in the set, then we just need to choose a distance smaller than $\min(\lvert p - 1 \rvert, \lvert p \rvert)$, which accounts for the distances from $p$ to $0$ and $1$. On the other hand, any other point in $M^{c}$ satisfies 
            \begin{equation*}
                \dfrac{1}{(n + 1)^{2}} < p < \dfrac{1}{n^{2}}
            \end{equation*}
        for some $i$ and we can just take our radius to be $\varepsilon < \min(p - \frac{1}{(n + 1)^{2}}, \frac{1}{n^{2}} - p)$. Because the sequence $1/n^{2}$ is decreasing, we know that none of the terms will be within this radius. So the closure is the set $\{1/n^{2} : n \in \mathbb{N}\} \cup \{0\}$ because the complement is open.

        (Boundary) If we look at the closure of $M^{c}$, we know that the rationals are dense in $\mathbb{R}$, so we can always find a rational sequence converging to every element of $M$. So this means that $\{1/n^{2} : n \in \mathbb{N}\}$ must be a subset of the closure. Therefore the closure of $M^{c}$ is just $\mathbb{R}$. The boundary is $M^{c} \cap (\{0\} \cup \{1/n^{2} : n \in \mathbb{N}\})$ which is 
            \begin{equation*}
                \{1/n^{2} : n \in \mathbb{N}\} \cup \{0\}
            \end{equation*}
    \end{proof}

\textbf{Exercise 4}: In $\mathbb{R}$, find the interior, closure and boundary of the rational number set $\mathbb{Q}$.
    \begin{proof}
        (Interior) A proof of the fact that there is an irrational between any two rationals will be left after this proof. Anyways, the claim is that the interior is $\emptyset$. This is because for any point $p \in \mathbb{Q}$, we have $(p - \varepsilon, p + \varepsilon)$ must contain $p$ and some other point. If it is irrational for all $\varepsilon > 0$, we are done. If there is a rational $q$, then using the fact that there is an irrational between $p, q$, we see that:
            \begin{equation*}
                \{r \in \mathbb{R} : \lvert p - r \rvert < \varepsilon\} \not\subseteq \mathbb{Q}
            \end{equation*}
        since irrationals are not in $\mathbb{Q}$. So the interior is $\emptyset$.

        (Closure) Since a closed set must contain the limit of all sequences on $\mathbb{Q}$, using the fact that the rationals are dense in $\mathbb{R}$, we can construct a sequence converging to an irrational, $q$. Pick an arbitrary rational say $p_{1}$. By the denseness of rationals, we can find a $p_{2}$ such that:
            \begin{equation*}
                q < p_{2} < p_{1}
            \end{equation*}
        if $p_{1} > q$ or 
            \begin{equation*}
                p_{1} < p_{2} < q
            \end{equation*}
        We continue finding $p_{i}$ and this creates a sequence converging to $q$. So the closure is the rationals union the irrationals which is all of $\mathbb{R}$.

        The boundary is the closure of the complement intersect the closure. So the closure of the irrationals must contain all the limits on the irrationals. We will show that there exists a sequence converging to a rational $p$. Let $q_{1}$ be an irrational $q_{1} < p$ wlog. Then we can find a rational $p_{1}$ between them:
            \begin{equation*}
                q_{1} < p_{1} < p
            \end{equation*}
        But by what will be proved later, there is an irrational between $p_{1}, p$. So we can keep going:
            \begin{equation*}
                q_{1} < p_{1} < q_{2} < p_{2} < q_{3} < p_{3} < \cdots < p
            \end{equation*}
        Now just take the irrational numbers in this and that forms a sequence converging to $p$. Therefore, the closure of $\mathbb{R} - \mathbb{Q}$ is $\mathbb{R} - \mathbb{Q} \cup \mathbb{Q} = \mathbb{R}$. So the boundary is $\mathbb{R} \cap \mathbb{R} = \mathbb{R}$

        (Proof of Claim) We want to show that between any two rationals, there is an irrational $q$:
            \begin{equation*}
                \dfrac{a}{b} < q < \dfrac{c}{d}
            \end{equation*}
        We will look at the case when $\frac{a}{b} > 0$ first. We know that $1 < \sqrt{2} < \frac{3}{2}$. So we have:
            \begin{equation*}
                \dfrac{a}{b} < \dfrac{a\sqrt{2}}{b}
            \end{equation*}
        If $\frac{a\sqrt{2}}{b} < \frac{c}{b}$, then we are done. Otherwise, take the ratio:
            \begin{align*}
                \dfrac{\dfrac{a\sqrt{2}}{b}}{\dfrac{c}{d}} &= r_{1} \\
                \dfrac{\dfrac{a\sqrt{2}}{b}}{\dfrac{a}{b}} &= r_{2}
            \end{align*}
        and find a rational number $f$ such that:
            \begin{equation*}
                r_{1} < f < r_{2}
            \end{equation*}
        So we have:
            \begin{align*}
                \dfrac{c}{d}r_{1}                         &=  \dfrac{a\sqrt{2}}{b}  \\
                \dfrac{c}{d}\left(\dfrac{r_{1}}{f}\right) &=  \dfrac{a\sqrt{2}}{bf} \\
                \dfrac{c}{d}                              &>  \dfrac{a\sqrt{2}}{bf}
            \end{align*}
        and
            \begin{align*}
                \dfrac{a}{b}r_{2}                         &=  \dfrac{a\sqrt{2}}{b}  \\
                \dfrac{a}{b}\left(\dfrac{r_{2}}{f}\right) &=  \dfrac{a\sqrt{2}}{bf} \\
                \dfrac{a}{b}                              &<  \dfrac{a\sqrt{2}}{bf}   
            \end{align*}
        Then 
            \begin{equation*}
                \dfrac{a}{b} < \dfrac{a\sqrt{2}}{fb} < \dfrac{c}{d}
            \end{equation*}
        as desired.

        Now if $\frac{a}{b} \leq  0 < \frac{c}{d}$, then we just apply the same idea with some rational $p$ between $0$ and $\frac{c}{d}$. If $\frac{a}{b} < \frac{c}{d} < 0$, we multiply through by a negative to get $0 < \frac{-c}{d} < \frac{-a}{b}$ and find our irrational $q$ such that $0 < \frac{-c}{d} < q < \frac{-a}{b}$, and multiply through by negative sign to get:
            \begin{equation*}
                \dfrac{a}{b} < -q < \dfrac{c}{d} < 0
            \end{equation*}
        so we are done as we have shown for all cases.
    \end{proof}

\textbf{Exercise 5}: Let $E$ be a compact set of $\mathbb{R}$, prove that $\max(E)$ and $\min(E)$ exist.
    \begin{proof}
        We know that a set $E \subseteq \mathbb{R}$ is compact if and only if it is bounded and closed. Since it is bounded, by the completeness axiom, we know that it exists.
    \end{proof}

\textbf{Exercise 6}: Let $S = \mathbb{R} - \{0\} = (-\infty, 0) \cup (0, \infty)$ with the usual metric. Show that $(-\infty, 0)$ is a closed subset of $S$.
    \begin{proof}
        We just need to show that its complement is open. The complement is $(0, \infty)$. Suppose that $p \in (0, \infty)$. Then we just choose $r < p$ as:
            \begin{equation*}
                \{s \in S : d(s, p) < r\}
            \end{equation*}
        we know that:
            \begin{equation*}
                d(s, p) = \lvert s - p \rvert < p
            \end{equation*}
        which means:
            \begin{equation*}
                -p < s - p < p
            \end{equation*}
        or 
            \begin{equation*}
                0 < s < 2p
            \end{equation*}
        so therefore, $s \in (0, \infty)$ which proves:
            \begin{equation*}
                \{s \in S : d(s, p) < r\} \subseteq (0, \infty)
            \end{equation*}
        Since $(0, \infty)$ is equal to its interior, we know that it is open. Therefore, $(-\infty, 0)$ is closed.
    \end{proof}




\end{document} 
