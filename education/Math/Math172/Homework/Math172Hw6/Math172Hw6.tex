%! TeX root = /Users/trustinnguyen/Downloads/Berkeley/Math/Math172/Homework/Math172Hw6/Math172Hw6.tex

\documentclass{article}
\usepackage{/Users/trustinnguyen/.mystyle/math/packages/mypackages}
\usepackage{/Users/trustinnguyen/.mystyle/math/commands/mycommands}
\usepackage{/Users/trustinnguyen/.mystyle/math/environments/article}

\title{Math172Hw6}
\author{Trustin Nguyen}

\begin{document}

    \maketitle

\reversemarginpar

\textbf{Exercise 1}: Find an explicit formula for $a_{n}$ if $a_{0} = 1$ and for any $n \geq 1$ we have $a_{n} = 3a_{n - 1} + 2^{n - 1}$.
    \begin{proof}
        We start with $a_{0} = 1$ and list out the numbers in the process:
            \begin{itemize}
                \item $a_{0}  = 1$

                \item $a_{1} = 3 + 1$

                \item $a_{2} = 3^{2} + 1 \cdot 3 + 2^{1}$

                \item $a_{3} = 3^{3} + 1 \cdot 3^{2} + 2^{1} \cdot 3 + 2^{2}$

                \item $\vdots $ 
            \end{itemize}
        It is visible that 
            \begin{equation*}
                2a_{n} + 3^{n + 1} - 3^{n} = 3a_{n}+ 2^{n}
            \end{equation*}
        This is because which we will prove inductively below. 
            \begin{equation*}
                a_{n} = 3^{n} + \sum_{k = 0}^{n - 1}3^{k} \cdot 2^{(n - 1) - k}
            \end{equation*}
        which evaluates to the same thing for both sides:
            \begin{equation*}
                3^{n} + 2\sum_{k = 0}^{n - 1}3^{k} \cdot 2^{(n - 1)  - k} + 3^{n + 1} = 3^{n + 1} + \sum_{k = 0}^{n}3^{k} \cdot 2^{n - k}
            \end{equation*}
        for the RHS and 
            \begin{align*}
                3^{n + 1} + \sum_{k = 0}^{n - 1}3^{k + 1}2^{(n - 1)- k} + 2^{n} &= 3^{n + 1} + 3^{n} + \sum_{k = 0}^{n - 2}3^{k + 1}2^{(n - 1)- k} + 2^{n} \\
                                                                                &= 3^{n + 1} + 3^{n} + \sum_{k = 0}^{n - 1} 3^{k}2^{n - k} \\
                                                                                &= 3^{n + 1} + \sum_{k =0}^{n - 1}3^{k}2^{n - k}
            \end{align*}
        Notice that the evaluation of the RHS shows that the form of $a_{n}$ holds inductively. We have shown the inductive case for this equivalence, which also holds for $n = 0$ because $a_{0} = 1$, which indeed fits the description of $a_{n}$ we have given above. Now we just solve for $a_{n}$:
            \begin{equation*}
                2a_{n} + 3^{n + 1} - 3^{n} = 3a_{n} + 2^{n}
            \end{equation*}
        which shows that:
            \begin{equation*}
                a_{n} = 3^{n + 1} - 3^{n} - 2^{n}
            \end{equation*}
    \end{proof}

\textbf{Exercise 2}: Find an explicit formula for $a_{n}$ if $a_{0} = 1$, $a_{1} = 4$ and for any $n \geq 2$ we have $a_{n} = 8a_{n - 1} - 16a_{n - 2}$.
    \begin{proof}
        We have 
            \begin{align*}
                \sum_{n \geq 2}a_{n}x^{n} &= \sum_{n \geq 2}8a_{n - 1}x^{n} + \sum_{n \geq 2} -16a_{n - 2}x^{n - 2} \\
                \sum_{n \geq 0}a_{n}x^{n} - 4x - 1 &= 8x\sum_{n \geq 2}a_{n - 1}x^{n - 1} - 16x^{2}\sum_{n \geq 2}a_{n - 2}x^{n - 2} \\
                \sum_{n \geq 0}a_{n}x^{n} - 4x - 1 &= 8x \sum_{n \geq 0}a_{n}x^{n} - 8x - 16x^{2}\sum_{n \geq 0}a_{n}x^{n}
            \end{align*}
        if we let $F(x) = \sum_{n \geq 0}a_{n}x^{n}$, then we have:
            \begin{align*}
                F(x) - 4x - 1 &= 8xF(x) - 8x - 16x^{2}F(x) \\
                F(x) - 8xF(x) + 16x^{2}F(x) &= -4x + 1 \\
                F(x)(1 - 8x + 16x^{2}) &= -4x + 1 \\
                F(x) &= \dfrac{-4x + 1}{1 - 8x + 16x^{2}}
            \end{align*}
        Now we calculate the roots of the denominator:
            \begin{equation*}
                Q(x) = (4x - 1)(4x - 1)
            \end{equation*}
        or we just simplify:
            \begin{equation*}
                F(x) = \dfrac{-1}{4x - 1} = \dfrac{1}{1 - 4x}
            \end{equation*}
        But we know that:
            \begin{equation*}
                \dfrac{1}{1 - y} = 1 + y + y^{2} + \cdots 
            \end{equation*}
        So 
            \begin{equation*}
                \dfrac{1}{1 - 4x} = 1 + 4x + (4x)^{2} + \cdots  
            \end{equation*}
        Now $a_{n}$ is the coefficient of $x^{n}$ in $F(x)$. Therefore, $a_{n} = 4^{n}$.
    \end{proof}

\textbf{Exercise 3}: Let $F(x) = \sum_{k \geq 0}a_{k}x^{k}$ be a formal power series. The \textit{formal derivative} of $F(x)$ is defined as the formal power series $\sum_{k \geq 0} ka_{k}x^{k - 1}$ and is denoted by $F^{\prime}(x)$ of $\dv{x}F(x)$.
    \begin{itemize}
        \item Show that for any formal power series $F(x), G(x)$ we have $\dv{x}(F(x) + G(x)) = F^{\prime}(x) + G^{\prime}(x)$ and $\dv{x}(F(x)G(x)) = F^{\prime}(x)G(x) + F(x)G^{\prime}(x)$.
            \begin{proof}
                We have $F(x) = \sum_{k \geq 0}a_{k}x^{k}$ and $G(x) = \sum_{j \geq 0}b_{j}x^{j}$. Then:
                    \begin{equation*}
                        F(x) + G(x) = \sum_{k \geq 0}(a_{k} + b_{k})x^{k}
                    \end{equation*}
                and therefore,
                    \begin{align*}
                        \dv{x}(F(x) + G(x)) = \sum_{k \geq 0}k(a_{k} + b_{k})x^{k - 1} &= \sum_{k \geq 0} ka_{k}x^{k - 1} + kb_{k}x^{k - 1} \\
                        &= \sum_{k\geq 0}ka_{k}x^{k - 1} + \sum_{j \geq 0} jb_{j}x^{j - 1} \\
                        &= F^{\prime}(x) + G^{\prime}(x)
                    \end{align*}
                so we are done with the additive one. Now for the multiplicative one:
                    \begin{equation*}
                        F(x)G(x) = \sum_{i \geq 0}\sum_{j \geq 0}^{i} a_{j}b_{i - j}x^{i}
                    \end{equation*}
                So the derivative:
                    \begin{align*}
                        \dv{x}F(x)G(x) &= \sum_{i \geq 0}\sum_{j \geq 0}^{i} (i)a_{j}b_{i - j}x^{i - 1} 
                    \end{align*}
                One the other hand, 
                    \begin{equation*}
                        F^{\prime}(x)G(x) = \sum_{i \geq 0}\sum_{j \geq 0}^{i} (i - j + 1)a_{j}b_{i - j + 1} x^{i}
                    \end{equation*}
                and
                    \begin{align*}
                        F(x)G^{\prime}(x) &= \sum_{i \geq 0}\sum_{j \geq 0}^{i - 1} (j + 1)a_{j + 1}b_{i - j}x^{i} \\
                                          &= \sum_{i \geq 0}\sum_{j \geq 1}^{i} (j)a_{j}b_{i - j + 1}x^{i}
                    \end{align*}
                so 
                    \begin{align*}
                        F(x)G^{\prime}(x) + F^{\prime}(x)G(x) &= \sum_{i \geq 0}\sum_{j \geq 0}^{i}(i - j + 1)a_{j}b_{i - j + 1}x^{i} + \sum_{i \geq 0}\sum_{j \geq 1}^{i}(j)a_{j}b_{i - j + 1}x^{i} \\
                        &= \sum_{i \geq 0}\sum_{j \geq 0}^{i}(i + 1)a_{j}b_{i - j + 1}x^{i} \\
                        &= \sum_{i \geq 0}\sum_{j \geq 0}^{i} (i)a_{j}b_{i - j}x^{i - 1} = \dv{x}F(x)G(x)
                    \end{align*}
                so we are done.
            \end{proof}


        \item Show that $\dv{x}(1 + x)^{a} = a(1 + x)^{a - 1}$ where $a$ is an arbitrary number (we have defined these expressions using binomial theorem). 
            \begin{proof}
                We know that 
                    \begin{equation*}
                        (1 + x)^{a} = \sum_{k \geq 0} \dbinom{a}{k} x^{k}
                    \end{equation*}
                So taking the derivative:
                    \begin{align*}
                        \dv{x}(1 + x)^{a} = \sum_{k \geq 0}k\dbinom{a}{k}x^{k - 1} &= \sum_{k \geq 1}\dfrac{a (a - 1)(a - 2) \cdots (a - k + 1)}{(k - 1)!}x^{k - 1} \\
                                                                                   &= \sum_{k \geq 1}a \dfrac{(a - 1) \cdots (a - k + 1)}{(k - 1)!}x^{k - 1} \\
                                                                                   &= a\sum_{k \geq 1} \dfrac{(a - 1)!}{(a - k)!(k - 1)!} x^{k - 1} \\
                                                                                   &= a \sum_{k \geq 1} \dbinom{a - 1}{k - 1} x^{k - 1} =  a\sum_{k \geq 0}\dbinom{a - 1}{k}x^{k} = a(1 + x)^{a - 1}
                    \end{align*}
                so we are done.
            \end{proof}
    \end{itemize}


\textbf{Exercise 4}: Find a closed expression for the generating function for the number of functions $[n] \rightarrow [k]$ with fixed $k$, that is find expression for 
    \begin{equation*}
        \sum_{n \geq 0}k^{n}x^{n}
    \end{equation*}
which does not involve infinite sums.

    \begin{proof}
        We have 
            \begin{equation*}
                S = \sum_{n \geq 0}k^{n}x^{n}
            \end{equation*}
        and
            \begin{equation*}
                xkS = \sum_{n \geq 0}k^{n + 1}x^{n + 1}
            \end{equation*}
        Then
            \begin{equation*}
                S - xkS = 1
            \end{equation*}
        and
            \begin{equation*}
                S(1 - xk) = 1
            \end{equation*}
        so
            \begin{equation*}
                S = \dfrac{1}{1 - xk}
            \end{equation*}
        which is the expression.
    \end{proof}

\textbf{Exercise 5}: Let $\kappa(n, k, j)$ denote the number of weak compositions of $n$ into $k$ parts that each part is less than $j$.
    \begin{itemize}
        \item Compute the generating function $\sum_{n \geq 0}\kappa(n, k, j)x^{n}$ (more precisely, express it as a rational function in $x$ without using summations with the number of summands depending on $k$ or $j$).
            \begin{proof}
                Notice that the number of weak compositions such that no part has less than $j$ things plus the number of compositions where at least one part has $j$ things is equal to the number of  compositions total give by $\frac{1}{(1 - x)^{k}}$. So we have 
                    \begin{equation*}
                        \kappa(n, k, j) + \kappa(n - j, k, n + 1) = \kappa(n, k, n + 1)
                    \end{equation*}
                or 
                    \begin{equation*}
                        \kappa(n, k, j) = \kappa(n, k, n + 1) - \kappa(n - j, k, n + 1)
                    \end{equation*}
                so this is given by:
                    \begin{equation*}
                        \dfrac{1}{(1 - x)^{k}} - \dfrac{x^{j}}{(1 - x)^{k}}
                    \end{equation*}
                But our choice of which part received at least a count of $j$ was not arbitrary. So we apply principle of inclusion exclusion: 
                    \begin{equation*}
                        \dfrac{1}{(1 - x)^{k}} - \dfrac{kx^{j}}{1 - x^{k}} + \dfrac{\dbinom{k}{2}x^{2j}}{1 - x^{k}} - \cdots 
                    \end{equation*}
            \end{proof}

        \item Prove that
            \begin{equation*}
                \kappa(n, k, j) = \sum_{r + sj = n}(-1)^{s}\dbinom{k + r - 1}{r} \dbinom{k}{s}.
            \end{equation*}
                \begin{proof}
                    By the previous part, we first know that 
                        \begin{equation*}
                            \dfrac{1}{(1 - x)^{k}} = \sum_{r \geq 0}\dbinom{k + r - 1}{r}x^{r}
                        \end{equation*}
                    The second term is just $\binom{k}{1}$ times this but with the sign flipped:
                        \begin{equation*}
                            \dfrac{kx^{j}}{(1 - x)^{k}} = \sum_{r \geq j} (-1)\dbinom{k + r - 1}{r }\dbinom{k}{1}
                        \end{equation*}
                    And in general:
                        \begin{equation*}
                            \dfrac{\dbinom{k}{s}x^{sj}}{(1 - x)^{k}} = \sum_{r + sj = n}(-1)^{s}\dbinom{k + r - 1}{r}\dbinom{k}{s}
                        \end{equation*}
                    and we keep going, summing up these terms over $s$ until $n = sj + r$ where $r < j$ which gives us the formula seen up top.
                \end{proof}
    \end{itemize}

















\end{document}
