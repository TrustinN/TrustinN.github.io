%! TeX root = /Users/trustinnguyen/Downloads/Berkeley/Math/Math172/Homework/Math172Hw10/Math172Ex10.tex

\documentclass{article}
\usepackage{/Users/trustinnguyen/.mystyle/math/packages/mypackages}
\usepackage{/Users/trustinnguyen/.mystyle/math/commands/mycommands}
\usepackage{/Users/trustinnguyen/.mystyle/math/environments/article}
\graphicspath{{./figures/}}

\title{Math172Ex10}
\author{Trustin Nguyen}

\begin{document}

    \maketitle

\reversemarginpar

\textbf{Exercise 4}: Fix $n \geq 3$ and let $A$ be the graph obtained from $K_{n}$ by removing one edge. Find the number of spanning trees of $A$ (we count them as labeled trees, so different trees will be counted separately even if they are isomorphic).
    \begin{proof}
        We know that the number of spanning trees on a complete graph is $n^{n - 2}$, because that is just the number of trees on $[n]$ which was proved in class. Now the number of spanning trees is a subset of these $n^{n - 2}$ strings. Here is the method for the code:
            \begin{itemize}
                \item Find the leaf with the minimal label

                \item Remove this leaf, and add to the code what the edge led to.

                \item Repeat this procedure for $n - 2$ times. 

                \item Stop when there is a tree on two vertices.
            \end{itemize}

        Without loss of generality, remove the edge connecting $1, n$ in $K_{n}$. Then there are two cases:
            \begin{itemize}
                \item Converting a prufer code to a tree ends in adding the edge $\{1, n\}$ at the end. Because there is a bijection between prufer codes and trees, consider the process of constructing a prufer code. 

                First show that one of the remaining vertices is always $n$. Every tree has at least $2$ leaves. Then since $n$ is the largest number in our graph, there will always be a smaller leaf $n^{\prime}$ that we will remove instead of $n$. So $n$ will be one of the remaining vertices at the end.

                Suppose the last two vertices are $1, n$. We know that whatever leaf we removed before must be connected to either $1$ or $5$. Suppose that it is connected to $5$, and had the label $m$. But that is a contradiction, because we must have the graph: 
                    \begin{center}
                        \begin{tikzcd}
                            n\ar[r, "", dash] & 1 \\
                            m\ar[u, "", dash] &     
                        \end{tikzcd}
                    \end{center}
                But then since $1$ is the smallest leaf, we will have removed $\{1, n\}$ as an edge, so $1, n$ are not the last vertices remaining. Then we know that graphs that contain the edge $\{1, n\}$ contribute to prufer codes that end in $1$.

                \item The edge $\{1, n\}$ is removed before the last step. Since the vertex $n$ is never a leaf that is removed until the last step, we know that it is the leaf $1$ that is removed that is connected to the vertex $n$. Then that means that in our prufer code, this corresponds to adding $n$ to our code. We must also have no $1$'s appearing after.

                If $\{1, n\}$ was removed in the first step, then the prufer code contains no $1$'s and starts with vertex $n$. Otherwise, $1$ was not a leaf. Then we must have removed several edges connecting to $1$, which would turn $1$ into a leaf, which is the smallest one. So then an instance of $1, n$ in the prufer code, followed by $0$ instances of $1$ indicates that we have removed $\{1, n\}$ as an edge.
            \end{itemize}
        Notice that the set of prufer codes that account for these cases are disjoint. For the first case, we require that $1$ appears as the last element of the code. So this is just $n^{n - 3}$, because we set the last element in our $n - 2$ element string to be $1$ and have $n$ options for the rest of $n - 3$ places in our string. 

        In the next case, suppose we had a string of length $n - 3$. Then we can account for an instance of $1, 5$ in our code with no $1$'s after by taking the last instance of $1$ in our code and replacing it with $1, 5$. If there are no $1$'s, then we append a $5$ to the start of the prufer code. There are $n^{n - 3}$ ways to have such strings in this case.

        So now we take the number of prufer codes total corresponding to the number of spanning trees on $K_{n}$
            \begin{equation*}
                n^{n - 2}
            \end{equation*}
        and remove the number of prufer codes that use the edge $\{1, n\}$:
            \begin{equation*}
                n^{n - 3} + n^{n - 3}
            \end{equation*}
        to get the number of spanning trees on $K_{n}$ with one edge removed:
            \begin{equation*}
                n^{n - 2} - 2n^{n - 3}
            \end{equation*}
        this is not defined for $n = 1, n = 2$, so the number of spanning trees for those is $1$ which completes the proof.
    \end{proof}






\end{document}
