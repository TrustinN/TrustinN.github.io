%! TeX root = /Users/trustinnguyen/Downloads/Berkeley/Math/Math128a/Homework/Math128aHw5/Math128aHw5.tex

\documentclass{article}
\usepackage{/Users/trustinnguyen/.mystyle/math/packages/mypackages}
\usepackage{/Users/trustinnguyen/.mystyle/math/commands/mycommands}
\usepackage{/Users/trustinnguyen/.mystyle/math/environments/article}
\graphicspath{{./figures/}}

\title{Math128aHw5}
\author{Trustin Nguyen}

\begin{document}

    \maketitle

\reversemarginpar


\section*{Exercise Set 3.4}
\hrule

\textbf{Exercise 6}: Let $f(x) = 3xe^{x} - e^{2x}$.
    \begin{itemize}
        \item [(a)] Approximate $f(1.03)$ by the Hermite interpolating polynomial of degree at most three using $x_{0} = 1$ and $x_{1} = 1.05$. Compare the actual error to the error bound.
            \begin{answer}
                I got $P(1.03) = 0.8093238572598097091415070281073$ using the divided difference method. The actual value is $0.80932361890170567442627169786669$. The absolute error is $0.00000023835810403471523533024060767165$

                Here is my code:
                \inputminted{matlab}{./code/Hermite/Hermite.m}
                \inputminted{matlab}{./code/Hermite/forwardPoly.m}
                \inputminted{matlab}{./code/script1.m}
            \end{answer}

        \item [(b)] Repeat $(a)$ with the Hermite interpolating polynomial of degree at most five using $x_{0} = 1$, $x_{1} = 1.05$, and $x_{2} = 1.07$.
            \begin{answer}
                I got $P(1.03) = 0.80932361724423707016740081598982$ with the actual value being $f(1.03) = 0.80932361890170567442627169786669$. The error is therefore $0.0000000016574686042588708818768687171581$

                Here is my code:
                \inputminted{matlab}{./code/Hermite/Hermite.m}
                \inputminted{matlab}{./code/Hermite/forwardPoly.m}
                \inputminted{matlab}{./code/script2.m}
            \end{answer}
    \end{itemize}

\textbf{Exercise 7}: The following table lists data for the function described by $f(x) = e^{0.1x^{2}}$. Approximate $f(1.25)$ by using $H_{5}(1.25)$ and $H_{3}(1.25)$, where $H_{5}$ uses the nodes $x_{0} = 1, x_{1} =2 $, and $x_{2} = 3$ and $H_{3}$ uses the nodes $\overline{x_{0}} = 1$ and $\overline{x_{1}} = 1.5$. Find error bounds for these approximations.
    \begin{align*}
        \begin{array}{ c c c }
            x                            & f(x) = e^{0.1x^{2}} & f^{\prime}(x) = 0.2xe^{0.1x^{2}} \\
            x_{0} = \overline{x_{0}} = 1 & 1.105170918         & 0.2210341836                     \\
            \overline{x_{1}} = 1.5       & 1.252322716         & 0.3756968148                     \\
            x_{1} = 2                    & 1.491824698         & 0.5967298792                     \\
            x_{2} = 3                    & 2.459603111         & 1.475761867                        
        \end{array}
    \end{align*}
        \begin{answer}
            Using two nodes, I got $P(1.25) = 1.1696528199671771819367904754472$ while the true value is $f(1.25) = 1.1691184461695044022981846915119$. The error is then $0.00053437379767277963860578393530507$.

            Using three nodes, I got $P(1.25) = 1.1709556512384045046104574794299$ with the true value being $f(1.25) = 1.1691184461695044022981846915119$. Then the error is $0.0018372050689001023122727879180055$.

            We see that this time, using two nodes was better than using three, because the two nodes were closer to where we wanted to approximate $f$ at.

            Here is my code:
            \inputminted{matlab}{./code/Hermite/Hermite.m}
            \inputminted{matlab}{./code/Hermite/forwardPoly.m}
            \inputminted{matlab}{./code/script3.m}
            \inputminted{matlab}{./code/script4.m}
        \end{answer}

\textbf{Exercise 10}: Let $z_{0} = x_{0}, z_{2} = x_{1}, $ and $z_{3} = x_{1}$. Form the following divided-difference table.
    \begin{align*}
        \begin{array}{ c c c c c }
            z_{0} = x_{0} & f[z_{0}] = f(x_{0}) &                                     &                        &                               \\
                          &                     & f[z_{0}, z_{1}] = f^{\prime}(x_{0}) &                        &                               \\
            z_{1} = x_{0} & f[z_{1}] = f(x_{0}) &                                     & f[z_{0}, z_{1}, z_{2}] &                               \\
                          &                     & f[z_{1}, z_{2}]                     &                        & f[z_{0}, z_{1}, z_{2}, z_{3}] \\
            z_{2} = x_{1} & f[z_{2}] = f(x_{1}) &                                     & f[z_{1}, z_{2}, z_{3}] &                               \\
                          &                     & f[z_{2}, z_{3}] = f^{\prime}(x_{1}) &                        &                               \\
            z_{3} = x_{1} & f[z_{3}] = f(x_{1}) &                                     &                        &                                 
        \end{array}
    \end{align*}
Show that the cubic Hermite polynomial $H_{3}(x)$ can also be written as $f[z_{0}] + f[z_{0}, z_{1}](x - x_{0}) + f[z_{0}, z_{1}, z_{2}](x- x_{0})^{2} + f[z_{0}, z_{1}, z_{2}, z_{3}](x - x_{0})^{2}(x - x_{1})$
    \begin{answer}
        The Hermite polynomial is given by:
            \begin{equation*}
                H_{2n + 1}(x) = f[z_{0}] + \sum_{k = 1}^{2n + 1} f[z_{0}, \ldots, z_{k}](x - z_{0})(x - z_{1}) \cdots (x - z_{k - 1})
            \end{equation*}
        There are two nodes so $n = 1$. Then we get:
            \begin{equation*}
                H_{3}(x) = f[z_{0}] + f[z_{0}, z_{1}](x - z_{0}) + f[z_{0}, z_{1}, z_{2}](x - z_{0})(x - z_{1}) + f[z_{0}, z_{1}, z_{2}, z_{3}](x - z_{0})(x - z_{1})(x - z_{2})
            \end{equation*}
        Substitute in $z_{0}, z_{1}, z_{2}, z_{3} = x_{0}, x_{0}, x_{1}, x_{1}$ to get:
            \begin{equation*}
                H_{3}(x) = f[z_{0}] + f[z_{0}, z_{1}](x - x_{0}) + f[z_{0}, z_{1}, z_{2}](x - x_{0})^{2} + f[z_{0}, z_{1}, z_{2}, z_{3}](x - x_{0})^{2}(x - x_{1})
            \end{equation*}
    \end{answer}

\newpage
\section*{Exercise Set 3.5}
\hrule

\textbf{Exercise 1}: Determine the natural cubic spline $S$ that interpolates the data $f(0) = 0$, $f(1) = 1$, ad $f(2) = 2$.
    \begin{answer}
        For a natural cubic spline, we require the conditions:
            \begin{itemize}
                \item [(a)] $S_{i}(x_{i + 1}) = S_{i + 1}(x_{i + 1})$

                \item [(b)] $S^{\prime}_{i}(x_{i + 1}) = S^{\prime}_{i + 1}(x_{i + 1})$

                \item [(c)] $S^{\prime\prime}_{i}(x_{i + 1}) = S^{\prime\prime}_{i + 1}(x_{i + 1})$.  
            \end{itemize}
        Then we have 
            \begin{align*}
                S_{0}(x) &= 0 + b_{0}(x - 0) + c_{0}(x - 0)^{2} + d_{0}(x - 0)^{3} \\
                S_{1}(x) &= 1 + b_{1}(x - 1) + c_{1}(x - 1)^{2} + d_{1}(x - 1)^{3}   
            \end{align*}
        By condition $(a)$:
            \begin{align*}
                S_{0}(1) &= b_{0} + c_{0} + d_{0} = 1 \\
                S_{1}(2) &= 1 + b_{1} + c_{1} + d_{1} = 2
            \end{align*}
        By condition $(b)$:
            \begin{align*}
                S^{\prime}_{0}(1) &= b_{0} + 2c_{0} + 3d_{0} \\
                S^{\prime}_{1}(1) &= b_{1}                 
            \end{align*}
        By condition $(c)$:
            \begin{align*}
                S^{\prime\prime}_{0}(1) &= 2c_{0} + 6d_{0} \\
                S^{\prime\prime}_{1}(1) &= 2c_{1}            
            \end{align*}
        are equal. Because it is a natural spline,
            \begin{align*}
                S^{\prime\prime}_{0}(0) &= 2c_{0} = 0          \\
                S^{\prime\prime}_{1}(2) &= 2c_{1} + 6d_{1} = 0   
            \end{align*}
        By condition $(c)$, $2c_{1} = 6d_{0}$, and from the natural boundary condition:
            \begin{equation*}
                0 = d_{0} + d_{1}, d_{0} = -d_{1}
            \end{equation*}
        From condition $(a)$, $b_{0} + d_{0} = 1$, and from $(b)$, 
            \begin{equation*}
                S^{\prime}_{0}(1) = 2d_{0} + 1 = b_{1} = S^{\prime}_{1}(1)
            \end{equation*}
        so
            \begin{equation*}
                2d_{0} + 1 = b_{1}
            \end{equation*}
        From condition $(a)$, 
            \begin{equation*}
                S_{1}(2) = 2 + 2d_{0} + c_{1} + d_{1} = 2
            \end{equation*}
        and
            \begin{equation*}
                0 = c_{1} - d_{1}
            \end{equation*}
        Recall $c_{1} = 3d_{0}$ from condition $(c)$. Then
            \begin{equation*}
                0 = 3d_{0} - d_{1} = -4d_{1}
            \end{equation*}
        and
            \begin{equation*}
                d_{1} = 0
            \end{equation*}
        This gives:
            \begin{align*}
                c_{0} &= 0             \\
                c_{1} &= 0  \\
                d_{0} &= 0 \\
                d_{1} &= 0  
            \end{align*}
        so far. By condition $(a)$, 
            \begin{align*}
                1            &= b_{0} 0                    \\
                2            &= 1 + b_{1} \\
                1 &= b_{1}                                     
            \end{align*}
        Then
            \begin{align*}
                S_{0}(x) &= x                                          \\
                S_{1}(x) &= 1 + (x - 1) = x
            \end{align*}
    \end{answer}

\textbf{Exercise 2}: Determine the clamped cubic spline $s$ that interpolates the data $f(0) = 0$, $f(1) = 1$ and $f(2) = 2$ and satisfies $s^{\prime}(0) = s^{\prime}(2) = 1$.
    \begin{answer}
        A cubic spline satisfies:
            \begin{itemize}
                \item [(a)] $S_{i}(x_{i + 1}) = S_{i + 1}(x_{i + 1})$

                \item [(b)] $S^{\prime}_{i}(x_{i + 1}) = S^{\prime}_{i + 1}(x_{i + 1})$

                \item [(c)] $S^{\prime\prime}_{i}(x_{i + 1}) = S^{\prime\prime}_{i + 1}(x_{i + 1})$ 
            \end{itemize}
        Since it is clamped, there are additional boundary conditions on the first derivative. Firstly:
            \begin{align*}
                S_{0}(x) &= b_{0}x + c_{0}x^{2} + d_{0}x^{3}                       \\
                S_{1}(x) &= 1 + b_{1}(x - 1) + c_{1}(x - 1)^{2} + d_{1}(x - 1)^{3}   
            \end{align*}
        We have by $(a)$:
            \begin{align*}
                S_{0}(1) &= b_{0} + c_{0} + d_{0} = 1     \\
                S_{1}(2) &= 1 + b_{1} + c_{1} + d_{1} = 2   
            \end{align*}
        By $(b)$:
            \begin{align*}
                S^{\prime}_{0}(1) &= b_{0} + 2c_{0} + 3d_{0} \\
                S^{\prime}_{1}(1) &= b_{1}                     
            \end{align*}
        are equal. By $(c)$:
            \begin{align*}
                S^{\prime\prime}_{0}(1) &= 2c_{0} + 6d_{0} \\
                S^{\prime\prime}_{1}(1) &= 2c_{1}            
            \end{align*}
        are equal. Finally, by the clamped boundary condition:
            \begin{align*}
                S^{\prime}_{0}(0) &= b_{0} = 1 \\
                S^{\prime}_{1}(2) &= b_{1} + 2c_{1} + 3d_{1} = 1   
            \end{align*}
        Because $b_{0} = 1$, by $(a)$, we have $c_{0} + d_{0} = 0$. Then by $(b)$, $b_{0} + 2c_{0} + 3d_{0} = b_{1}$ means that $1 + d_{0} = b_{1}$. By $(c)$, we also have the relation $2c_{0} + 6d_{0} = 2c_{1}$ and therefore, $c_{1} = 2d_{0}$. Now by $(a)$, $1 + b_{1} + c_{1} + d_{1} = 2$ means that $d_{1} = 1 - b_{1} - c_{1}$, and substituting into $S^{\prime}_{1}(2)$, 
            \begin{equation*}
                b_{1} + 2c_{1} + 3(1 - b_{1} - c_{1}) = 1
            \end{equation*}
        or
            \begin{equation*}
                2 = 2b_{1} + c_{1}
            \end{equation*}
        Using $b_{1} = 1 + d_{0}, c_{1} = 2d_{0}$,
            \begin{equation*}
                2 = 2(1 + d_{0}) + 2d_{0}
            \end{equation*}
        so
            \begin{equation*}
                2 = 4d_{0}, d_{0} = 0
            \end{equation*}
        This also gives us
            \begin{align*}
                b_{1} &= 1  \\
                c_{1} &= 0 
            \end{align*}
        We have
            \begin{align*}
                d_{1} &= 1 - b_{1} - c_{1}               \\
                d_{1} &= 0
            \end{align*}
        Putting this all together:
            \begin{align*}
                S_{0}(x) &= x                                   \\
                S_{1}(x) &= x 
            \end{align*}
    \end{answer}

\end{document}
