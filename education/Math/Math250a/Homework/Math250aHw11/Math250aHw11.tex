%! TeX root = /Users/trustinnguyen/Downloads/Berkeley/Math/Math250a/Homework/Math250aHw11/Math250aHw11.tex

\documentclass{article}
\usepackage{/Users/trustinnguyen/.mystyle/math/packages/mypackages}
\usepackage{/Users/trustinnguyen/.mystyle/math/commands/mycommands}
\usepackage{/Users/trustinnguyen/.mystyle/math/environments/article}
\graphicspath{{./figures/}}

\title{Math250aHw11}
\author{Trustin Nguyen}

\begin{document}

    \maketitle

\reversemarginpar

\textbf{Exercise 1}: Let $E = \mathbb{Q}(\alpha)$, where $\alpha$ is a root of the equation
    \begin{equation*}
        \alpha^{3} + \alpha^{2} + \alpha + 2 = 0
    \end{equation*}
Express $(\alpha^{2} + \alpha + 1)(\alpha^{2} + \alpha)$ and $(\alpha - 1)^{-1}$ in the form
    \begin{equation*}
        a\alpha^{2} + b\alpha + c
    \end{equation*}
with $a, b, c \in \mathbb{Q}$.
    \begin{proof}
        (First Expression) Since $\alpha^{3} + \alpha^{2} + \alpha + 2 = 0$, we know that 
            \begin{equation*}
                \alpha^{3} + \alpha^{2} + \alpha  = -2
            \end{equation*}
        Algebra manipulations:
            \begin{align*}
                (\alpha^{2} + \alpha + 1)(\alpha^{2} + \alpha) &= \alpha (\alpha^{2} + \alpha + 1)(\alpha + 1) \\
                                                               &= (\alpha^{3} + \alpha^{2} + \alpha) (\alpha + 1)\\
                                                               &= (-2)(\alpha + 1) \\
                                                               &= -2\alpha - 2
            \end{align*}
        So $a = 0, b = -2, c = -2$.

        For the second, let $(\alpha - 1)^{-1} = a\alpha^{2} + b\alpha + c$. We have:
            \begin{align*}
                (\alpha - 1)^{-1}(\alpha - 1)           &= 1 \\
                (a\alpha^{2} + b\alpha + c)(\alpha - 1) &= 1   
            \end{align*}
        Rewriting, we get:
            \begin{equation*}
                (a\alpha^{3} + b\alpha^{2} + c\alpha) - (a - 1)^{-1} = 1   
            \end{equation*}
        Using $ a\alpha^{3} + a\alpha^{2} + a\alpha + 2a = 0$, we have:
            \begin{align*}
                (b - a)\alpha^{2} + (c - a)\alpha - 2a - (a - 1)^{-1} &= 1 \\
                (b - 2a)\alpha^{2} + (c - a - b)\alpha - 2a - c       &= 1   
            \end{align*}
        Since we only have a degree 3 relation:
            \begin{equation*}
                \alpha^{3} + \alpha^{2} + \alpha + 3 = 1
            \end{equation*}
        And there is probably no polynomial that divides $x^{3} + x^{2} + x + 2$, in $\mathbb{Z}[x]$, set the coefficients of
            \begin{equation*}
                (b - 2a)\alpha^{2} + (c - a - b)\alpha - 2a - c = 1
            \end{equation*}
        To $0$ to get the system:
            \begin{align*}
                b - 2a    &= 0 \\
                c - a - b &= 0 \\
                -2a - c   &= 1   
            \end{align*}
        Solving this system, we get $a = \frac{-1}{5}, b = \frac{-2}{5}, c = \frac{-3}{5}$. Therefore, 
            \begin{equation*}
                (\alpha - 1)^{-1} = \dfrac{-1}{5}\alpha^{2} - \dfrac{2}{5}\alpha - \dfrac{3}{5}
            \end{equation*}
    \end{proof}

\textbf{Exercise 3}: Let $\alpha$ and $\beta$ be two elements which are algebraic over $F$. Let $f(X) = \mathop{Irr}(\alpha, F, X)$ and $g(X) = \mathop{Irr}( \beta, F, X)$. Suppose the $\deg f$ and $\deg g$ are relatively prime. Show that $g$ is irreducible in the polynomial ring $F(\alpha) [X]$. 
    \begin{proof}
        Consider the field extensions:
            \begin{align*}
                F \subset & F(\alpha)  \subset F(\alpha, \beta) \\
                F \subset & F(\beta)   \subset F(\alpha, \beta)   
            \end{align*}
        Let $\deg f(x) = m$ and $\deg g(x) = n$. Then $\gcd{m, n} = 1$. We will use the fact that
            \begin{equation*}
                [F(\alpha, \beta): F(\beta)] [F(\beta): F] = [F(\alpha, \beta) : F] = [F(\alpha, \beta) : F(\alpha)][F(\alpha) : F] 
            \end{equation*}
        So we have:
            \begin{equation*}
                n[F(\alpha, \beta) : F(\beta)] = m[F(\alpha, \beta): F(\alpha)]
            \end{equation*}
        because the gcd is $1$, we have $n \divides [F(\alpha, \beta): F(\alpha)]$. So the irreducible polynomial of $F(\alpha, \beta)$ over $F(\alpha)$ that kills $\beta$ is of degree $n$. Furthermore, this polynomial must divide $g(X)$. So this polynomial is $g(X)$, which concludes the proof.
    \end{proof}

\textbf{Exercise 4}: Let $\alpha$ be the real positive fourth root of $2$. Find all intermediate fields in the extension $\mathbb{Q}(\alpha)$ of $\mathbb{Q}$.
    \begin{proof}
        We know that
            \begin{equation*}
                \alpha^{4} - 2 = 0
            \end{equation*}
        So $\mathbb{Q}(\alpha) \cong \mathbb{Q}[\alpha] \cong \mathbb{Q}[X]/(X^{4} - 2)$. Let $F$ be an intermediate field. Then we know that $[\mathbb{Q}(\alpha):\mathbb{Q}] =[\mathbb{Q}(\alpha) : F][F : \mathbb{Q}]$. Then 
            \begin{equation*}
                4 = [\mathbb{Q}(\alpha) : F][F : \mathbb{Q}]
            \end{equation*}
        If $[F : \mathbb{Q}] = 4$, then we know there is an isomorphism between $\mathbb{Q}(\alpha), F$. Otherwise, if $[F : \mathbb{Q}] = 2$, then we can consider $\alpha^{2}$, which satisfies the equation $x^{2} - 2 = 0$. So $\mathbb{Q}(\alpha^{2})$ is the intermediate field. So we have the intermediate field ordering as: 
            \begin{equation*}
                \mathbb{Q} \subset \mathbb{Q}(\alpha^{2}) \subset \mathbb{Q}(\alpha)
            \end{equation*}
        There are no other intermediate field extensions, because the degree of the extension will divide $2$:
            \begin{equation*}
                2 = [\mathbb{Q}(\alpha)^{2} : E][E : \mathbb{Q}], 2 = [\mathbb{Q}(\alpha) : \mathbb{Q}(\alpha^{2}][\mathbb{Q}(\alpha^{2}): \mathbb{Q}]
            \end{equation*}
        in which case, one of the extensions is of degree $1$ and is trivial.
    \end{proof}

\textbf{Exercise 10}: Let $\alpha$ be a real number such that $\alpha^{4} = 5$.
    \begin{itemize}
        \item [(a)] Show that $\mathbb{Q}(i\alpha^{2})$ is normal over $\mathbb{Q}$.
            \begin{proof}
                We have that $(i\alpha^{2})^{2} = -\alpha^{4}$. So $(i\alpha^{2})^{2} + 5 = 0$. Then $[\mathbb{Q}(i\alpha^{2}) : \mathbb{Q}] = 2$ because $1, i\alpha$ forms a basis for $\mathbb{Q}(i\alpha^{2})$ over $\mathbb{Q}$. But any extension of degree $2$ is normal. This is because if the irreducible poly is $x^{2} + bx + c = (x - \alpha) (x - \beta)$, we have  $b = -\alpha - \beta$. So $\beta = b + \alpha$ and $\beta \in \mathbb{Q}(i\alpha^{2})$.
            \end{proof}

        \item [(b)] Show that $\mathbb{Q}( \alpha + i\alpha)$ is normal over $\mathbb{Q}(i\alpha^{2})$.
            \begin{proof}
                We have $(\alpha + i\alpha)^{2} = \alpha^{2} + 2i\alpha^{2} - \alpha^{2} = 2i\alpha^{2}$. Therefore, $(\alpha + i\alpha)^{2} - 2i\alpha^{2} = 0$. The minimal polynomial is $x^{2} - 2i\alpha^{2}$, which is of degree $2$. This polynomial is irreducible in $\mathbb{Q}(i\alpha^{2})[x]$. So $[\mathbb{Q}(\alpha + i\alpha) : \mathbb{Q}(i\alpha^{2})]= 2$ and any extension of degree $2$ is normal.
            \end{proof}

        \item [(c)] Show that $\mathbb{Q}(\alpha + i\alpha)$ is not normal over $\mathbb{Q}$. 
            \begin{proof}
                By the previous problems, we have:
                    \begin{equation*}
                        [\mathbb{Q}(\alpha + i\alpha) : \mathbb{Q}] = [\mathbb{Q}(\alpha + i\alpha) : \mathbb{Q}(i\alpha^{2})][\mathbb{Q}(i\alpha^{2}) : \mathbb{Q}] = 4
                    \end{equation*}
                We have $x^{2} - 2i\alpha^{2} = 0$ if $x = (\alpha + i\alpha)$. We need a polynomial in $\mathbb{Q}[X]$ that kills $(\alpha + i\alpha)$ so $(x^{2} - 2i\alpha^{2}) (x^{2} + 2i\alpha^{2}) = 0$, $x^{4} + 4\alpha^{4} = x^{4} + 20 = 0$. If $\mathbb{Q}(\alpha + i\alpha)$ is a normal extension, then it must contain all roots of $x^{4} + 20$.

                So $x^{2} + 2i\alpha^{2} = 0$ for some $x \in \mathbb{Q}(\alpha + i\alpha)$. We see that $i(\alpha + i\alpha) = i\alpha - \alpha$ is a root. But if $i\alpha - \alpha \in \mathbb{Q}(\alpha + i\alpha)$, Then $\mathbb{Q}(\alpha + i\alpha) = \mathbb{Q}(i\alpha^{2})$, which is not true, because there is a degree $2$ extension from $\mathbb{Q}(i\alpha^{2})  \subseteq \mathbb{Q}(\alpha + i\alpha)$.

                Then this is not a normal extension because not all roots of $x^{4} + 20$ are in $\mathbb{Q}(\alpha + i\alpha)$.
            \end{proof}
    \end{itemize}












\end{document}
