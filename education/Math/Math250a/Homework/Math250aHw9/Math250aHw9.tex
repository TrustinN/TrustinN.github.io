%! TeX root = /Users/trustinnguyen/Downloads/Berkeley/Math/Math250a/Homework/Math250aHw9/Math250aHw9.tex

\documentclass{article}
\usepackage{/Users/trustinnguyen/.mystyle/math/packages/mypackages}
\usepackage{/Users/trustinnguyen/.mystyle/math/commands/mycommands}
\usepackage{/Users/trustinnguyen/.mystyle/math/environments/article}
\graphicspath{{./figures/}}

\title{Math250aHw9}
\author{Trustin Nguyen}

\begin{document}

    \maketitle

\reversemarginpar

\textbf{Exercise 1}: Recall that $R = \mathbb{Z}[i]$ is a UFD. What is the content of the polynomial $(5 + 5i)X^{2} + (-1 + 3i)X + 2 \in R[X]$?
    \begin{proof}
        We know that $\mathbb{Z}[i]$ is a Euclidean domain with a measure of size as the norm. Then we have:
            \begin{equation*}
                5 + 5i =  5(1 + i)
            \end{equation*}
        where $1 + i$ is irreducible because 
            \begin{equation*}
                \lVert 1 + i \rVert = 2
            \end{equation*}
        and then if $1 + i$ is a product of two complex numbers: $z_{1}z_{2}$, then $\lVert z_{1} \rVert\lVert z_{2} \rVert = 2$, so one of them is a unit. Now for $5$, we see that:
            \begin{equation*}
                5 = 1^{2} + 2^{2}
            \end{equation*}
        so we can take:
            \begin{equation*}
                (1 + 2i)(1 - 2i) = 5
            \end{equation*}
        Now for the second coefficient, we have:
            \begin{equation*}
                \lVert -1 + 3i \rVert = 10
            \end{equation*}
        and if $-1 + 3i = (a + bi)(c + di)$, then 
            \begin{equation*}
                (a^{2}+ b^{2})(c^{2} + d^{2}) = 10
            \end{equation*}
        which means that either factor is either $1$ and $10$ or $2$ and $5$. If it is $1$, then one factor is a unit and we are done. If 
            \begin{equation*}
                a^{2} + b^{2} = 2 \text{ and } c^{2} + d^{2} = 5
            \end{equation*}
        we have
            \begin{align*}
                a &= \pm 1                   \\
                b &= \pm 1                   \\
                c &= \pm 1 \text{ or } \pm 2 \\
                d &= \pm 1 \text{ or } \pm 2   
            \end{align*}
        We get 
            \begin{equation*}
                (a + bi)(c + di) = (ac - bd) + (ad + bc)i = -1 + 3i
            \end{equation*}
        and by trial and error, we find:
            \begin{equation*}
                (1 + i)(1 + 2i) = -1 + 3i
            \end{equation*}
        and both factors are irreducible. Finally
            \begin{equation*}
                2 = (1 + i)(1 - i)
            \end{equation*}
        So we have
            \begin{equation*}
                (1 + 2i)(1 - 2i)(1 + i)X^{2} + (1 + i)(1 + 2i)X + (1 + i)(1 - i)
            \end{equation*}
        Then we pull out the largest power of each prime that divides each coefficient. We get that the content is 
            \begin{equation*}
                (1 + i)
            \end{equation*}
        so we are done.
    \end{proof} 

\textbf{Exercise 2}: 
    \begin{itemize}
        \item [(a)] Find an ideal in $\mathbb{Z}[X]$ that cannot be generated by 2-elements.
            \begin{proof}
                Consider the ideal $(4, 2x, x^{2})$. This ideal is not generated by one element, otherwise, the 3 polynomials would share a common root or they would all be constants, which is not the case. We can look at the ideals generated by the leading coefficients which gives $(4) \subset (2) \subset (1)$. Now suppose that this was generated by an ideal $(f, g)$ on two elements. We also consider the ideal generated by their leading coefficients which would be $(a) \subset (b)$. Now suppose that the degrees of $f$ and $g$ were less than or equal to $2$. Then it follows that assuming $(f, g) \subseteq (4, 2x, x^{2})$, we require that $a, b \in\{1, 2, 4\}$. But no matter how we chain the ideals such as:
                    \begin{equation*}
                        (4) \subseteq (2) \subseteq (2), (4) \subseteq (1) \subseteq (1), (2) \subseteq (1) \subseteq (1)
                    \end{equation*}
                we find that no generators can be found with the corresponding two coefficients that generate $(4, 2x, x^{2})$. So $(f, g) \not\supseteq (4, 2x, x^{2})$. If either had degree greater than or equal to  $2$, then we should be able to reduce them so that their degrees were less than or equal to $2$, assuming that $(4, 2x, x^{2}) \subseteq (f, g)$ and show that they are not equal by the same process.
            \end{proof}

        \item [(b)] Give an example of a prime ideal in $\mathbb{Z}[X]$ that is not maximal.
            \begin{proof}
                A prime ideal is $(x)$ because if $f \in (x)$, then $x \divides f$. So if $f = gh$, $h \notin (x)$, then $x \divides gh$ means that $x \divides g$. So $(x)$ is prime. But $(x) \subseteq (x, 2)$ and $(x, 2) \neq \mathbb{Z}[X]$. So $(x)$ is prime but not maximal since $2 \ndivides x$.
            \end{proof}

        \item [(c)] Show that every maximal ideal in $\mathbb{Z}[X]$ is generated by $2$-elements. 
            \begin{proof}
                Let $I$ be an ideal in $\mathbb{Z}[X]$. Consider the set of leading coefficients $S_{i}$ attached to the term $x^{i}$. Then we have:
                    \begin{equation*}
                        S_{0} \subseteq S_{1} \subseteq S_{2} \subseteq \cdots \subseteq S_{n} = S_{n + 1} \cdots
                    \end{equation*}
                Then it follows that if $P_{m} = \bigcup_{i = 0}^{m} S_{i}$, then $P_{m}$ is generated by one element as $\mathbb{Z}$ is a Euclidean domain. So if $P_{n} = (k)$, then $(x, k)$ contains the ideal $I$. If $k = 1$, we say $P_{n - 1} = (k^{\prime})$ and $(x, k^{\prime})$ contains $I$. 

                In either case, we have have found an ideal containing $I$ that is not the whole ring. Furthermore, if $k, k^{\prime}$ are not prime, we can find a prime dividing it called $p$ and $(x, k) \subseteq (x, p) \neq \mathbb{Z}[X]$. 

                Finally, maximal ideals cannot be one element ideals because either it is a constant and we can adjoin $x$, or it is a non constant polynomial, to which we can adjoin the constant to get a two element proper ideal. So this tells us that 

                So any maximal ideal has at least two elements, and any ideal with more than $2$ elements is contained in some proper ideal generated by two elements. So every maximal ideal is generated by $2$ elements.
            \end{proof}
    \end{itemize}

\textbf{Exercise 3}: Show that the rings $\mathbb{Z}[2i]$ and $\mathbb{Z}[\sqrt{-7}]$ are not integrally closed. Give explicit examples where unique factorization fails.
    \begin{proof}
        $\mathbb{Z}[2i]$ is not integrally closed because $1 + i = \frac{2 + 2i}{2} \in \mathbb{Q}[2i]$ is a root of the polynomial:
            \begin{equation*}
                x^{2} - 2i \in \mathbb{Z}[2i][x]
            \end{equation*}
        $\mathbb{Z}[\sqrt{-7}]$ is not integrally closed because we have that $\frac{1 + \sqrt{-7}}{2} \in \mathbb{Q}[\sqrt{-7}]$ but it is the root of the polynomial of 
            \begin{equation*}
                x^{2} + x + 1 - \sqrt{-7}
            \end{equation*}
        as
            \begin{align*}
                \left(\dfrac{1 + \sqrt{-7}}{2}\right)^{2} + \left(\dfrac{1 + \sqrt{-7}}{2}\right) + 1 - \sqrt{-7} &= \dfrac{1 + 2\sqrt{-7} -7}{4} + \dfrac{1 + \sqrt{-7}}{2} + 1 - \sqrt{-7} \\
                                                                                                                  &= \dfrac{-6 + 2\sqrt{-7}}{4} + \dfrac{2 + 2\sqrt{-7}}{2} + 1 - \sqrt{-7}  \\
                                                                                                                  &= \dfrac{-4 + 4\sqrt{-7}}{4} + 1 - \sqrt{-7}                              \\
                                                                                                                  &= -1 + \sqrt{-7} + 1 - \sqrt{-7}                                          \\
                                                                                                                  &= 0                                                                         
            \end{align*}
        Now for the example where unique factorization fails, we have that:
            \begin{equation*}
                2i \cdot2i = -4 = -1 \cdot2 \cdot2
            \end{equation*}
        We know that $2$ is prime because if $2 \divides(x + 2yi)(w + 2zi)$, then $2 \divides xw$. Since $2$ is prime in $\mathbb{Z}$, then $2 \divides x$ or $2 \divides w$. So $2$ divides either $(x + 2yi)$ or $(w + 2zi)$. We also know that $2i$ is prime because $\mathbb{Z}[2i]/(2i) = \mathbb{Z}$ which is an integral domain. Since $2i$ and $2$ do not differ by a unit, we have that unique factorization fails.

        For the other one, we have:
            \begin{equation*}
                (1 + \sqrt{-7})(1 - \sqrt{-7}) = 8 = 2 \cdot2 \cdot2
            \end{equation*}
        which is a decomposition of unequal numbers of irreducible elements.
    \end{proof}

\textbf{Exercise 4}: Let $G$ be the group of complex $n$-th roots of unity. Let $\zeta \in G$ act on $k[X, Y]$ by $\zeta \cdot X = \zeta X$ and $\zeta \cdot Y = \zeta^{-1}Y$. Show that there is an isomorphism $k[X, Y]^{G} = k[U, V, W]/(UW - V^{n})$.
    \begin{proof}
        Consider the term $X^{i}Y^{j}$. Then if $\varphi: G \actson k[X, Y]$ as defined above, we have:
            \begin{equation*}
                \varphi(X^{i}Y^{j}) = \varphi(X^{i})\varphi(Y^{j}) = \zeta^{i}\zeta^{-j}X^{i}Y^{j}
            \end{equation*}
        We require $\zeta^{i}\zeta^{-j} = \zeta^{i - j} = 1$. We have that $\zeta^{zn} = 1$ exactly when $z \in \mathbb{Z}$. So $i - j = zn$. So $i \equiv j \pmod{n}$. So this tells us that:
            \begin{equation*}
                k[X, Y]^{G} = k[X^{n}, XY, Y^{n}] = k[U, V, W]/\sim 
            \end{equation*}
        where $\sim $ is the relation, $UW = V^{n}$. So we have:
            \begin{equation*}
                k[X, Y]^{G} = k[U, V, W]/(UW - V^{n})
            \end{equation*}
        which concludes the proof.
    \end{proof}

\textbf{Exercise 5}: Write $X_{1}^{3} + X_{2}^{3}$ as a polynomial in $e_{1} = X_{1} + X_{2}$ and $e_{2} = X_{1}X_{2}$.
    \begin{proof}
        We first note that
            \begin{equation*}
                e_{1}^{3} = (X_{1} + X_{2})^{3} = X_{1}^{3} + 3X_{1}^{2}X_{2} + 3X_{1}X_{2}^{2} + X_{2}^{3}
            \end{equation*}
        Then 
            \begin{equation*}
                e_{1}e_{2} = (X_{1} + X_{2})X_{1}X_{2} = X_{1}^{2}X_{2} + X_{1}X_{2}^{2}
            \end{equation*}
        So
            \begin{equation*}
                3e_{1}e_{2} = 3X_{1}^{2}X_{2} + 3X_{1}X_{2}^{2}
            \end{equation*}
        and so
            \begin{equation*}
                e_{1}^{3} - 3e_{1}e_{2} = X_{1}^{3} + X_{2}^{3}
            \end{equation*}
        which is what we wanted.
    \end{proof}





















\end{document}as
