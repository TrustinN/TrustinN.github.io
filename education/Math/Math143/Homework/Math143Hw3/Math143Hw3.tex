%! TeX root = /Users/trustinnguyen/Downloads/Berkeley/Math/Math143/Homework/Math143Hw3/Math143Hw3.tex

\documentclass{article}
\usepackage{/Users/trustinnguyen/.mystyle/math/packages/mypackages}
\usepackage{/Users/trustinnguyen/.mystyle/math/commands/mycommands}
\usepackage{/Users/trustinnguyen/.mystyle/math/environments/article}

\title{Math143Hw3}
\author{Trustin Nguyen}

\begin{document}

    \maketitle

\reversemarginpar

\textbf{Exercise 1}: Let $R$ be a ring and let $I \subseteq R$ be an ideal.
    \begin{itemize}
        \item [(a)] Show there is a natural bijection between ideals in $R/I$ and ideals in $R$ containing $I$.
            \begin{proof}
                Consider the homomorphism $\varphi : R \rightarrow R/I$ given by the projection. Let $J$ be an ideal containing $I$, and consider the image of $\varphi(J)$ given by $j \in J$:
                    \begin{equation*}
                        \varphi(j) = j + I
                    \end{equation*}
                We will show that the image is an ideal. Since $J \supseteq I$, we must have a $j_{0} \in I$ which implies that $j_{0} \in J$. Then
                    \begin{equation*}
                        \varphi(j_{0}) = 0 + I
                    \end{equation*}
                The image of a homomorphism restricted to $J$ is closed under addition, which can be seen. Furthermore, if $r + I \in R/I$, then $(r + I)(j + I) = rj + I = \varphi(rj)$. Since $rj \in J$, we have that the ideal generated by the image of $J$ in this mapping is closed under multiplication from $R/I$. Notice that the image of $\varphi(J)$ is just $J/I$, since elements of the image are the cosets of $I$ with representatives in $J$. Therefore, $J/I$ is an ideal of $R/I$ that we map $J$ to.

                (Injectivity) We will check for injectivity. Suppose that two ideals containing $I$ from $R$, denoted $J_{1}, J_{2}$ map to the same ideal $J/I$. Suppose that $j \in J_{1}$ and $j \mapsto j + I$. We can find a $j^{\prime} \in J_{2}$ such that $j^{\prime} \mapsto j^{\prime} + I$ and 
                    \begin{equation*}
                        j + I = j^{\prime} + I
                    \end{equation*}
                But we have the following conclusions:
                    \begin{align*}
                        j + I &= j^{\prime} + I     \\
                        0 + I &= j^{\prime} - j + I   
                    \end{align*}
                So $j^{\prime} - j \in I$. But $I \subseteq J_{2}$, therefore, $j^{\prime} - j \in J_{2}$ and so $j \in J_{2}$. We conclude $J_{1} \subseteq J_{2}$. By the same argument, $J_{2} \subseteq J_{1}$, so $J_{1} = J_{2}$.

                (Surjectivity) Consider an ideal of $R/I$ which is generated by a number of cosets:
                    \begin{equation*}
                        J = (a_{1} + I, a_{2} + I, \ldots ) \in R/I
                    \end{equation*}
                Consider the ideal generated by the elements of the union of the cosets:
                    \begin{equation*}
                        J^{\prime} = (a_{1} + I \cup a_{2} + I \cup \ldots )
                    \end{equation*}
                ($J \subseteq \varphi(J^{\prime})$) Clearly, $J^{\prime}$ maps surjectively into $J$ by $\varphi$. We just take $a_{1} + i \mapsto a_{1} + i + I = a_{1} + I$. So we have a way to mapping to the generators. 

                ($\varphi(J^{\prime}) \subseteq J$) Now consider an arbitrary element of $J^{\prime}$ which is of the form:
                    \begin{equation*}
                        r_{1}(a_{1} + i_{1}) + r_{2}(a_{2} + i_{2}) + \ldots \mapsto r_{1}a_{1} + r_{2}a_{2} + \ldots + I
                    \end{equation*}
                which is an element of $J/I$. Therefore, we have that $\varphi(J^{\prime}) = J$ by our double inclusion proof, showing that $\varphi$ is surjective.
            \end{proof}

        \item [(b)] Show that the bijection in part (a) induces a bijection between radical ideals in $R$ and radical ideals in $R/I$.
            \begin{proof}
                We will show that radical ideals map to radical ideals. Consider the homomorphism given by the previous problem, where $\varphi(\sqrt{J}) \mapsto \sqrt{J}/I$. Suppose that $p^{k} + I = (p + I)^{k} \in \sqrt{J}/I$. We use the same construction which gave us the fact that $\varphi$ was surjective in the previous problem. Consider the ideal generated by the elements of the union of the cosets:
                    \begin{equation*}
                        J^{\prime} = (a_{1} + I \cup a_{2} + I \cup \ldots )
                    \end{equation*}
                We know that $p^{k} + I$ contains $p^{k}$. So $J^{\prime}$ contains $p^{k}$. From the injectivity inherited from problem $(a)$, we have that $J^{\prime} \mapsto \sqrt{J}/I$ and $\sqrt{J} \mapsto \sqrt{J}/I$, therefore, $p^{k} \in \sqrt{J}$. We conclude that $p \in \sqrt{J}$, therefore, $p + I \in \sqrt{J}/I$. So we have that radical ideals map to radical ideals. This part of the proof has also shown that the pre-image of a radical ideal must also be a radical ideal. Therefore, our map is surjective.
            \end{proof}

        \item [(c)] Show that the bijection in part (a) induces a bijection between maximal ideals in $R$ and maximal ideals in $R/I$. Conclude that if there is a surjection $\varphi: R \rightarrow L$ where $L$ is a field, then the kernel of $\varphi$ is a maximal ideal. 
            \begin{proof}
                We will show that maximal ideals map to maximal ideals. Let $I \subseteq J \subseteq R$ where $J$ is a maximal ideal. Suppose that $b + I \in J/I$. We will show that $J/I + (b + I) = R/I$, which would allow us to conclude that $J/I$ maximal. Since $b + I \notin J/I$, we must have $b \notin J$ otherwise, we would have, by the mapping we established in part (a):
                    \begin{equation*}
                        \varphi(b) = b + I \in J/I
                    \end{equation*}
                a contradiction. Since $J$ is maximal, we have that the ideal $R^{\prime} \supseteq J$ containing $b$ must be the entire ring. We can write $1$ as:
                    \begin{equation*}
                        a_{1}j + a_{2}b = 1
                    \end{equation*}
                To which we see:
                    \begin{equation*}
                        R/I = 1 + I = \varphi(a_{1}j + a_{2}b) = a_{1}j + I + a_{2}b + I = J/I + (b + I)
                    \end{equation*}
                Therefore showing $J/I$ is maximal. Now to see surjectivity, suppose $J/I$ is maximal in $R/I$. Then $R/I/J/I \cong R/J$ which is a field. So $J$ is maximal and 
                    \begin{equation*}
                        \varphi(J) = J/I
                    \end{equation*}
                So there is a maximal ideal that maps to $J/I$. Injectivity is inherited from $\varphi$.
            \end{proof}
    \end{itemize}

\textbf{Exercise 2}: Practice with maximal ideals:
    \begin{itemize}
        \item [(a)] Let $I \subseteq k[x_{1}, \ldots, x_{n}]$ be an ideal. Show that $I$ is radical if and only if it is equal to the intersection of all the maximal ideals containing $I$.
            \begin{proof}
                ($\rightarrow $) Suppose $I = \sqrt{I}$. We will show that $I = \bigcap_{}^{} M_{i}$ for $M_{i}$ maximal ideals containing $I$. We have that $I \subseteq \bigcap_{}^{} M_{i}$, which was given by the problem statement. Since $k$ is algebraically closed, by one of the Weak Nullstellensatz's, we can conclude that maximal ideals correspond algebraic sets that are points. Consider the points that are killed by $I$, which can be extracted by the fact that
                    \begin{equation*}
                        I = I(V(I)) = \sqrt{I}
                    \end{equation*}
                Then we know that $V(I) = \{p_{1}, p_{2}, \ldots \}$. So we are considering the ideal that kills all $p_{i}$. But that is just the intersection of the ideal generated by each point as an algebraic set. So we have
                    \begin{equation*}
                        \bigcap_{i}^{} I(p_{i}) \subseteq I(V(I))
                    \end{equation*}
                Since the $I(p_{i})$'s are maximal ideals, we conclude that
                    \begin{equation*}
                        \bigcap_{i}^{} M_{i} \subseteq I(V(I)) = I
                    \end{equation*}
                Note that it does not matter if there are more maximal ideals, since the intersection will be smaller and still a subset. We have shown a double inclusion, which finishes the proof.

                ($\leftarrow $) Suppose that $I = \bigcap_{}^{} M_{i}$ for $M_{i}$ maximal ideals. Since maximal ideals are prime and prime ideals are maximal, we have that $M_{i}$'s are radical. Since $I \subseteq M_{i}$, we must also have that if $p^{k} \in I$, then $p^{k} \in M_{i}$ and therefore, $p \in M_{i}$. But that means that $\sqrt{I} \subseteq M_{i}$. So $\sqrt{I} \subseteq \bigcap_{}^{} M_{i}$ We have the string of inclusions $I \subseteq \sqrt{I} \subseteq \bigcap_{}^{} M_{i}$. But since $\bigcap_{}^{} M_{i} = I$, we have $I = \sqrt{I}$.
            \end{proof}

        \item [(b)] Show that the radical of the ideal $I = (x^{2} - 2xy^{4} + y^{6}, y^{3} - y) \subseteq \mathbb{C}[x, y]$ is the intersection of three maximal ideals.
            \begin{answer}
                To get the radical ideal, by the Nullstellensatz, we can take 
                    \begin{equation*}
                        I(V((x^{2} - 2xy^{4} + y^{6}, y^{3} - y)))
                    \end{equation*}
                So
                    \begin{align*}
                        V((x^{2} - 2xy^{4} + y^{6}, y^{3} - y)) &= V((x^{2} - 2xy^{4} + y^{6}) + (y^{3} - y))       \\
                                                                &= V((x^{2} - 2xy^{4} + y^{6})) \cap V((y^{3} - y)) \\
                                                                &= (0, 0) \cup (1, 1) \cup (1, -1)                    
                    \end{align*}
                Now we have
                    \begin{equation*}
                        I((0, 0) \cup (1, 1) \cup (1, -1)) = I((0, 0)) \cap I((1, 1)) \cap I((1, -1))
                    \end{equation*} 
                By one of the Weak Nullstellensatz theorems, we have that ideals generated by points as algebraic sets correspond to the maximal ideal of $\mathbb{C}[x, y]$. So we are done.
            \end{answer}
    \end{itemize}

\textbf{Exercise 3}: Let $X = V(x^{2} - yz, xz - x) \subseteq \mathbb{A}_{\mathbb{C}}^{3}$. Find the irreducible components of $X$ and their corresponding prime ideals. Make sure you justify your solution.
    \begin{answer}
        We start by solving the equations for $0$:
            \begin{align*}
                x^{2} - yz &= 0 \\
                xz - x     &= 0   
            \end{align*}
        So we have that either $x = 0$ or $z = 1$. In the case of $x = 0$, we have
            \begin{equation*}
                -yz = 0
            \end{equation*}
        So we have
            \begin{align*}
                V(x, yz) &= V(x) \cap V(yz) \\
                &= V(x) \cap (V(y) \cup V(z)) \\
                &= (V(x) \cap V(y)) \cup (V(x) \cap  V(z)) \\
                &= V(z) \cup V(y)
            \end{align*}
        The second case is when $z = 1$, so we get $V(z - 1, x^{2} - y)$. But since $x^{2} - y$ is irreducible in $\mathbb{A}_{\mathbb{C}}^{2}$, this is irreducible. Therefore, the decomposition is
            \begin{equation*}
                X = V(z) \cup V(y) \cup V(z - 1, x^{2} - y)
            \end{equation*}
        We have $I(V(z))$ is $\sqrt{(z)}$ which is just $(z)$. The same reasoning gives us $I(V(y)) = (y)$. Finally, the last algebraic set is the parabola on the $z = 1$ $xy$-plane. We know that an algebraic set is irreducible if and only if its ideal is prime. Therefore, $I(V(z - 1, x^{2} - y))$ is prime. But that is just $\sqrt{(z - 1, x^{2} - y)}$. So we are done.
    \end{answer}

\textbf{Exercise 4}: Practice with field extensions:
    \begin{itemize}
        \item [(a)] Let $k \subseteq L$ be a field extension. Show that the set of elements in $L$ that are algebraic over $k$ form a subfield of $L$ containing $k$. (Hint: suppose $v^{n} + a_{1}v^{n - 1} + \ldots + a_{n} = 0$) with $a_{n} \neq 0$. Notice that $v(v^{n - 1} + \ldots) = -a_{n}$.
            \begin{proof}
                Suppose that $\alpha$ is algebraic in $k$. Then we have some polynomial
                    \begin{equation*}
                        f(x) = k_{n}x^{n} + k_{n - 1}x^{n - 1} + \cdots + k_{1}x + k_{0} = \sum_{i = 0}^{n} k_{i}x^{i}
                    \end{equation*}
                where $k_{n} \neq 0$ such that $f(\alpha) = 0$. Then we have
                    \begin{equation*}
                        g(\alpha^{-1}) = \alpha^{-n} \cdot f(\alpha) = 0
                    \end{equation*}
                where $g$ is also a polynomial in $k[x]$. So if an element is algebraic, its multiplicative inverse is also algebraic. We can also consider the additive inverse $-\alpha$. If we take $f^{\prime}$ preserve the coefficients $k_{i}$ of $f$ but change the parity for odd powers of $x$, we have
                    \begin{equation*}
                        f^{\prime} = \sum_{i = 0}^{n} k_{i}(-1)^{i}x^{i}
                    \end{equation*}
                and
                    \begin{equation*}
                        f^{\prime}(-\alpha) = 0
                    \end{equation*}
                So the additive inverse is also algebraic over $k$. To prove that it is a subring with inverses, we also need to show that the set is closed under addition/multiplication. If $\alpha, \beta$ are algebraic, we have that 
                    \begin{equation*}
                        1, \alpha, \alpha^{2}, \ldots , \alpha^{n-1}
                    \end{equation*}
                for some $n - 1$ forms a basis of the extension $k[\alpha]$. And likewise for $\beta$:
                    \begin{equation*}
                        1, \beta, \beta^{2}, \ldots , \beta^{n - 1}
                    \end{equation*}
                So we also notice that we will have a finite number of linearly independent and spanning basis elements for $k[\alpha, \beta]$:
                    \begin{align*}
                        \begin{array}{ c c c c c c }
                            1             & \alpha              & \alpha^{2}              & \alpha^{3}              & \ldots  & \alpha^{n - 1}              \\
                            \beta         & \alpha\beta         & \alpha^{2}\beta         & \alpha^{3}\beta         & \ldots  & \alpha^{n - 1}\beta         \\
                            \beta^{2}     & \alpha\beta^{2}     & \alpha^{2}\beta^{2}     & \alpha^{3}\beta^{2}     & \ldots  & \alpha^{n - 1}\beta^{2}     \\
                            \vdots        & \vdots              & \vdots                  & \vdots                  & \ddots  & \vdots                      \\
                            \beta^{m - 1} & \alpha\beta^{m - 1} & \alpha^{2}\beta^{m - 1} & \alpha^{3}\beta^{m - 1} & \ldots  & \alpha^{n - 1}\beta^{m - 1}   
                        \end{array}
                    \end{align*}
                So we take the powers of $\alpha\beta$ or any other element of the array:
                    \begin{equation*}
                        1, \alpha\beta, (\alpha\beta)^{2}, \ldots , (\alpha\beta)^{nm}
                    \end{equation*}
                This list has $nm + 1$ elements, but our basis has $nm$ elements. So our list must be linearly dependent. Therefore, we have a finite extension. Therefore, the product of algebraic elements of $k$ is also algebraic. As for the sum, we define its powers by:
                    \begin{equation*}
                        (\alpha + \beta)^{n} = \sum_{k = 0}^{n} \dbinom{n}{k}\alpha^{k}\beta^{n - k}
                    \end{equation*}
                which can be written all can be written as a combination of elements of our basis. By the same argument, we take $nm + 1$ powers and get a linearly dependent list. So we can conclude that $\alpha + \beta$ is algebraic over $k$ also. So we have that if $k \subseteq L$ is a field extension, then the set of elements of $L$ that are algebraic over $k$ is a subfield.
            \end{proof}

        \item [(b)] Suppose $L$ is a finite extension of $k$ and $k \subseteq R \subseteq L$ for a ring $R$. Prove that $R$ is a field.
            \begin{proof}
                Finite extensions are algebraic. Therefore, every element in $L$ is algebraic over $k$. But this means that $R$ is a subset of $L$ algebraic over $k$. By the previous problem, we have that there must be inverses for every algebraic element of $R$. So every element in $R$ has an inverse except $0$, therefore showing that $R$ is a field.
            \end{proof}
    \end{itemize}

\textbf{Exercise 5}: Suppose $k \subseteq L$ is an algebraic extension, and $L \subseteq L^{\prime}$ is an algebraic extension. Prove that $k \subseteq L^{\prime}$ is an algebraic extension. (Hint: If $\alpha \in L^{\prime}$ is algebraic over $L$, then there exist $c_{i} \in L$ such that $\alpha^{n} = c_{0} + c_{1}\alpha + \ldots  + c_{n - 1}\alpha^{n - 1}$. Then show that $\alpha$ and $c_{0}, \ldots , c_{n - 1}$ are contained in a finite extension of $k$.)
    \begin{proof}
        Let $\alpha \in L^{\prime}$ be arbitrary. Then we know that
            \begin{equation*}
                \alpha^{n} = c_{0} + c_{1}\alpha + \ldots  + c_{n - 1}\alpha^{n - 1}
            \end{equation*}
        for some $c_{i} \in L$. Observe that since each $c_{i}$ are algebraic over $k$, we have that a finite extension of $k$ with any $c_{i}$ induces a finite vector space over $k$ since we have
            \begin{equation*}
                c_{i}^{m}k_{m} + c_{i}^{m - 1}k_{m - 1} + \ldots + c_{1}k_{1} + k_{0} = 0
            \end{equation*}
        for some $k_{j} \in k$, as $c_{i}$'s are algebraic over $k$. Now we consider all combinations of the products of $\alpha, c_{n - 1}, \ldots , c_{0}$ of the form
            \begin{equation*}
                \alpha^{b_{1}}c_{n - 1}^{b_{2}}\cdots c_{0}^{b_{n + 1}}
            \end{equation*}
        There are a finite number of such products, the collection of which forms a basis over the extension $k[\alpha, c_{n - 1}, \ldots , c_{0}]$. Therefore, we have that $L^{\prime}$ is an algebraic extension over $k$ because any element is part of some finite and therefore algebraic extension of $k$.
    \end{proof}

















\end{document} 
