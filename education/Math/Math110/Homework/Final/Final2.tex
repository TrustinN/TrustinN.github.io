%! TeX root =

\documentclass{article}
\usepackage{/Users/trustinnguyen/MyStyle/mystyle}

\title{Extra Problems on Jordan Normal Form}
\author{Trustin Nguyen}


\begin{document}
\maketitle
\reversemarginpar

\textbf{Exercise 1}: Let $V$ be a complex $n$-dimensional space and let $T \in \mathcal{L}(V)$ be such that $\ker{T^{n - 3}} \neq \ker{T^{n - 2}}$. How many distinct eigenvalues can $T$ have?
    \begin{proof}
        Notice that the kernel of $T^{n - 2}$ is non-trivial, as it is a superset of $\ker{T^{n - 3}}$. This means that there is a Jordan Block for the eigenvalue of 0 and additionally, there exists at least one block of size $n - 2 \times n - 2$. This tells us that for $n$ larger than or equal to 4, there is exactly one Jordan Block of size $n - 2 \times n - 2$ for the eigenvalue 0 and this means that the maximum number of distinct eigenvalues is 3 as 0 is an eigenvalue and that we can add two more eigenvalues corresponding to linearly independent vectors. This corresponds to a maximum of $n$ eigenvectors.  
    \end{proof}
\textbf{Exercise 2}: Let $V$ be a complex finite-dimensional vector space and let $T \in \mathcal{L}(V)$ have eigenvalues $-1, 0, 1$. Given the dimensions of the corresponding null spaces below, determine the Jordan normal form of $T$.
\begin{align*}
    \begin{array}{ | c | c | c | c | c | c | }
        \lambda & \dim{\ker{(T - \lambda I)}} & \dim{\ker{(T - \lambda I)^{2}}} & \dim{\ker{(T - \lambda I)^{3}}} & \dim{\ker{(T - \lambda I)^{4}}} & \dim{\ker{(T - \lambda I)^{5}}} \\
        -1      & 3                          & 5                               & 6                               & 6                               & 6                               \\
        0       & 2                          & 4                               & 6                               & 7                               & 7                               \\
        1       & 3                          & 4                               & 5                               & 5                               & 5                               \\
    \end{array}
\end{align*}
    \begin{proof}
        We see that for the eigenvalue of $-1$, there are a total of $3$ Jordan blocks, since each Jordan block contributes a dimension of 1 to the null space of $T - \lambda I$. For the next column, there is are 2 Jordan blocks with dimension greater than 1, because the Jordan blocks with size $1 \times 1$ stabilize at $\ker{(T - \lambda I)}$ while the blocks with greater dimension do not stabilize. Now for the next column, there is one block that has dimension greater than $2 \times 2$ since $6 - 5 = 1$. Repeating the same process, we deduce that there are 2 Jordan blocks for eigenvalue 0, one of which has dimension 3, and the other dimension 4. For eigenvalue 1, there are three Jordan blocks, two having dimenision $1 \times 1$ and one having dimension $3 \times 3$. We put this in a matrix:
        \begin{align*}
            \text{JNF}(T) = 
                \begin{bmatrix}
                    -1 &    &    &    &    &    &   &   &   &   &   &   &   &   &   &   &   &   \\
                       & -1 & 1  &    &    &    &   &   &   &   &   &   &   &   &   &   &   &   \\
                       & 0  & -1 &    &    &    &   &   &   &   &   &   &   &   &   &   &   &   \\
                       &    &    & -1 & 1  & 0  &   &   &   &   &   &   &   &   &   &   &   &   \\
                       &    &    & 0  & -1 & 1  &   &   &   &   &   &   &   &   &   &   &   &   \\
                       &    &    & 0  & 0  & -1 &   &   &   &   &   &   &   &   &   &   &   &   \\
                       &    &    &    &    &    & 0 & 1 & 0 &   &   &   &   &   &   &   &   &   \\
                       &    &    &    &    &    & 0 & 0 & 1 &   &   &   &   &   &   &   &   &   \\
                       &    &    &    &    &    & 0 & 0 & 0 &   &   &   &   &   &   &   &   &   \\
                       &    &    &    &    &    &   &   &   & 0 & 1 & 0 & 0 &   &   &   &   &   \\
                       &    &    &    &    &    &   &   &   & 0 & 0 & 1 & 0 &   &   &   &   &   \\
                       &    &    &    &    &    &   &   &   & 0 & 0 & 0 & 1 &   &   &   &   &   \\
                       &    &    &    &    &    &   &   &   & 0 & 0 & 0 & 0 &   &   &   &   &   \\
                       &    &    &    &    &    &   &   &   &   &   &   &   & 1 &   &   &   &   \\
                       &    &    &    &    &    &   &   &   &   &   &   &   &   & 1 &   &   &   \\
                       &    &    &    &    &    &   &   &   &   &   &   &   &   &   & 1 & 1 & 0 \\
                       &    &    &    &    &    &   &   &   &   &   &   &   &   &   & 0 & 1 & 1 \\
                       &    &    &    &    &    &   &   &   &   &   &   &   &   &   & 0 & 0 & 1 \\
                \end{bmatrix}
        \end{align*}
    \end{proof}
\textbf{Exercise 3}: Let $T \in \mathcal{L}(\mathcal{P}_{3}(\mathbb{C}))$ be the operator
    \begin{equation*}
        T : f(x) \mapsto f(x - 1) + x^{3}f^{\prime\prime\prime}(x)/3
    \end{equation*}
Find the Jordan normal form and a Jordan basis for $T$.
    \begin{proof}
        We first look at the action of $T$ on our basis vectors $\{1, x, x^{2}, x^{3}\}$:
            \begin{align*}
                T: 1 &\mapsto 1 \\
                T: x &\mapsto x - 1 \\
                T: x^{2} &\mapsto (x - 1)^{2} \\
                T: x^{3} &\mapsto (x - 1)^{3} + 2x^{3}
            \end{align*}
        It can be seen that our matrix representation with respect to this basis is 
            \begin{align*}
                T = 
                \begin{bmatrix}
                    1 & -1 & 1  & -1 \\
                    0 & 1  & -2 & 3  \\
                    0 & 0  & 1  & -3 \\
                    0 & 0  & 0  & 3  \\
                \end{bmatrix}
            \end{align*}
        So the Jordan normal form to aim for is 
            \begin{align*}
                \begin{bmatrix}
                    1 & 1 & 0 & 0 \\
                    0 & 1 & 1 & 0 \\
                    0 & 0 & 1 & 0 \\
                    0 & 0 & 0 & 3 \\
                \end{bmatrix}
            \end{align*}
    \end{proof}
\textbf{Exercise 4}: Let $V$ be a complex (finite-dimensional) vector space and let $T \in \mathcal{L}(V)$. Prove that there exist operators $D$ and $N$ in $\mathcal{L}(v)$ such that $T = D + N$, $D$ is diagonalizable, $N$ is nilpotent, and $DN = ND$.
    \begin{proof}
        Clearly, there exists a Jordan normal form for $T$, so we have that 
            \begin{align*}
                \text{JNF}(T) =
                \begin{bmatrix}
                    \lambda_{1} & 1      & 0           &             &        &             &        &             &             &             &  &  &  \\
                    0           & \ddots & 1           &             &        &             &        &             &             &             &  &  &  \\
                    0           & 0      & \lambda_{1} &             &        &             &        &             &             &             &  &  &  \\
                                &        &             & \lambda_{2} & 1      & 0           &        &             &             &             &  &  &  \\
                                &        &             & 0           & \ddots & 1           &        &             &             &             &  &  &  \\
                                &        &             & 0           & 0      & \lambda_{2} &        &             &             &             &  &  &  \\
                                &        &             &             &        &             & \ddots &             &             &             &  &  &  \\
                                &        &             &             &        &             &        & \lambda_{n} & 1           & 0           &  &  &  \\
                                &        &             &             &        &             &        &             & \lambda_{n} & 1           &  &  &  \\
                                &        &             &             &        &             &        &             & 0           & \lambda_{n} &  &  &  \\
                                &        &             &             &        &             &        &             &             &             &  &  &  \\
                                &        &             &             &        &             &        &             &             &             &  &  &  \\
                                &        &             &             &        &             &        &             &             &             &  &  &  \\
                \end{bmatrix}
            \end{align*}
        which has a diagonal component and 1's on the top. Observe that $N$ is upper triangular with $0^{\prime}$s on the diagonal. This tells us that the minimal polynomial has only the root $0$ where $z$ is the only factor. Therefore, we have that $P(N) = N^{k} = 0$ for some $k$. So therefore, $N$ is nilpotent. 
    \end{proof}
\textbf{Exercise 5}: Suppose the Jordan form of an operator $T \in \mathcal{L}(V)$ consists of Jordan blocks of size $3 \times 3$, $4 \times 4$, $1 \times q$, $5 \times 5$, $2 \times 2$, corresponding to eigenvalues $\lambda_{1}, \lambda_{2}, \lambda_{3}, \lambda_{2}, \lambda_{1}$, repectively. Assuming that $\lambda_{i} \neq \lambda_{j}$ for $i \neq j$, find the minimal polynomial of $T$.
    \begin{proof}
        To find the minimal polynomial, we note that we will have factors $(z - \lambda_{1}), (z - \lambda_{2}), (z - \lambda_{3})$. We choose the multiplicity corresponding the the maximal block size. Therefore, our $p_{\text{min}}$ is 
        \begin{equation*}
            p_{\text{min}} = (z - \lambda_{1})^{3}(z - \lambda_{2})^{5}(z - \lambda_{3})
        \end{equation*}
    \end{proof}
\end{document}


